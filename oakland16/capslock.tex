%%%%%%%%%%%%%%%%%%%%%%%%%%%%%%%%%%%%%%%%%%%%%%%%%%%%%%%%%%%%%%%%%%%%%%%%%%%%%%%%
\subsection{Sanitizing Caps-Lock Errors}
\label{app:caps-lock-err}

As mentioned in the paper body, preliminary analysis of the data revealed that a large
fraction of errors was caused by accidental pressing of the caps-lock
key. Measuring the rate of caps-lock errors is more challenging than
for other typos, for two reasons. First, caps-lock key presses are not
recordable via keystroke logging, and thus not directly detectable
remotely.  Second, if the user engages the caps-lock key while typing
one password, there's a chance that it will remain on (erroneously)
while the next one is entered. In MTurk, if an individual worker is to
be permitted to enter more than one password---even across multiple
HITs---propagation of caps-lock typos across passwords is therefore
methodologically unavoidable. This second issue accounts for the
(artificially) high rate of caps-lock typos observed in our
experiments. We found that 76 HITs contributed to 1120 caps-lock
errors.

To adjust for the effect of such propagation errors in determining the rate of
caps-lock typos, we do the following. We define a caps-lock error as an
incorrect password which, when the cases of all the letters are
inverted, becomes correct. In sequentially processing the passwords in a HIT, we
use a variable {\sf CL-ERR} $\in \{{\tt 0}, {\tt 1}\}$ to denote a heuristic
determination as to whether the caps lock is in an error state 
when the user entered a password in a HIT. (An error state could either be that 
caps lock is on and the user should have typed lower-case letters, or caps lock is
off and the user should have typed upper case letters.) We initially let {\sf CL-ERR} =
${\tt 0}$. When we detect a caps-lock error in a password, we record it and
set {\sf CL-ERR} = ${\tt 1}$. If it is already the case that {\sf CL-ERR} =
${\tt 1}$ when we reach a password in a HIT, we discard the password. That is,
in such cases, we do not count it in our computation of error rates for any
typo. Additionally, for every password in a HIT, we determine (heuristically)
whether the caps lock has been turned off during entry of the password. If the
password contains at least one letter and the password was submitted correctly, 
then we set {\sf CL-ERR} = ${\tt 0}$. 

In general, the intuition here is that we keep track heuristically of whether
the caps-lock key appears to be engaged erroneously. If the entry of a password
in a HIT has been affected by the state of the erroneously engaged caps-lock
key, we treat it as ``tainted,'' and thus discard it from our experiment. (We
assume heuristically that caps-lock errors are independent of other typos. The
global effect of discarding ``tainted'' passwords and not recording typos they
contain in addition to caps-lock errors is small in any case.)

% Measuring the rate of such errors is more challenging than for other
% typos, for two particular reasons. First, caps lock key presses are
% not remotely detectable. We use a heuristic rule to determine whether
% the caps lock key was on during the entry of a password: If the
% password could be corrected by inverting the case of alphabetic
% characters, we deem the caps lock to have been on. A second difficulty
% is that caps lock engagement is the only form of typographical error
% that propagates across passwords.  We therefore estimate the rate of
% caps-lock-induced typos as follows. We treat every ``eligible''
% password as an independent experiment

% On investigating it in detail, we
% discovered an anomaly in our experimental approach, which is causing
% such a high count for caps lock related errors. The caps lock error,
% if not corrected, will spread over all the passwords in the HIT, and
% will be counted towards as caps lock error, even though this is not a
% realistic scenario for website login.  There are \fixme{70} HITs which
% contributes to \fixme{1000} caps lock errors. To remove this error, we
% adjusted our count of typos in the following way.  Whenever there is a
% continuous run of caps lock errors, we only considered the first
% password as a caps lock error and all the rest are considered as
% correct typing.  In figref{fig:top10-typo} we present both the
% percentages before the adjustment (second from the right column) and
% after the adjustment (right most column).

%%% Local Variables:
%%% mode: latex
%%% TeX-master: "main"
%%% End:
