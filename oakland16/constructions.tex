%%%%%%%%%%%%%%%%%%%%%%%%%%%%%%%%%%%%%%%%%%%%%%%%%%%%%%%%%%%%%%%%%%%%%%%%%%%%%%%%
\section{Building Approximate Password Checkers}
\label{sec:constructions}

Our final goal is to create an authentication mechanism that accepts typo in the
passwords. The major challenge in building a system like that is the
unavailability of the original password at the time of authentication.  The
cryptographic hash functions are built to make sure that no plain text operation
can be translated to any operation on the hash digest. Hence, it is difficult to
figure out whether or not the entered password is a typo.   

\begin{itemize}
\item It will require more traction to modify the existing user data.
\item Moreover, changing any hashing scheme will affect the security of stored
  passwords, and we don't want to change the offline security (i.e., if the
  server data gets leaked our typo tolerance should have any effect on the security.)
\item So, the only option left is generating a set of candidate ``real
  passwords'' ($r_w$) for each of the entered password $t_w$.
\item Now question is, how to generate these candidates given a trained typo
  model, which is a weight matrix of edits.
\end{itemize}





%%% Local Variables:
%%% mode: latex
%%% TeX-master: "main"
%%% End:
