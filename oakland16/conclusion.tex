%%%%%%%%%%%%%%%%%%%%%%%%%%%%%%%%%%%%%%%%%%%%%%%%%%%%%%%%%%%%%%%%%%%%%%%%%%%%%%%%
\section{Conclusion}
\label{sec:conclusion}

We presented the first treatment of typo-tolerant password authentication. We
demonstrated, with large-scale, real-world experiments, that password typos are
a real and common source of user errors in authentication systems. We found that
a few types of typo-corrections account for an overwhelming number of password
typos. We provided a formal framework for exploring typo-tolerant password
checkers, and focused on a class of them called relaxed checkers that are
backwards-compatible with existing password hashing schemes. We showed, via what
we call the free corrections theorem, that there exist relaxed checkers against
which the best attack performs no better than the best attack against an exact
checker. Unfortunately the construction requires exact knowledge of the password
distribution. We therefore gave a number of practical typo-tolerant checkers
inspired by it,  and analyzed their security empirically, showing that one can
easily obtain significant utility improvement with minimal or no security
degradation.

In future work, we plan to investigate whether typo-tolerance will actually
serve to improve overall security. Because allowing for password typos
increases login success rates in benign scenarios, it may help to make
adversarial login attempts stick out.  This would strengthen the signals used to
detect online password attacks as used in Internet-scale authentication systems. 


