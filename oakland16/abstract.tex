\begin{abstract} 
  We provide the first treatment of typo-tolerant password authentication for
  arbitrary user-selected passwords. Such a system, rather than simply rejecting
  a login attempt with an incorrect password, tries to correct
  common typographical errors on behalf of the user. Limited forms of
  typo-tolerance have been used in some industry settings, but to date there has
  been no analysis of the utility and security of such schemes.

  We quantify the kinds and rates of typos made by users via studies conducted
  on Amazon Mechanical Turk and via instrumentation of the production login
  infrastructure at Dropbox. The instrumentation at Dropbox did not record user
  passwords or otherwise change authentication policy, but recorded only the
  frequency of observed typos. Our experiments reveal that almost 10\% of
  failed login attempts fail due to a handful of simple, easily correctable typos, such
  as capitalization errors.  We show that correcting just a few of these typos
  would reduce login delays for a significant fraction of users as well as
  enable an additional 3\% of users to achieve successful login.

  We introduce a framework for reasoning about typo-tolerance, and
  investigate the seemingly inherent tension here between security and
  usability of passwords. We use our framework to show that there
  exist typo-tolerant authentication schemes that can get corrections
  for ``free'': we prove they are as secure as schemes that always
  reject mistyped passwords. Building off this theory, we detail a
  variety of practical strategies for securely implementing
  typo-tolerance.
%Our results in this paper have already triggered a change to the Dropbox operational
%environment (a caps lock indicator, rather than a typo corrector) that has
%evidenced significant usability improvements.
\end{abstract}


%%% Local Variables:
%%% mode: latex
%%% TeX-master: "main"
%%% End:
