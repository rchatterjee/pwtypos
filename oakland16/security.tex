%%%%%%%%%%%%%%%%%%%%%%%%%%%%%%%%%%%%%%%%%%%%%%%%%%%%%%%%%%%%%%%%%%%%%%%%%%%%%%%%
\section{Practical Typo-Tolerant Checkers \\and their Security}
\label{sec:security}


In the previous section, we presented an optimal checker \opchecker
that achieves the maximum acceptance utility that is achievable with
no loss in security (relative to optimal attacker). Unfortunately,
\opchecker is hard in general to implement, as it requires exact
knowledge of the distribution pair $(\pwprob, \typoprob)$, which is
not practically obtainable in most settings.

Here, we explore checkers that do not rely on exact distribution knowledge and
are simple to implement. The first tries all corrections in some
checker set. The latter two incorporate heuristics to try to avoid balls with
high aggregate mass; these are directly inspired by the results regarding
$\opchecker$.  As we show experimentally, our checkers can achieve high
acceptance utility with minimal security degradation, and the heuristics help
reduce security loss even 
against adversaries with exact knowledge of the probability distribution~$\pwprob$.  We also investigate the security of these checkers
against more realistic adversaries that must themselves estimate the 
distribution~$\pwprob$. For these adversaries our 
results here suggest that typo tolerance does
not really help adversaries at all because of the difficulty of getting
estimates right.

\paragraph{The tolerant checkers.} 
For the following, let $\pwtypo$ denote the input to the checker and
$\hatball(\pwtypo)$ the ball of potential passwords to check as defined by the
set of correctors $\typoset$ for the checker. Presented in increasing order of
sophistication (and similarity to $\opchecker$), the checkers are:

\begin{newitemize}
\item \emph{Check-always construction} ($\checkerall$): This checker checks all passwords in
  $\hatball(\pwtypo)$. Among the three checkers presented here, it achieves the greatest acceptance utility---and,
  conversely, the largest potential security degradation.
  
\item \emph{Blacklist construction} ($\checkerbl$). This checker uses a blacklist $L$ of
  (ostensibly high-probability) passwords. It checks $\pwtypo$ and
  every other password $\pw \in \hatball(\pwtypo)$ such that
  $\pw \not\in L$. Blacklisting in $\checkerbl$ aims to prune or
  eliminate non-singleton balls with high aggregate probability (as
  \opchecker does).  In our experiments, we use for the blacklist the 1,000 most popular
  passwords in RockYou, although one could use other blacklists as
  well, such as Twitter's banned password list~\cite{twitterbanned}.\footnote{We
  emphasize that the blacklist is only used for typo corrections: we do not
  assume users are restricted from registering blacklisted passwords.}
  %We know this black listing is very specific to
  %RockYou, and might not generalize. Also, there are other ways of
  %creating blacklist, but we leave exploration of that to future work.

\item \emph{Approximately optimal construction} ($\checkerapprox$). 
  This checker heuristically approximates
  \opchecker. It estimates the distribution $\pwprob$ of passwords
  using the empirical distribution of the RockYou password leak, and
  the distribution $\typoprob$ of typos using the empirical
  distribution learned from our MTurk study
  (see~\figref{fig:top10-typo}). We denote these empirically derived
  distributions respectively by $\epwprob$ and~$\etypoprob$. $\checkerapprox$ 
  computes $\ball(\pwtypo)$ using the
  constraints used by $\opchecker$ (see the last section), but under the empirical
  distribution pair $(\epwprob,\etypoprob)$, rather than the
  (generally unknown) true distribution pair $(\pwprob,\typoprob)$.
  We set $q=10^3$ for our experiments with $\checkerapprox$. We note that for
  the correction set sizes we consider, $\chbudget \le 5$,  
  solving the optimization problem is fast, as only $2^\chbudget$ possibilities for $\ball(\pwtypo)$ must be considered.
\end{newitemize}
We will investigate these checkers for typo correction sets $\toptwo =
\{\swcall,\swcfirst\}$, $\topthree = \toptwo \cup \{\rmlast\}$ and $\topfive =
\topthree \cup \{\rmfirst,\dtoslast\}$. In terms of utility, we know from the
second Dropbox study (\secref{sec:dropbox}) the improvements obtained when using 
$\checkerall$ with $\topthree$. The other two constructions will obtain slightly less utility due
to the fact that some corrections will not be checked. 

Our preliminary analysis, however, suggests that this utility reduction will be
slight: both strategies, by design, prevent corrections only to popular
passwords, which are rarely induced by typos in the first place (see
\secref{sec:mturk}).  For example, we can simulate acceptance utility for a
given checker as defined in \secref{sec:formal} by letting $\pwprob$ be defined
to be the RockYou empirical distribution and $\typoprob$ to be the empirical
frequencies of typo types observed.  Then for $\topthree$  we have that the
blacklist and approximately optimal strategies only reduce utility by 0.03
percentage points and 0.08 percentage points, respectively.\footnote{The
absolute acceptance utilities for $\topthree$ in these simulations are 0.9628,
0.9625, and 0.9620. But the low overall rate of typos in the MTurk experiments
means that exact checking here obtains 0.9564 acceptance utility already, which
is significantly less than what is implied by our Dropbox measurements.}

% Of course the best way to measure utility for $\checkerbl$ and $\checkerapprox$
% would be to perform experiments with these schemes in situ at Dropbox. We are in
% the process of obtaining the necessary approvals from Dropbox to do these
% additional experiments. We expect to have them approved and included in the next
% version of the paper.

\iffalse
% This was for reference for above discussion of decrease in utility for
% blacklist and approx optimal. 
\begin{tabular}[t]{lcrrr}
    \toprule
    \textbf{Checker} & \textbf{Set} & $\utilinc$ (MTurk) & $\utilinc$ (Dropbox) &  $\secloss_{q}$ (\%)\\
    \midrule
    \multirow{3}{*}{\checkerall} 
                     & $\toptwo$ &  0.59\%& 2.30\%& 4.5\%\\
                     & $\topthree$& 0.67\%& 3.00\%& 13.2\%\\
                     & $\topfive$ & 0.68\%& 3.19\%& 14.5\%\\
    \midrule
    \multirow{3}{*}{\checkerbl} 
                     & $\toptwo$ &  0.57\%&    & 2.9\%  \\
                     & $\topthree$ &  0.64\%&  & 8.2\%\\
                     & $\topfive$ &  0.65\%&   & 8.2\%\\
    \midrule
    \multirow{3}{*}{\checkerapprox} 
                     & $\toptwo$ &  0.52\%&  & 1.0\%\\
                     & $\topthree$ & 0.59\%& & 1.0\% \\
                     & $\topfive$ &  0.60\%& & 1.0\% \\

                     % \swcall   &1.835&3.0& 0.60\\
                     % \swcfirst &0.225&1.6& 0.14\\
                     % \rmlast   &0.084&11.6&$<0.01$\\
                     % \rmfirst  &0.013&1.7& $<0.01$\\
                     % \dtoslast &0.001&0.2& $<0.01$\\
    \bottomrule
  \end{tabular}
\fi

\paragraph{Implementation considerations.}
The checkers above are all easy to implement, but care must be taken to 
optimize performance and ensure timing attacks do not arise.
Generally, each checker should first run $\exchecker(\pwtypo)$ since this must
always be computed. If that fails, then a constant-time check of
the remainder of the ball should be performed. This involves running $\exchecker$ for the maximum number of checks that could occur for
any $\pwtypo$, i.e., $|\typoset|$. If implemented in this manner, timing and
other side-channels will only potentially leak that a user made a typo, but
nothing else about their password.  Users that correctly input their
passwords experience no performance degradation compared to 
existing systems.

If one instead does not use a constant time
implementation, for example just running a check for each string in
$\ball(\pwtypo)$, then timing
%Naive implementations of the typo checkers would be vulnerable
side channels will arise that leak partial information about a user's password.
For example, checking a singleton ball (which is induced by some inputs and
not other inputs for $\checkerbl$ and $\checkerapprox$) would be faster 
than checking a ball with multiple passwords. Thus the
side-channel would reveal whether the user entered a high-probability password.
%This problem can be fixed easily by implementing the
%checkers in a constant-time fashion, always running $\exchecker$ the same
%number of times.  


\paragraph{Security evaluation.} In the remainder of this section we evaluate the
security of our schemes against two types of attacker:

\begin{newenum}
%\item The checkers are parameterized
%  by a set of correctors, and the utility will also vary across the three
%  strategies. We use simulations using password leaks and our measurements from
%  Dropbox to evaluate the  utility improvement of these schemes relative to
%  exact checking for a variety of settings.
   % To do so, we have to analyze the utility enhancement and security loss for the correctors we described in~\sefref{sec:mturk}.
%\item Next we would like to compare the performance of our three
%  empirical checkers with the corrector set we found in from the
%  prior analysis.  
\item[(A)] {\bf Exact-knowledge attackers:} We start by evaluating security 
  of the constructions in the face of attackers that (unrealistically) 
  know the precise distribution from which passwords are drawn. We will use a
  range of simulated password distributions and adversarial query budgets.
  %evaluate the robustness  in the face of a wrong setting of~$q$ or
  %estimate of the password distribution.
  %of the real distribution of the passwords and the value of~$q$.  This is
  %particularly important for $\checkerbl$ and $\checkerapprox$ that themselves
  %make use of estimates of password distributions.
 % So, we would like to know
 % the how $\checkerapprox$ performs when the password distribution is
 % different from the one the checker assumes or how having significantly more
 % guesses than  
 % how does the security loss change when the attacker gets more then
 % $q$ guesses to make.
\item[(B)] 
  {\bf Estimating attackers:} We will then turn to more realistic attackers that
  do not have exact knowledge of the
  password distribution. Our evaluations will show that in this context an attacker 
  attempting to take advantage of tolerant checking, even when they know the
  precise checker, can be quite error-prone: attackers can even do worse than
  naive approaches that just guess the most probable passwords in order.
\end{newenum}
Our approach for these analyses will be to utilize different password
leaks to simulate true password selection. We will use the RockYou,
phpBB, and Myspace leaks for these purposes. These leaks contain respectively the
passwords of more than 32 million, 255,421, and 41,545 users of
three different websites. Below when we say the RockYou, phpBB, or
Myspace distribution we mean sampling according to the empirical
distribution given by the indicated leak.
Note that this means for some analyses we will use RockYou 
both within the designs of 
$\checkerbl$ and $\checkerapprox$ as well as to test those designs' security,
optimistically modeling that a ``best-case'' estimate of the distribution is known to
the checker. While we could use a holdout set (sampled from RockYou without
replacement, for example) to be more realistic, we instead simply perform 
analyses using the independent Myspace and phpBB data sets and report all of
them for completeness.



\subsection{Security against exact-knowledge attackers}


We now evaluate the security of our constructions against attackers that have
exact knowledge of the password distribution.  
Thus in this section we assume that the adversary knows not only the exact
functioning of the checker being used (i.e., what typos it corrects for any
submitted password),
but also the precise distribution of
passwords. The latter is a conservative assumption. Attackers in
practice will lack such knowledge and we are therefore measuring worst-case
security from this point of view.

We will focus on the greedy success rate increase
$\greedylambda_q - \lambda_q$ for various values of the query
budget~$q$.  % As noted above, the greedy algorithm is known to provide
% a good approximation for the weight max cover problem; given the small
% set sizes in our setting especially, we expect
% $\greedylambda_q \approx \fuzzlambda_q$. Our experiments should thus
% approximate closely the security loss against the best remote guessing
% attacks.
We will also report on $\lambda_q$ to put loss in context.
To compute these values, we use the RockYou, Myspace, and phpBB
distributions as a stand-ins to simulate a challenge distribution
$\pwprob$.  Since the optimal attacker is assumed to know the
distributions exactly, in the exact checking setting she will simply
guess the most probable $q$ passwords. Here $\lambda_q$ is
straightforwardly computable (just sum the probabilities of the
top~$q$ passwords in the challenge distribution).  In the
typo-tolerant settings, the attacker will construct a sequence of
queries that achieves $\greedylambda_q$ using the algorithm given in
\apref{sec:faster-attack}.


% \begin{figure}[t]
% \centering
% \footnotesize
% \begin{tabular}[t]{llrrrr}
%   \multicolumn{6}{c}{q=1000}\\[2pt]
%     \toprule
%      \textbf{Challenge}                     &              &
%      \multicolumn{3}{c}{$\greedylambda_{q} - \lambda_q$
%                           ($\times 10^{-2}$)} & \multicolumn{1}{c}{$\lambda_q$ ($\times 10^{-2}$)}\\
%     \textbf{Dist.} & \textbf{Set} &  \checkerall & \checkerbl & \checkerapprox & \exchecker \\
%     \midrule
%     \multirow{3}{*}{RockYou} & $\toptwo$    & 0.51 &  0.32 & 0.00 &\multirow{3}{*}{11.23}\\
%                              & $\topthree$  & 1.41 & 0.86 & 0.00\\
%                              & $\topfive$   & 1.57 & 0.87 & 0.00 \\\midrule
%     \multirow{3}{*}{phpBB} & $\toptwo$    &   0.38 & 0.19 & 0.15 & \multirow{3}{*}{12.71}\\
%                              & $\topthree$  & 1.01 & 0.60 & 0.42\\
%                              & $\topfive$   & 1.13 & 0.72 & 0.47 \\\midrule
%     \multirow{3}{*}{Myspace} & $\toptwo$    & 0.49 &  0.45 & 0.35 & \multirow{3}{*}{9.54}\\
%                              & $\topthree$  & 2.46 & 2.21 & 1.59 \\
%                              & $\topfive$   & 3.00 & 2.66 & 1.94 \\\midrule
%     \bottomrule
%   \end{tabular}
% \shepherd{
%   %%%% q=100
%   \begin{tabular}[t]{llrrrr}\\
%     \multicolumn{6}{c}{q=100}\\ \toprule \textbf{Challenge} & &
%     \multicolumn{3}{c}{$\greedylambda_{q} - \lambda_q$ ($\times
%       10^{-2}$)} & \multicolumn{1}{c}{$\lambda_q$ ($\times
%       10^{-2}$)}\\
%     \textbf{Dist.} & \textbf{Set} & \checkerall & \checkerbl & \checkerapprox & \exchecker \\
%     \midrule \multirow{3}{*}{RockYou}
%     & $\toptwo$ & 0.15 & 0.05 & 0.00 &\multirow{3}{*}{4.50}\\
%     & $\topthree$ & 0.56 & 0.14 & 0.00\\
%     & $\topfive$ & 0.63 & 0.18 & 0.00 \\\midrule
%     \multirow{3}{*}{phpBB}
%     & $\toptwo$ & 0.12 & 0.02 & 0.00 & \multirow{3}{*}{5.50}\\
%     & $\topthree$ & 0.28 & 0.04 & 0.01 \\
%     & $\topfive$ & 0.31 & 0.05 & 0.02 \\\midrule
%     \multirow{3}{*}{Myspace}
%     & $\toptwo$ & 0.15 & 0.12 & 0.03 & \multirow{3}{*}{2.86}\\
%     & $\topthree$ & 0.62 & 0.46 & 0.32 \\
%     & $\topfive$ & 0.87 & 0.68 & 0.52 \\\midrule \bottomrule
%   \end{tabular}
% }\shepherd{
%   %%%% q = 10
% \begin{tabular}[t]{llrrrr}\\
%   \multicolumn{6}{c}{q=10}\\[2pt]
%     \toprule
%      \textbf{Challenge}                     &              &
%      \multicolumn{3}{c}{$\greedylambda_{q} - \lambda_q$
%                           ($\times 10^{-2}$)} & \multicolumn{1}{c}{$\lambda_q$ ($\times 10^{-2}$)}\\
%     \textbf{Dist.} & \textbf{Set} &  \checkerall & \checkerbl & \checkerapprox & \exchecker \\
%     \midrule
%     \multirow{3}{*}{RockYou} & $\toptwo$    & 0.03 &  0.00 & 0.00 &\multirow{3}{*}{1.95}\\
%                              & $\topthree$  & 0.22 & 0.03 & 0.00\\
%                              & $\topfive$   & 0.25 & 0.06 & 0.00 \\\midrule
%   \multirow{3}{*}{phpBB}     & $\toptwo$    & 0.03 & 0.00 & 0.00 &  \multirow{3}{*}{2.75}\\
%                              & $\topthree$  & 0.19 & 0.02 & 0.00\\
%                              & $\topfive$   & 0.20 & 0.03 & 0.01 \\\midrule
%     \multirow{3}{*}{Myspace} & $\toptwo$    & 0.03 & 0.01 & 0.00 & \multirow{3}{*}{0.79}\\
%                              & $\topthree$  & 0.17 & 0.06 & 0.02 \\
%                              & $\topfive$   & 0.27 & 0.15 & 0.04 \\\midrule
%     \bottomrule
%   \end{tabular}
% }

%   \caption{Improvement in exact-knowledge adversary's success for each scheme and corrector set across different
%   challenge distributions. Here $q = 1,000$ (top), $q=100$ (middle), and $q=10$ (bottom).}
%   \label{fig:compare-sets-security}
% \end{figure}


\begin{figure*}[t]
\centering
\gamesfontsize
%%%% q=1000
\begin{tabular}[t]{ll*{4}{r}|*{4}r|*{4}{r}}
    \toprule
     \textbf{Challenge}  && \multicolumn{4}{c|}{$q=10$} & \multicolumn{4}{c|}{$q=100$} & \multicolumn{4}{c}{$q=1000$} \\
  
     % \multicolumn{3}{c}{$\greedylambda_{q} - \lambda_q$
     %                      ($\times 10^{-2}$)} & \multicolumn{1}{c}{$\lambda_q$ ($\times 10^{-2}$)}\\
    \textbf{Dist.} & \textbf{Set} &  \textsf{All} & \textsf{wBL} & \textsf{AOp} & \textsf{Ex} & \textsf{All} & \textsf{wBL} & \textsf{AOp} & \textsf{Ex} & \textsf{All} & \textsf{wBL}  & \textsf{AOp} & \textsf{Ex} \\
    \midrule
    \multirow{3}{*}{RockYou} 
  & $\toptwo$    & 0.03 &  0.00 & 0.00 & \multirow{3}{*}{1.95} & 0.15 & 0.05 & 0.00 &\multirow{3}{*}{4.50}& 0.51 &  0.32 & 0.00 & \multirow{3}{*}{11.23} \\
  & $\topthree$  & 0.22 & 0.03 & 0.00 && 0.56 & 0.14 & 0.00 && 1.41 & 0.86 & 0.00 \\ 
  & $\topfive$   & 0.25 & 0.06 & 0.00 && 0.63 & 0.18 & 0.00 && 1.57 & 0.87 & 0.00 \\\midrule
  
  \multirow{3}{*}{phpBB} 
  & $\toptwo$    & 0.03 & 0.00 & 0.00 &  \multirow{3}{*}{2.75} & 0.12 & 0.02 & 0.00 & \multirow{3}{*}{5.50} & 0.38 & 0.19 & 0.15 & \multirow{3}{*}{12.71} \\
  & $\topthree$  & 0.19 & 0.02 & 0.00 && 0.28 & 0.04 & 0.01 && 1.01 & 0.60 & 0.42\\
  & $\topfive$   & 0.20 & 0.03 & 0.01 && 0.31 & 0.05 & 0.02 && 1.13 & 0.72 & 0.47 \\\midrule

    \multirow{3}{*}{Myspace} 
  & $\toptwo$     & 0.03 & 0.01 & 0.00 & \multirow{3}{*}{0.79} & 0.15 & 0.12 & 0.03 & \multirow{3}{*}{2.86} & 0.49 &  0.45 & 0.35 & \multirow{3}{*}{9.54}\\
  & $\topthree$  & 0.17 & 0.06 & 0.02  && 0.62 & 0.46 & 0.32 && 2.46 & 2.21 & 1.59\\
  & $\topfive$   & 0.27 & 0.15 & 0.04  && 0.87 & 0.68 & 0.52 && 3.00 & 2.66 & 1.94\\\bottomrule
  \end{tabular}

  \caption{Percentage improvements in an exact-knowledge adversary's success
    ($\greedylambda_q-\lambda_q$) for each setting (corrector strategy and correction
    set) and each of the challenge distributions, for $q \in \{10, 100, 1000\}$.}
  \label{fig:compare-sets-security}
\end{figure*}



We start by comparing security for attackers given $q = 1,000$ queries across
the various distributions, schemes, and corrector sets. We are here being
conservative: a query budget
of 1,000 is very generous to an attacker, as many websites will lock an account
after tens of failed requests.  \figref{fig:compare-sets-security} reports the
optimal success probability $\lambda_q$ against an exact checker for each
setting, as well as the improvements $\greedylambda_q - \lambda_q$ for each typo
tolerant checker, correction set pair.  All numbers are reported as percentages.
The worst degradation occurs for correcting all top five errors in the Myspace setting, 
where the attacker's success probability increases by 3\% (from 9.5\% to 12.5\%).  To put this worst-case 
in perspective, consider the naive
(and incorrect) assumption that seems to underlie the criticism of typo
tolerance~\cite{zdnet2011Facebook}: it suggests
instead a fivefold increase in attacker success when correcting five
errors and thus an increase to 47.5\% in the Myspace setting.




Elsewhere the increase is much smaller. For example, with
Rockyou, one can always correct all top five errors with increase only 1.6\%: an
attacker's probability of success goes from 11.2\% to 12.8\%, a small
improvement. This means that the adversary's first 1,000 guesses
against a typo-tolerant checker do not benefit much from
high-probability balls.  

Moving from $\toptwo$ to $\topthree$ can result in a relatively big jump in security loss. The reason is that
the \rmlast typo corrector admits many higher-mass balls than only
correcting the considered capitalization errors. For example, adding a character to many popular
passwords results in another popular password: \texttt{password} and \texttt{password1}, \texttt{abc123} and
\texttt{abc1234}. Fewer such pairs exist for capitalization errors since
fewer users choose passwords with capital letters. Indeed in the worst case for
$\toptwo$ we see a just a 0.5\% improvement in adversarial success compared to
the 2.42\% worst-case jump for $\topthree$. It is no
coincidence, perhaps, that Facebook's policy seems to align with $\checkerall$
for $\toptwo$. Our measurements are the first reported validation of
this policy.

Even though security loss is low for $\checkerall$, one may  
want to do better. The blacklist and approximately
optimal checkers help. 
When the challenge distribution is RockYou the
approximately optimal checker $\checkerapprox$ is, in this case, actually
optimally secure by
construction, hence it suffers no security loss at all. Also note that
$\checkerbl$ may benefit unduly by knowing exactly the top 1,000 passwords from
RockYou. Thus the more important analyses are when tested on independent
distributions. Here we see some loss as one would expect given that the attacker in these cases
has, after all, more information about the challenge distribution than the
checker. But now the loss is small, and $\checkerapprox$ 
reduces the security loss compared to $\checkerall$ by
0.53\% on average over the Myspace and phpBB settings. $\checkerbl$
also reduces loss compared to $\checkerall$ by 0.27\% on average over Myspace
and phpBB, but never improves security more than $\checkerapprox$.

We now turn to what happens as $q$ varies.
In \figref{fig:compare-sets-security} shows the attack success increases for the
$q = 10$ and $q = 100$ cases. We note that the most realistic in practice is
$q=10$, since companies often will raise alarms after 10 consecutive 
failed login attempts.  Here we see that attackers benefit little from
typo-tolerance, and our $\checkerapprox$ reduces loss to $0.04\%$ or less. Often it
is zero.

It is concievable that in some settings 
an attacker might be able to make more than $q = 1,000$ queries, which implies
that our checker assumed too low of a bound on $q$.
We focus for simplicity on $\toptwo$,
a choice we expect many deployments to utilize, and show in 
%\devd{allowing 100 guesses is very very generous. Doing the graphs till 10000
%seems extreme to me.} We focus on the corrector set $\toptwo$ because of its si
%when the true password distribution varies.  
\figref{fig:secloss-q} the security loss using RockYou as the distribution
for a range of $q \in \{1,100,200,300,\ldots,\,10\textnormal{,}000\}$.  We have
drawn a vertical dotted line at $q = 1,000$, which was used by $\checkerapprox$
as the expected query budget. 
%The loss for $q = 100$ is very small; for
%$\checkerall$ attacks only improve by 0.15\% and for \checkerbl only by
%0.05\%. 
As before, $\checkerapprox$ has no loss below $q = 1,000$, and 
only after the attacker gets more than 1,000 queries does the attacker
obtain an improvement over the exact checking case. \figref{fig:secloss-q-phpBB} shows the same type of chart but now for phpBB.
This distribution leads to security loss seeing big discrete jumps 
for larger~$q$, suggesting that at certain points the attacker can take advantage of
new balls that just come in to play as higher mass than individual passwords.
A chart for Myspace would exhibit similar trends as the one for phpBB, we omit it for the sake
of brevity.

\begin{figure}[t]
  \centering 
  \begin{tikzpicture}[scale=0.55]
    \pgfplotstableread[col sep=comma]{images/guess_rockyou.csv}{\guesstable}
    \begin{axis}[
%      ymin=0.00, ymax=0.50,
%      axis y line=right, 
%      axis x line=none,
%      y axis style=red!75!black,
      xlabel={Number of guesses allowed per account},
      ylabel={$\greedylambda_q - \lambda_q$ ($\times 10^{-2}$)},
      legend style={at={(.99,0.01)}, anchor=south east},
      y tick label style={/pgf/number format/fixed,
        /pgf/number format/1000 sep = \thinspace % Optional if you want to replace comma as the 1000 separator 
      },
      each nth point=1, filter discard warning=false, unbounded coords=discard,
      % scaled y ticks={base 10:2}
      legend entries={\checkerall, \checkerbl, \checkerapprox},
      x tick label style={/pgf/number format/fixed},
      xtick = {1,2000,4000,...,10000},
      scaled x ticks = false,
        % cycle list={
        %   {mark=triangle*, draw=blue!70}, 
        %   {mark=square*,draw=red!70},
        %   {mark=*,draw=violet!80},
        % }
      ]
      \addplot [mark repeat=5, mark phase=1, color=blue, mark=*] table [x expr=\thisrow{q}, y expr=(\thisrow{chkall}-\thisrow{chkexact})*100] \guesstable;
      \addplot [mark repeat=5, mark phase=1, color=red, mark=square*] table [x expr=\thisrow{q}, y expr={(\thisrow{chkbl}-\thisrow{chkexact})*100}] \guesstable;
      \addplot [mark repeat=5, mark phase=1, color=brown, mark=triangle*] table [x expr=\thisrow{q}, y expr={(\thisrow{chkaop}-\thisrow{chkexact})*100}] \guesstable;
      % \addplot [mark repeat=5, mark phase=7, color=blue, mark=*] table [x expr=\thisrow{q}, y expr=\thisrow{chkall}/\thisrow{chkexact}] \guesstable;
      % \addplot [mark repeat=5, mark phase=7, color=red, mark=square*] table [x expr=\thisrow{q}, y expr=\thisrow{chkbl}/\thisrow{chkexact}] \guesstable;
      % \addplot [mark repeat=5, mark phase=7, color=brown, mark=triangle*] table [x expr=\thisrow{q}, y expr=\thisrow{chkaop}/\thisrow{chkexact}] \guesstable;
      \draw [dashed] ({axis cs:1000,0}|-{rel axis cs:0,1}) -- ({axis cs:1000,0}|-{rel axis cs:0,0});
    \end{axis}
  \end{tikzpicture}  

    \caption{The security loss as a function of $q$ for challenge distribution
    RockYou and $\toptwo$.}
  \label{fig:secloss-q}
\end{figure}

\begin{figure}[t]
  \centering 
  \begin{tikzpicture}[scale=0.55]
    \pgfplotstableread[col sep=comma]{images/guess_phpbb.csv}{\guesstable}
    \begin{axis}[
%      ymin=1, ymax=4,
      xlabel={Number of guesses allowed per account},
%      ylabel={$\log_2\greedylambda_q$},
      ylabel={$\greedylambda_q - \lambda_q$ ($\times 10^{-2}$)},
      legend style={at={(.99,0.01)}, anchor=south east},
      y tick label style={/pgf/number format/fixed,
         /pgf/number format/1000 sep = \thinspace % Optional if you want to replace comma as the 1000 separator 
       },       
       each nth point=1, filter discard warning=false, unbounded coords=discard,
       % scaled y ticks={base 10:2}
       % legend entries={$\exchecker$, $\checkerall$, $\checkerbl$, $\checkerapprox$},
       legend entries={$\checkerall$, $\checkerbl$, $\checkerapprox$},
       x tick label style={/pgf/number format/fixed},
       xtick = {1,2000,4000,...,10000},
       scaled x ticks = false,
        % cycle list={
        %   {mark=triangle*, draw=blue!70}, 
        %   {mark=square*,draw=red!70},
        %   {mark=*,draw=violet!80},
        % }
      %ymode=log
       %xmode=log
      ]
      %\addplot [mark repeat=5, mark phase=1, color=blue, mark=x] table [x expr=\thisrow{q}, y expr=\thisrow{chkexact}] \guesstable;
      \addplot [mark repeat=5, mark phase=1, color=blue, mark=*] table [x expr=\thisrow{q}, y expr=100*(\thisrow{chkall}-\thisrow{chkexact})] \guesstable;
      \addplot [mark repeat=5, mark phase=1, color=red, mark=square*] table [x expr=\thisrow{q}, y expr=100*(\thisrow{chkbl}-\thisrow{chkexact})] \guesstable;
      \addplot [mark repeat=5, mark phase=1, color=brown, mark=triangle*] table [x expr=\thisrow{q}, y expr=100*(\thisrow{chkaop}-\thisrow{chkexact})] \guesstable;
%       % \addplot [mark repeat=5, mark phase=1, color=blue, mark=*] table [x expr=\thisrow{q}, y expr=\thisrow{chkall}/\thisrow{chkexact}] \guesstable;
%       % \addplot [mark repeat=5, mark phase=1, color=red, mark=square*] table [x expr=\thisrow{q}, y expr=\thisrow{chkbl}/\thisrow{chkexact}] \guesstable;
%       % \addplot [mark repeat=5, mark phase=1, color=brown, mark=triangle*] table [x expr=\thisrow{q}, y expr=\thisrow{chkaop}/\thisrow{chkexact}] \guesstable;
      \draw [dashed] ({axis cs:1000,0}|-{rel axis cs:0,1}) -- ({axis cs:1000,0}|-{rel axis cs:0,0});
    \end{axis}
  \end{tikzpicture}  

    \caption{Difference in exact-knowledge adversary success against typo-tolerant
    schemes and exact checking, as a function of $q$ for challenge distribution
    phpBB and $\toptwo$.}
  \label{fig:secloss-q-phpBB}
\end{figure}


Observe that as $q$ gets larger, the improvement
$\greedylambda_q - \lambda_q$ flattens out in both charts. The
attacker in the typo-tolerant cases runs out of heavily-weighted balls
to take advantage of and ends up just querying passwords that cover
only one (high-probability) guess. For RockYou, we see that the
improvement is never more than 1\% and, for phpBB, never
more than 0.8\%.  This all suggests that even as $q$ grows to values
unlikely ever to arise in practice, typo-tolerance nevertheless does
not improve the attacker's rate of success by much. We note that our
blacklisting and approximately optimal checkers can be made even more
conservative should one desire, by blacklisting more passwords or
setting $q$ larger, respectively.

As noted in the previous section, 
the greedy algorithm is known to provide a
good approximation for the weighted max coverage problem. Given that
the password probabilities range over a very dense space, and our
correction sets are quite small, we expect
$\greedylambda_q \approx \fuzzlambda$. We do not have any theoretical
proof for this claim, and leave analysis as an important question for future
work.  We can of course always bound the actual value of $\secloss_q$ via
$\seclosso_q\le 1.582\,\seclossg_q + 0.582\lambda_q$. So, for example with 
the RockYou challenge distribution, $q=10$, and the $\topfive$ corrector
set we have that $\seclosso_q\le 0.0153$ as compared to $\greedylambda_q -
\lambda_q = 0.0063$. We expect this three-fold decrease in relative security to be
quite pessimistic: the better the greedy algorithm approximates the problem the
worse the adjustment to compute $\seclosso_q$ becomes.

%take advantage of ball coverage
%gets proportionally less helpful the deeper in the distribution one goes. 

% The English is just WOW!
% Our experiments have two main objectives. First, we wish to determine
% the effect of our heuristics on acceptance utility and security. As
% they do not rely on exact distribution knowledge, they will incur some
% security degradation, which we set out to measure. At the same time, a
% relatively promiscuous checking approach such as $\checkerall$ also
% trades off security against utility, and can achieve higher utility
% than more secure checkers.






\subsection{Estimating attackers}

We have so far considered attackers that have exact
knowledge of the password distribution (even when the system designer may
not). In practice such attackers do not exist, and instead adversaries must try to estimate
the distribution of passwords. We refer to these as estimating attackers. 
As before, we assume adversaries know the exact checking algorithm in use.

We started by considering an adversary that estimates the password distribution
using the Weir et al.~probabilistic context-free grammar
(PCFG)~\cite{weiretal:2009}, a trained model of password
distributions used to build effective crackers. However, our
experiments with this showed that it provides poor efficacy in online guessing
attacks, doing significantly worse than the approaches we describe below and,
importantly, it
did equally poorly against the typo-tolerant checkers in all settings.

\iffalse
We train the model
using the Myspace leak with dictionary dict-0294. This was the best training
method reported in~\cite{weiretal:2009}. The attacker then performs exactly in the
way the greedy attacker $\advA^*$ works, only this time the precise knowledge of the 
distribution of passwords is replaced with the trained PCFG model.  
For exact checkers, this means the attacker uses the top $q$ passwords according
to the PCFG model, and for tolerant checkers, the attacker uses 
the algorithm specified in~\figref{fig:attack-algo} but using the PCFG model
instead of $\pwprob$.

%In more detail, as the checker algorithm is known to the adversary, for any
%given string the attacker can find the ball of passwords induced by the checker
%it is attacking, and also compute the estimated probability mass of that ball
%using his estimate of the password distribution.  For making a guess the
%attacker finds the password string from the set of all strings that has the
%highest cumulative probability mass of the induced ball excluding the passwords
%that have already been covered by his previous guesses.  The attacker generates
%$q$ guesses in this way.  Again, the algorithm specified
%in~\figref{fig:attack-algo} can be used to efficiently compute the best guesses
%for this sub-optimal attacker by setting the value of $\pwprob$ to the
%attacker's estimated distribution of passwords.  

We tested this attacker model using as challenge distribution each of the three
password leaks and using $\toptwo$. When the challenge distribution is Myspace we are training the
attacker on the same data as it is being tested on, but this can only be
advantageous to the attacker.
%The absolute success rates are shown in \figref{fig:secloss-suboptimal}. 
This strategy performs quite poorly: the attack against an exact checker
succeeds with probability 1.7\% for RockYou, 2.1\% for phpBB, and 0.2\% for
Myspace using $q = 1,000$ queries.   Recall that above the 
guessing attacker against RockYou had $\lambda_q = 11.2\%$ success. The reason that the Weir et
al.~password model does so poorly here is that it was tuned for offline attacks
where it is, generally speaking, more important to generate a large number of
probable passwords rather than exactly get the 1,000 most probable passwords. 
When attempting to take
advantage of typo tolerance, the attacker does \emph{no better}, achieving
the same success rates as in the exact case for any of
$\checkerall$, $\checkerbl$, and $\checkerapprox$. 
\fi


\begin{figure}[t]
  \centering\footnotesize
  \begin{tabular}[t]{c|l|rrr|}
    \cline{2-5}
     & \multicolumn{1}{c|}{\textbf{Attacker}}& \multicolumn{3}{c|}{\textbf{Challenge distribution}\Tstrut}\\\cline{3-5}
%    \parbox[t]{2mm}{\multirow{3}{*}{\rotatebox[origin=c]{90}{Attacker's Pw\\ distribution}}}
    &
    \multicolumn{1}{c|}{\textbf{distribution}}&RockYou&phpBB&Myspace\\\cline{2-5}
    \multirow{3}{*}{\exchecker} &  RockYou&11.23&3.21&{9.34\Tstrut}\\
    & phpBB  &8.10 &12.71&1.81\\
    & Myspace&3.57 &3.32&9.54\\\cline{2-5}\cline{2-5}
    \multirow{3}{*}{\checkerall} % & \multicolumn{1}{c|}{\textbf{Attacker}}&
%     \multicolumn{3}{c|}{\textbf{Challenge distribution}\Tstrut}\\\cline{3-5}
% %    \parbox[t]{2mm}{\multirow{3}{*}{\rotatebox[origin=c]{90}{Attacker's Pw\\ distribution}}}
%     & \multicolumn{1}{c|}{\textbf{distribution}}&RockYou&phpBB&Myspace\\\cline{2-5}
    & RockYou& +0.51 &+0.28 &{\Tstrut +0.25} \\
    & phpBB  & +0.25 & +0.38 &+0.11 \\
    & Myspace& \textbf{-0.15}  &\textbf{-0.02} & +0.49\\\cline{2-5}
    \multirow{3}{*}{\checkerbl} % & \multicolumn{1}{c|}{\textbf{Attacker}}&
%     \multicolumn{3}{c|}{\textbf{Challenge distribution}\Tstrut}\\\cline{3-5}
% %    \parbox[t]{2mm}{\multirow{3}{*}{\rotatebox[origin=c]{90}{Attacker's Pw\\ distribution}}}
%     & \multicolumn{1}{c|}{\textbf{distribution}}&RockYou&phpBB&Myspace\\\cline{2-5}
    & RockYou& +0.32 &+0.11 &{+0.20\Tstrut} \\
    & phpBB  & +0.06  & +0.19 & +0.05\\
    & Myspace& \textbf{-0.26} & \textbf{-0.20}& +0.46\\\cline{2-5}
    \multirow{3}{*}{\checkerapprox} % & \multicolumn{1}{c|}{\textbf{Attacker}}&
%     \multicolumn{3}{c|}{\textbf{Challenge distribution}\Tstrut}\\\cline{3-5}
% %    \parbox[t]{2mm}{\multirow{3}{*}{\rotatebox[origin=c]{90}{Attacker's Pw\\ distribution}}}
%     & \multicolumn{1}{c|}{\textbf{distribution}}&RockYou&phpBB&Myspace\\\cline{2-5}
    & RockYou& 0.00 &0.00 &{0.00\Tstrut}\\
    & phpBB  & \textbf{-0.11}  & +0.15 & \textbf{-0.04}\\
    & Myspace&\textbf{-0.27}  & \textbf{-0.14} & +0.35\\\cline{2-5}
  \end{tabular}\\
  \vspace{0.1in}
  \caption{The top table shows the success rate of an attack against the exact checking
  scheme for the attacker-estimated distribution (row) used 
  against the challenge distribution (column). The remaining tables
  show the \emph{difference} between success rate of an attacker against the tolerant scheme and the exact
  checking scheme, for the indicated attacker-estimated and actual challenge distribution pairs.
  All values are in percentages.}
\label{fig:leakvsleak}
\end{figure}


We therefore turn to a different adversarial strategy for estimating the
password distribution. We measure the success rate of an attacker that uses one
of the password leaks as its estimate of the distribution. This is a typical
strategy in practice. 
We test these attacks
against the other two distributions and for each of the exact checking,
$\checkerall$, $\checkerbl$, and $\checkerapprox$. The latter three use $\toptwo$. 
The security loss for all combinations are tabulated
in~\figref{fig:leakvsleak}. (Note that the left-to-right diagonals
reflect some of the results already shown for the exact-knowledge attacker in
\figref{fig:compare-sets-security}.) 

%The first observation is that using different leaks does a better job at online
%guessing attacks against an exact checker than the Weir et al.~model.  Second,
The improvement the attacker obtains when one switches to a tolerant
checking system is never greater than 0.28\%.  More interestingly, in some cases
the difference is negative, which means that the attacker did \emph{worse}
against the typo-tolerant scheme. This may be counterintuitive, but here the
estimates the attacker makes about the distribution can often be wrong. This can
lead her to choose a set of guesses that maximizes the total success probability
according to her estimate but not according to the challenge distribution.
We give an example for the curious reader in \apref{sec:example}.


In summary, our simulations here suggest that a carefully designed typo-tolerant checker will result in little to no security loss against realistic adversaries.


%%%%%%%%%%%%%%%%%%%%%%%%%%%%%%%%%%%%%%%%%%%%%%%%%%%%%%%%%%%%%%%%%%%%%%%%%%%%%%%%%
%%% JUNKYARD 
%%%%%%%%%%%%%%%%%%%%%%%%%%%%%%%%%%%%%%%%%%%%%%%%%%%%%%%%%%%%%%%%%%%%%%%%%%%%%%%%\iffalse

\iffalse
  \begin{tabular}[t]{llrrrr}
    \toprule
     \textbf{Challenge}                     &              & \multicolumn{3}{c}{$\secloss_{q}$
                          (\%)} & \multicolumn{1}{c}{$\lambda_q$ (\%)}\\
    \textbf{Dist.} & \textbf{Set} &  \checkerall & \checkerbl & \checkerapprox & \exchecker \\
    \midrule
    \multirow{3}{*}{RockYou} & $\toptwo$    & 4.5 &  2.9 & 0.0 &\multirow{3}{*}{11.2\%}\\
                             & $\topthree$  & 13.2 & 8.2 & 0.0\\
                             & $\topfive$   & 14.5 & 8.2 & 0.0 \\\midrule
    \multirow{3}{*}{Myspace} & $\toptwo$    & 5.2 &  4.8& 3.7 & \multirow{3}{*}{9.5\%}\\
                             & $\topthree$  & 25.9 & 23.4 & 18.0 \\
                             & $\topfive$   & 31.5 & 28.5 & 22.0 \\\midrule
    \multirow{3}{*}{phpBB} & $\toptwo$    &   3.0 & 1.5 & 1.2 & \multirow{3}{*}{12.7\%}\\
                             & $\topthree$  & 7.9 & 4.6 & 3.2\\
                             & $\topfive$   & 8.9 & 5.5 & 3.7 \\\midrule
    \bottomrule
  \end{tabular}

\begin{tabular}[t]{llrrrr}
    \toprule
     \textbf{Challenge}                     &              &
     \multicolumn{3}{c}{$\greedylambda_{q}$
                          (\%)} & \multicolumn{1}{c}{$\lambda_q$ (\%)}\\
    \textbf{Dist.} & \textbf{Set} &  \checkerall & \checkerbl & \checkerapprox & \exchecker \\
    \midrule
    \multirow{3}{*}{RockYou} & $\toptwo$    & 11.7 &  11.5 & 11.2 &\multirow{3}{*}{11.2}\\
                             & $\topthree$  & 12.6 & 12.1 & 11.2\\
                             & $\topfive$   & 12.8 & 12.1 & 11.2 \\\midrule
    \multirow{3}{*}{Myspace} & $\toptwo$    & 10.0 &  10.0 & 9.9 & \multirow{3}{*}{9.5}\\
                             & $\topthree$  & 11.9 & 11.7 & 11.2 \\
                             & $\topfive$   & 12.5 & 12.2 & 11.6 \\\midrule
    \multirow{3}{*}{phpBB} & $\toptwo$    &   13.1 & 12.9 & 12.9 & \multirow{3}{*}{12.7}\\
                             & $\topthree$  & 13.7 & 13.3 & 13.1\\
                             & $\topfive$   & 13.8 & 13.4 & 13.2 \\\midrule
    \bottomrule
  \end{tabular}
The column labeled as $\exchecker$ denotes the success rate for each
challenge distribution when there is no typo tolerance, and that is
the baseline success rate of our attacker. When we enable typo
tolerance the success rate has almost no effect for RockYou and Phpbb
distributions, and very minimal increase ($0.01\%$) for Myspace
distribution.  Also, we can see that the this model performs
substantially worse than the optimal attacker. For example, the
success probability against $\exchecker$ is less than $0.02$ for
RockYou and phpBB and less than $0.002$ for Myspace, which are orders
of magnitude less than what an optimal attacker would have achieved
(c.f.~\figref{fig:secloss-checkers}).  

\iffalse
\begin{figure}[t]
  \centering
\pgfplotstableread[format=inline, col sep=comma]{
  index,pwdist,excheck,chkall,chkbl,chkaop
  0,RockYou,1.70,1.71,1.71,1.71
  1,phpBB,2.05,2.05,2.05,2.05      
  2,Myspace,0.22,0.23,0.23,0.23
}\suboptsecloss

% \begin{tikzpicture}[scale=0.65]
%   \begin{axis}[
%     axis x line = bottom,
%     axis y line = left,
%     ymin=0.1, ymax=2.5,
%     xlabel={\large Checkers},
%     ylabel={\large Success of optimal attacker},
%     xtick=data,
%     ticks=both,
%     xticklabels from table={\suboptsecloss}{pwdist},
%     legend cell align=left,
%     legend style={at={(0.97,0.97)}, anchor=north east, column sep=1ex},
%     enlarge x limits=0.25,
%     ybar,
%     y tick label style={/pgf/number format/fixed,
%       /pgf/number format/1000 sep = \thinspace % Optional if you want to replace comma as the 1000 separator 
%     },
%     cycle list={
%       {fill=black!20,draw=black!20},
%       {fill=black!40,draw=black!40}, 
%       {fill=black!60,draw=black!60},
%       {fill=black!80,draw=black!80},
%     }
%     ]
%     \addplot table [x={index}, y={excheck}, ] {\suboptsecloss};
%     \addplot table [x={index}, y={chkall}, ] {\suboptsecloss};
%     \addplot table [x={index}, y={chkbl}, ] {\suboptsecloss};
%     \addplot table [x={index}, y={chkaop},] {\suboptsecloss};
%     \legend{\exchecker, \checkerall, \checkerbl, \checkerapprox},
%   \end{axis}
% \end{tikzpicture}

  \small
   \begin{tabular}[t]{p{0.5in}*{4}{c}}
      \toprule
      Challenge & $\lambda_q$ & \multicolumn{3}{c}{$\greedylambda_q - \lambda_q$}\\
      Distribution &\exchecker& \multicolumn{1}{c}{\checkerall}& \multicolumn{1}{c}{\checkerbl}&\multicolumn{1}{c}{\checkerapprox}\\\midrule
      RockYou &0.017 & 0.0  & 0.0 & 0.0\\\midrule
      phpBB   &0.021 & 0.0  & 0.0 & 0.0\\\midrule      
      Myspace &0.002 & 0.0  & 0.0 & 0.0\\
      \bottomrule
    \end{tabular}\\

    % \tnote{Possibly put the latter three columns as percentages over first
    % column, i.e. $\greedylambda_q$ as a percentage. So it will be zero
    % everywhere. Maybe not thinking...} 
  \caption{The success probability of an attacker that uses the 
    Weir et al.~\cite{weiretal:2009} PCFG against checkers using~$\toptwo$ and
    $q = 1,000$ queries.}
    % \caption{corrector set = $\topthree$}
%  \end{subfigure}
    \label{fig:secloss-suboptimal}
\end{figure}
\fi

\begin{figure}[t]
  \centering
\pgfplotstableread[format=inline, col sep=comma]{
  index,pwdist,excheck,chkall,chkbl,chkaop
  0,RockYou,0.113,0.118,0.116 ,0.113
  1,phpBB,0.127,0.129,0.129 ,0.129 
  2,Myspace,0.095,0.100,0.100,0.0989
}\seclosstable

\begin{tikzpicture}[scale=0.65]
  \begin{axis}[
    axis x line = bottom,
    axis y line = left,
    ymin=0.07, ymax=0.15,
    xlabel={\large Checkers},
    ylabel={\large Success of optimal attacker},
    xtick=data,
    ticks=both,
    xticklabels from table={\seclosstable}{pwdist},
    legend cell align=left,
    legend style={at={(0.97,0.97)}, anchor=north east, column sep=1ex},
    enlarge x limits=0.25,
    ybar,
    y tick label style={/pgf/number format/fixed,
      /pgf/number format/1000 sep = \thinspace % Optional if you want to replace comma as the 1000 separator 
    },
    cycle list={
      {fill=red!30,draw=black!80},
      {fill=blue!30,draw=black!840}, 
      {fill=green!30,draw=black!80},
      {fill=black!30,draw=black!80},
    }
    ]
    \addplot table [x={index}, y={excheck}, ] {\seclosstable};
    \addplot table [x={index}, y={chkall}, ] {\seclosstable};
    \addplot table [x={index}, y={chkbl}, ] {\seclosstable};
    \addplot table [x={index}, y={chkaop},] {\seclosstable};
    \legend{\exchecker, \checkerall, \checkerbl, \checkerapprox},
  \end{axis}
\end{tikzpicture}
{\small
  \begin{tabular}[t]{p{.5in}rrrr}
    \toprule
    PW Dist. & $\exchecker$ &  \checkerall& \checkerbl & \checkerapprox \\\midrule
    RockYou  & 0.113& 0.118& 0.116 & 0.113\\\midrule
    phpBB    & 0.127& 0.129& 0.129 & 0.129 \\\midrule
    Myspace  & 0.095& 0.100& 0.100 & 0.099 \\
    % Weir PCFG &&0.40&0.39&0.34 \\\bottomrule
    % Utility Increase\Tstrut && 2.1\% & 2.0\%& 1.9\%\\
    \bottomrule
  \end{tabular}
}
% \tnote{Maybe we should put these as security losses to be consistent. Also
%   it will look impressively low I think across the board.} 
% \rcnote{The issue with putting secloss in percentage is then exchecker will be 0.0\%}
\caption{Absolute success rates ($\lambda_q$ for $\exchecker$ and
  $\greedylambda_q$ for others) of an optimal attacker against different
  checkers using $\toptwo$ for the RockYou, phpBB, and Myspace
  password distributions. The attacker is optimal in sense that it is
  aware of the real challenge distribution of passwords and tries to
  maximize its success rate.  }
    \label{fig:secloss-distributions}
\end{figure}

%To evaluate our checkers we
%use three publicly available password leaks: RockYou, Phpbb and
%Myspace. % We compute the empirical distribution of passwords
%% for each of the leaks. We refer to each such distribution by the name
%% of the leak; e.g., RockYou means the empirical distribution of the
%% passwords in the RockYou leak.
%For each checker and for each leak we compute the success rate of an
%optimal attacker where the challenge password distribution is assumed
%to be the same as the empirical distribution of that leak and the
%attacker has complete knowledge of that distribution. In the left
%table of \figref{tab:secloss-checkers} we report the success rates
%when the attacker is allowed to make 1,000 guesses per account. Note
%that the success rate of the optimal attacker against $\exchecker$ is
%the same as the $\lambda_{10^3}$ of that distribution. As expected the
%optimal attacker obtains maximum success against $\checkerall$. 

%Also the optimal attacker performs no better against $\checkerapprox$
%than against $\exchecker$ when the distribution of the checker (which
%is RockYou) and the challenge distribution matches. For Phpbb and
%Myspace, even though there is an increase in success rate for the
%attacker against the $\checkerapprox$, it performs the best among all
%three heuristic checkers. This answers our third question about how
%the $\checkerapprox$ will perform when the estimate of the real
%distribution is incorrect. We can learn a proxy distribution for a
%challenge password distribution that leaks minimal information about
%actual passwords but when used with $\checkerapprox$ works well
%against any adversary.


% This alsoanswers part of are third question about the effect on $\secloss_q$ of
% $\checkerapprox$ for wrong estimation of password
% distribution. $\checkerapprox$ achieves best security loss for
% RockYou, $\greedylambda_q=\lambda_q$. But, for other leaks it loses non
% zero bits of security.  \todo{Fix this part with discussion of the
%   result in.}  we see the $\checkerapprox$ incurs zero security loss
% for RockYou. For other leaks also, we see $\checkerapprox$ performs
% the best.







\iffalse
\rcnote{
  What are the big point we want to address?
  \begin{enumerate}
  \item {\bf Big Picture.} A significant fraction of users make
    typos. Most of these typos are very easy to fix. Fixing those
    degrades security by minuscule amount.
  \item How to back these statements? 
    \begin{enumerate}
    \item We conducted typing experiment in MTurk to get statistics
      about the fraction of people make typos, and what are the broad
      types of those typos.
    \item based on the data we collected, we found that a small subset
      of transforms can fix large amount of typos.  We create
      $\topfive$ using that knowledge.
    \item We want to know what is an optimal guessing attack against a
      typo tolerant system, and how much is the increase in attacker's
      success for that algorithm when the attacker has perfect
      knowledge about the password distribution and when he doesn't .
    \item We observed that there is a degradation of 0.16 bits of
      entropy. Analyzing the data we figured out that most of this
      degradation is caused because typo of many popular passwords
      fall in the ball of other unpopular passwords. We designed two
      schemes to deal with this, first choose a list of blacklisted
      passwords (we used Twitter's banned password list) and then
      don't check any blacklisted password in the ball unless it is
      the center of the ball (\pfiveb). In the second case, the server
      checks at most one password from the blacklisted set in the ball
      including the ball center (\pfivebone). We found $\pfiveb$
      performs extremely well with correcting nearly 35\% of typos,
      and only increasing the success rate from 12.1\% to 12.7\% for
      q=10^3. (The success rate changes from \fixme{10\%} to
      \fixme{10.1\%} if we allow q=100.)
    \item 
    \end{enumerate}
  \end{enumerate}
}
\fi
\paragraph{Optimal Attack}
\label{sec:optimal-attack}
First, we consider an optimal attack, where we assume the attacker has
complete knowledge about the distribution of real registered passwords
in the server that allow typo in the password.  Also, the set of
mutations that the server apply on the entered password is a public
information. Given these two information, the attacker wants to find a
set of $q$ strings in $\Gamma^*$ that maximizes his chance of wining
the game in \figref{fig:acc-security}.  

We bound the success of an optimal attacker by
$\fuzzlambda_q(\pwprob, \tcf)$, which is defined as follows,
\[
  \fuzzlambda_q(\pwprob, \tcf) = \max_{S\in 2^{\Gamma^*}}\sum_{\pwtypo\in S}\sum_{\pw\in\ball(\pwtypo)} \pwprob(\pw) 
\]

Computing $\fuzzlambda_q$ requires searching over the power set of
$\Gamma^*$ and, hence, it is extremely resource intensive. The
attacker can optimize his search by making following two improvement
to his algorithm. First, he only need to consider strings that are in
neighborhood\footnote{Remember, for a password $\pw$, neighborhood
  $\newnh{\pw}$ is the set of passwords that the server will accept
  for $\pw$, while $\ball(\pwtypo)$ is the set of registered passwords
  that have $\pwtypo$ in their neighborhood.} of passwords with
non-zero probability mass.  Second, we know $q_1$ best guesses of the
attacker will be subset of $q_2$ best guesses whenever $q_1\le q_2$.
This amounts to say, that the attacker can do a greedy search over the
search space, and build the best $q$ guesses incrementally from
$q=1,2,...$.  This still could be large, as for example, say the
attacker considers ten million most probable passwords according to
$\pwprob$, and if the maximum size of a neighborhood around a password
is $n$, the total search space could be as large as $n\times 10^7$. We
describe some more optimization to the attacker's algorithm of finding
$q$ guesses in~\apref{sec:faster-attack}.


We define security loss or additional advantage of the optimal
attacker as
$ \secloss_q = \log_2\left(\fuzzlambda_q\over\lambda_q\right)$.  We
present the value of $\epsilon_q$ for each of the mutations we have
proposed in~\figref{tab:topten-typo}.  The proposed mutations are
exclusive in sense that no two mutations, for a given $\pwtypo$ will
produce same mutated string $\pw$ unless the mutated string is same as
$\pwtypo$. Also, we assumed identity mutation $f_0$ is always part of
the mutation list while computing $\secloss_q$
in~\tabref{tab:topten-typo} (\textbf{right most column}).


There are three dimension to the trade off the server can perform: the
loss of security ($\secloss_q$), increase in usability (typo
correction rate), and increase in the hash computation cost (max size
of a $\ball$). \todo{Not clear yet! we need specify this three in
  better ways.} We fix the number of hash computation or maximum size
of a $\ball$ to 3 and 5. For each of them we, computed the maximum
amount of typo that can be corrected for a given security loss. We
present this data in \figref{fig:sec-to-typocorr}.
\rcnote{lost!! Its probably better to define all three--security loss,
  hash computation, and typo correction}.

\newcommand{\addtext}[5]{      
      \node[style={fill=#5,circle,inner sep=0pt,minimum
        size=4pt},pin={[text=red,text width=#4,pin
        edge={black,thick}]#3:{\small{\tt #1}}}] at (axis cs:#2) {
      }; 
}

\begin{figure*}[t]
  \centering\small
  \begin{tikzpicture}[scale=0.7]
    \begin{axis}[
      axis x line*=bottom,
      axis y line*=left,
      xlabel={Security loss (bits)},
      ylabel={Fraction of typo fixed}
      ]
      \addplot [blue] table [col sep=comma,
      x=security, y=typofix] {images/sec-to-corr3.csv};

      % \addplot [violet, thick] table [col sep=comma, row sep=crcr,
      % x=security, y=typofix] {images/sec-to-corr5.csv};
      \pgfplotsset{
        after end axis/.code={
          \addtext{\{swc-all,\\swc-first,\\rm-lasts\}}{0.0750168937967,0.358060430613}{-45}{2cm}{blue}
          \addtext{\{swc-all,\\sws-last1\}}{0.0652155185933,0.345033472046}{175}{1.8cm}{blue}
          \addtext{\{swc-all\}}{0.0425743952525,0.307219106206}{180+45}{1.4cm}{blue}
        }
      }
    \end{axis}
  \end{tikzpicture}
  \begin{tikzpicture}[scale=0.7]
    \begin{axis}[
      axis x line*=bottom,
      axis y line*=left,
      xlabel={Security loss (bits)},
      ylabel={Fraction of typo fixed}
      ]
      \addplot [violet] table [col sep=comma,
      x=security, y=typofix] {images/sec-to-corr5.csv};

      % \addplot [violet, thick] table [col sep=comma, row sep=crcr,
      % x=security, y=typofix] {images/sec-to-corr5.csv};
      \pgfplotsset{
        after end axis/.code={
          \addtext{\{swc-all,swc-first,\\rm-lastl,rm-lasts,\\rm-firstc\}}{0.152436812067,0.378686448344}{-20}{1.5cm}{violet}
\addtext{\{swc-all,swc-first,\\rm-lasts,sws-last1,\\rm-firstc\}}{0.102597967735,0.372257479601}{90}{1.0cm}{violet}
% rm-firstc|swc-all|sws-last1|swc-first|rm-lasts,0.102597967735,0.372257479601
\addtext{\{swc-all,\\swc-first,\\up2cap,\\rm-lasts\}}{0.0757133037698,0.36029011786}{-80}{1.8cm}{violet}
          \addtext{\{swc-all,\\swc-first\}}{0.0652155185933,0.345784224841}{135}{1.6cm}{violet}
          \addtext{\{swc-all\}}{0.0425743952525,0.307219106206}{180+45}{1.2cm}{violet}
        }
      }
    \end{axis}
  \end{tikzpicture}

  \caption{This figure shows the maximum amount of typo that can be
    fixed for a given performance budget (number of hash computation)
    and the amount of tolerable security loss (increase in beta
    success rate).  Security loss is calculated as
    $\log_2\left(\fuzzlambda_{q} \over \lambda_{q}\right)$ for $q=1000$, and
    performance budget is set to 3 ({\bf left chart})
    and 5 ({\bf right chart}).}
  \label{fig:sec-to-typocorr}
\end{figure*}

\begin{figure}[t]
  \centering\gamesfontsize
  \begin{tabular}[t]{lrrr}
    \toprule
    \textbf{Corrector} & $\utilinc$ (MTurk) & $\utilinc$ (Dropbox) \\\midrule
    \swcall   &0.41\%& 0.40\%& 3.0\%\\
    \swcfirst &0.19\%& 1.90\%& 1.6\%\\
    \rmlast   &0.08\%& 0.70\%& 11.6\%\\
    \rmfirst  &0.01\%& 0.12\%& 1.7\%\\
    \dtoslast &0.00\%& 0.07\%& 0.2\%\\
    \bottomrule
  \end{tabular}
  \caption{For each corrector (row labels), we show the utility
    increase (\%) and security loss (\%) when it is used by itself with
    $\checkerall$.  Passwords were drawn according to RockYou for all
    measurements, while for utility typos are applied as per the MTurk
    or Dropbox empirical distribution.}
    %utility improvement and loss ratios are given in percentage and in the final column
    %we report the ratio of utility improvement to security loss.}
    %The rightmost column represents the weight of a corrector computed
    %as ${\utilinc-1\over \secloss_{10^3}-1}$. }
  \label{fig:correctors-alpha}
\end{figure}

\fi 


%%% Local Variables:
%%% mode: latex
%%% TeX-master: "main"
%%% End:
