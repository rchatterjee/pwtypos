%%%%%%%%%%%%%%%%%%%%%%%%%%%%%%%%%%%%%%%%%%%%%%%%%%%%%%%%%%%%%%%%%%%%%%%%%%%%%%%%
\subsection{Further Related Work}
\label{sec:relwork}

The study of password checking systems
originates with Morris and Thompson's seminal paper on the security of the UNIX
login system~\cite{morris}. Specifically they showed that a majority of
user-selected passwords were easily cracked via offline brute-force attacks:
given access to the hashes of passwords, one can repeatedly guess passwords and
check for consistency with the hash. A long line of subsequent studies have
confirmed that users tend to choose weak passwords~\cite{bonneau12,of,papers}. 
%As one notable example relevant to our results later, Bonneau instrumented the
%Yahoo!~login infrastructure, collected 69 million keyed hashes, and investigated
%a number of guessing metrics on the empirical distribution of passwords.  He
%proposes a metric called $\alpha$-guesswork for measuring the difficulty of
%mounting offline brute-force attacks relative to passwords chosen according 
%to some distribution.

There has been a wide array of research into memorability
and its relationship to security. Password strength
meters help signal to users when their password is
weak~\cite{others,komanduri2014telepathwords}, whereas password strength
policies force users to pick passwords according to some recipe that hopefully
makes them harder to guess.  Ur et al.~\cite{ur2012does} studied the effect of
password strength meters on user-selected passwords, and their results suggest
that only stringent meters can help increase security. somethin  Shay et
al.~\cite{shay2014can} showed that strength policies vary considerably in terms
of security and usability. Password update policies require users to change
passwords, and not reuse old passwords, but studies suggest that this
decreases usability without noticeably increasing
security~\cite{zhang:2010:security}.  

Some have suggested using passphrases --- sequences of words from some
dictionary --- instead of passwords (c.f.,~\cite{porter1982password}).
Human-chosen English passphrases exhibit similarly poor security as
human-chosen passwords~\cite{bonneau2012linguistic}.  The infamous
CorrectHorseStapleBattery-type passwords (first popularized by the xkcd comic
strip~\cite{xkcd}) suggest a user pick words from a dictionary and take the
password to be the concatenation of them. 

System-generated passwords are an alternative to user-chosen passwords. Here a
randomized computer program selects a password, and the user must memorize it.
In some systems the user is offered multiple passwords to choose from.  Bonneau
and Schechter~\cite{bonneau2014towards} show that in theory it is possible to
train users to memorize system-generated passwords with 56 bits of
unpredictability.  System-generated versions of the CorrectHorseStapleBattery
system have also been suggested~\cite{shay2012correct}, i.e., the system
randomly selects the words on behalf of the user. It was observed in this work
that one can perform some kinds of typo checking on behalf of the user, should
one pick an appropriate dictionary.

In the context of system-generated passwords,  Shay et
al.~\cite{shay2012correct} performed a study on Mechanical Turk and explore login
failure rates for generated random passwords, pronouncable passwords~\cite{gasser1975random} and
CorrectHorseStapleBattery-type passphrases. They report that
7.9\% of password entries failed due to capitalization typos, and another
15.2\% are simple typos (edit distance one). They also report a correlation of
typo likelihood with password/phrase length.
They suggest some simple strategies for tolerating typos here:
ignoring white spacing and, for CorrectHorseStapleBattery passphrases, checking
for the nearest dictionary word. Such a mechanism

\tnote{Need to add into the above a reference to that one-author tech report on
horsestaple type passphrases, and typo tolerance}


\begin{itemize}
\item Typo tolerant order independent Hashing.~\cite{bard2007spelling}
\item Error-correcting and hiding all partial information ~\cite{1543961}.
\item Fuzzy Extractor~\cite{dodisetal:2004,juelswattenberg:1999}.
\item Usability of Passphrases and error tolerance in that~\cite{shay2012correct}.
\end{itemize}




%%% Local Variables:
%%% mode: latex
%%% TeX-master: "main"
%%% End:
