%%%%%%%%%%%%%%%%%%%%%%%%%%%%%%%%%%%%%%%%%%%%%%%%%%%%%%%%%%%%%%%%%%%%%%%%%%%%%%%%
\subsection{Computing $\greedylambda_q$}
\label{sec:faster-attack}

%%%%%%%%%%%%%%%%%%%%%%%%%%%%%%%%%%%%%%%%%%%%%%%%%%%%
We start by showing that computing the optimal $q$ guesses to make
against a relaxed checker is NP-hard in general. Later, we present an
efficient approximate algorithm for the problem.


\begin{definition} {\bf Best-$q$-guess.} Given a function
  $\ball:\strings \rightarrow \PW^*$, a password distribution
  $\pwprob$ over $\PW\subseteq\strings$, and a query budget $q$, find
  a $q$-size subset $P\subseteq \strings$, such that
  $\sum_{\pw\in C'}\pwprob(\pw)$ is maximized, where
  $C' = \bigcup_{\pwtypo\in P}\ball(\pwtypo)$.
\end{definition}

\begin{definition} {\bf Maximum coverage problem.} Given a ground set
  $E=\{e_1, e_2,\ldots e_n\}$, a collection of $m$ subsets of $E$,
  $S = \{S_1, S_2,\ldots,S_m\}$, and a weight function
  $\gamma:E\rightarrow \R^+$ that assigns weights to each element
  $e\in E$, find a subset $C\subseteq S$ of size $k$, such that
  the following quantity is maximized, $ \sum_{e\in C'}\gamma(e) $,
  where $C' = \bigcup_{S_i\in C} S_i$.
\end{definition}

The maximum coverage problem is known to be
NP-hard~\cite{feige1998threshold}. We can thus prove the
following theorem.

\begin{theorem}
  \label{th:finding-best-q}
  If there is a polynomial time algorithm for best-$q$-guess, there
  exists a polynomial time algorithm for maximum coverage
  problem.
\end{theorem}

\noindent\proof We shall show a polynomial time reduction from the maximum coverage
problem to the best-$q$-guess problem. To start with, we are given an
instance of maximum coverage problem with $(E, S, \gamma, k)$, and we
want to construct an instance of best-$q$-guess problem.  To do so, we
set $\PW = E$ and probability $\pwprob$ as proportional to
$\gamma$. (We might have to normalize $\gamma$ to make it a
probability distribution.)  The function
$\ball:\strings \rightarrow \PW^*$ is defined as follows.  First add
to $\strings$ a set $W^* = \{\pwtypo^*_i\}_{i=1}^m$, with
$\pwprob(\pwtypo^*_i) = 0$, and for each $S_i\in S$, set
$\ball(\pwtypo^*_i) = S_i \,\bigcup\, \{\pwtypo^*_i\}$. For all other
$\pwtypo\in \strings\setminus W^*$, let
$\ball(\pwtypo) = \{\pwtypo\}$. This is a valid instance of the
best-$q$-guess problem, and if we can find an polynomial time
computable solution to this, we can solve the maximum weighted set
cover problem in polynomial time. \qedsym

% The best
% guesses output by the solver can be mapped back to the best set of $k$
% sets which maximizes the cumulative weight.  The reduction is in
% polynomial time.


% where $\pwtypo_i\in \strings$.

%  For Best-$q$-guess problem
% we have a checker $\checker$ that checks every password in a ball
% $\ball(\pwtypo)\subseteq\PW$ on any submitted password $\pwtypo$. Let
% $E=\PW$ be the set of elements, and
% $S_i = \ball(\pwtypo_i)\subseteq \PW$. Now, in our context
% $S = \{\ball(s)\cap\PW\,|\,s\in \strinsg\}$. The weight function is
% $\pwprob$ that assigns a probability to each $\pw\in \PW$.  The
% problem is asked to find a set $P\subset\strings$ of size $q$ such
% that $\sum_{\pw\in\C'}\pwprob(\pw)$ is maximized, where
% $C'=\bigcup_{s\in P}\ball(s)$.


% Given any set cover problem with parameter $(E, S, \gamma, k)$ we can
% reduce that to optimal $q$-guessing problem by the way we just
% described. So, if there is a polynomial time algorithm for the later
% than so would be for the former. But, maximum weighted set cover is a
% well known NP-hard problem, so is $q$-guessing problem.


Nevertheless, a greedy algorithm can achieve
$(1-{1\over e})$ times the optimal $\fuzzlambda$ value and, as shown by  Feige~\cite{feige1998threshold,nemhauser1978analysis}, yields
an optimal approximation
for the maximum coverage problem. Na\"{i}ve implementation of this
greedy algorithm for best-$q$-guess, however, requires searching over
exponentially many strings in $\strings$. We can exploit the fact that in our setting, a small number of correctors are used, and these correctors are efficiently invertible. Additionally, password weights are highly non-uniform. Thus it is efficient to enumerate all balls above a certain threshold weight, yielding the efficient implementation of Feige's greedy
approximation algorithm specified in \figref{fig:attack-algo}. The algorithm intuition is this: as an invariant, in any iteration of the
external while loop, all new balls (balls of $\pwtypo$) pushed onto
the heap have max-weight password $\pw$, and
hence total weight $\le b\cdot \pwprob(\pw)$. This observation enables a global selection of balls in weighted order.


\iffalse
Here we give an algorithm for more efficiently computing $\greedylambda_q$. We
also discuss more the relationship between finding optimal query strategies and 
the weighted max set cover problem. 

Recall the greedy algorithm we sketched in 
\secref{subsec:security}. The problem with that algorithm is it iterates
over all the strings in the set $\strings$ which could be huge, and
%the ball around many of those strings will have zero induced
probability mass.  

In our game setting let $\checker$ the checking algorithm which uses
$\typoset$ correctors. The attacker assumes that the distribution of
passwords is $\pwprob$.  Note if $\pwprob$ is indeed the same as the
real password distribution then we call this attacker an exact-knowledge
attacker, otherwise an estimating attacker.  For a submitted password
$\pwtypo$, recall $\ball(\pwtypo)$ is the set of password that is
checked by the checker.  We define another variable called the
neighbor of a password $\pw$, which denotes the set of
$\pwtypo\in\strings$ which contain $\pw$ in their balls.  Let
$n = \max_\pw{|\nh{\pw}|}$, $b = \max_{\pwtypo} |\ball(\pwtypo)|$, and
$A = \{\pwtypo \in \strings \:|\: \exists \pw\in\PW,\,
\pwprob(\pw)>\varepsilon \mbox{ and } \pwtypo\in\nh{w}\}$.
{\claim Given a checker $\checker$ that checks maximum $b$ passwords
  for a submitted password, and a password will end up being be
  checked by submitting at most $n$ different strings, the top $q$
  guesses can be found within top $(q-1)nb+1$ strings of $A$ sorted
  based on ball
  weights}.  \\
\Proof Let, $A'$ be the set of $(q-1)nb+1$ heaviest balls of A. Each
guess can result in removing at most $b$ passwords from the set $P$,
each of which, in turn, can affect the weights of maximum $n$
different balls in $A'$. So, each guess can decrease the weights of at
most $nb$ balls. So, after making $q-1$ guesses, at least one string
$\pwtypo'$ will remain in $A'$ whose weight is not changed over the
course of the algorithm, and by definition of $A'$, there cannot be
any ball in $A\setminus A'$ that has higher weight than the ball
$\pwtypo'$. So, the algorithm will never require to choose any string
from $A\setminus A'$.

The attacker can utilize the aforementioned claim to expedite the
search for the best guesses drastically. The modified algorithm is
given in~\figref{fig:attack-algo}. In the algorithm the external while
loop, will be iterated at most $(q-1)bn$ times.  Another observation
in the algorithm, that in the process of generating the potential
guess list $A$, the attacker can emit a guess if the ratio of the
heaviest ball in $A$ and the recently inserted password probability is
more than $b$.  The idea is the any subsequent ball can have weight at
most the weight of the current ball times the size of the maximum
ball. So, if that product is less than the weight of heaviest ball in
A, the heaviest ball will be the attacker's next best guess.  Note,
the value of the $\varepsilon$ is clear from the algorithm.

In the
worst case (if the password distribution is close to uniform) then the
time complexity will end up being checking all the passwords in $\PW$
times $nb$.  But in practice, thanks to very skewed distribution of
user chosen passwords, the attack will terminate much faster than
that.

\fi

% \rcnote{Why? All new balls will have
%   lower weight than $b\times$probability of current $\pw$.}
% for w in B(wtilde_m)
%    for wtilde in N(w)
%
%for w in {wtilde in A | \exists w in B(w_m) and wtilde in N(w)}
%

\begin{figure}[t]
  \centering
  \def\aindent{\hspace*{10pt}}
  \fpage{0.4}{
    \small
    \underline{nextPw()}\\
    \textnormal{Returns the passwords in $\PW$ in
    decreasing order of their probability $\pwprob$.}\\[3pt]
    \underline{FindGuesses$(q)$}\\[2pt]
    /* $\ball(\pwtypo) =$ ball around $\pwtypo$, and  $b = \max_\strings |\ball(\pwtypo)|$ */ \\
    /* $\nh{\pw} = \{\pwtypo \,|\, \pw\in\ball(\pwtypo)\}$ */\\
    $P\leftarrow \PW$\\
    $A\leftarrow$ MaxHeap() \aindent /* val$(\pwtypo) = \pwprob(\ball(\pwtypo)\cap P)$ */\\
    $g\leftarrow \phi$; \\
    do $\{$ \\
    \aindent $\pw \leftarrow \mbox{nextPw}()$\\
    \aindent $\pwtypo_m \leftarrow A.\mbox{popmax}()$\\
    \aindent while $\;\pwprob(\ball(\pwtypo_m)\cap P)\ge b\cdot\pwprob(\pw)\; \{$\\
    \aindent \aindent $g\leftarrow g\cup \{\pwtypo_m\}$\\
    \aindent \aindent $P \leftarrow P\setminus \ball(\pwtypo_m)$\\
    \aindent \aindent foreach $\;\pwtypo \in \{\pwtypo\in A\,|\, \ball(\pwtypo)\cap\ball(\pwtypo_m)\cap P \ne \phi\}\;$\\
    \aindent \aindent \aindent $A.\mbox{updateweight}\left(\pwtypo\right)$\\
    \aindent \aindent $\pwtypo_m \leftarrow A.\mbox{popmax}()$\\
    \aindent $\}$\\
    \aindent $A.\mbox{heappush}(\pwtypo_m)$\\
    \aindent foreach $\;\pwtypo \in \left(\nh{\pw}\setminus A\right)\; $\\
    \aindent \aindent $A.\mbox{heappush}(\pwtypo)$\\
    % \aindent \aindent if $\;A.\mbox{length}() > (q-1)nb\;$ $\{$\\
    % \aindent \aindent\aindent $\pwtypo' \leftarrow A.\mbox{popmin()}$\\
    % \aindent \aindent $\}$\\
    % \aindent $\}$\\
    \} while ($|g|<q$)\\
    return $g$
  }
  \caption{Figure presents a greedy algorithm to compute the best $q$
    guesses and thereby compute $\greedylambda_q$, for an attacker who
    estimates the the password distribution with $\pwprob$. % If the
    % estimated distribution matches with real distribution, we call the
    % attacker as optimal attacker, otherwise estimating attacker.
  }
  \label{fig:attack-algo}
\end{figure}











%%% Local Variables:
%%% mode: latex
%%% TeX-master: "main"
%%% End:
