%%%%%%%%%%%%%%%%%%%%%%%%%%%%%%%%%%%%%%%%%%%%%%%%%%%%%%%%%%%%%%%%%%%%%%%%%%%%%%%%
\section{Modeling Typos and Password Checkers}
\label{sec:formal}


In the past sections we saw that typos account for a large fraction of login
failures. This burdens users and can even cost companies money because users end
up not engaging with their services. We've also seen how correcting passwords on
user behalf can efficiently increase usability. The remaining question is
whether one can do so securely. We now answer this question. 

We start by giving a formal framework for typo-tolerant password checking, and
show how to realize the checking schemes suggested by our studies above. We
will also show, via what we refer to as the ``crossover theorem'' that one can
often build typo-tolerant checking schemes for which there is \emph{no security
loss}. This theorem depends upon the nature of the distribution of passwords 
and so we finally turn to empiricism using password leaks to investigate show
that typo-tolernace comes with insignificant security loss for the kinds of 
passwords used in practice.

%We will also introduce more subtle mechanisms, such as blacklisting of popular
%passwords 

%we give a framework for reasoning about and specifying typo-tolerant password checkers. 
%We will formalize the notion of
%a typo model, which captures the probability of particular errors
%arising across a user population. We will then describe a design for 
%typo-tolerance that takes advantage of a typo distribution model to efficiently
%check for high-probability typos. Along the way we will define the threat model
%and highlight the key issue with typo-tolerance:  the trade-off between
%usability and security.

\iffalse
A distance measure $\dist$ is a function
$\dist\Colon\PW\times\PW\rightarrow\R^+$.  We will often denote
$\dist(\pw,\pwtypo)$ by $\dist_\pw(\pwtypo)$.  The neighborhood of
$\delta$-close points of a password $\pw \in \PW$ is the set
$\ball_{\dist,\delta}(\pw) = \{\pwtypo \;|\; \pwtypo \in \PW
\textnormal{ and } \dist_\pw(\pwtypo) \le \delta\}$.
When $\dist$ and $\delta$ are clear from context we will write simply
$\ball(\pw)$ and call this the ball centered at $\pw$.
\fi

\subsection{Typos, balls, and neighborhoods}
\paragraph{Typo model.} Let $\PW$ be the set of all passwords, i.e., strings
over some alphabet $L$ up to a maximum length $\pwlength$. We might take
$L$ to be the set of all printable ASCII characters and the maximum length $\pwlength$ to be
sufficiently large to encompass user-selected passwords, e.g., $\pwlength = 64$. We associate with $\PW$ a
distribution~$\pwprob$ that models the probability of user selection of
passwords; thus $\pwprob(\pw)$ is the probability that a user selects a given password $\pw$. We assume for simplicity that passwords are independently drawn from~$\pwprob$. 

A key feature of our approach is that we do not appeal to a lexicographic notion of distance (e.g., Levenshtein distance) to model typos. Instead, we rely on a simple, key observation: Some typos are more likely than others. Thus we reason about typos in terms of probabilities, rather than distances. Specifically, we let $\typoprob_{\pw}(\pwtypo)$ denote the probability that upon authenticating, a user with password $\pw$ types $\pwtypo$. (If $\pwtypo \neq \pw$ then $\pwtypo$ is a typo; $\typoprob_{\pw}(\pw)$ is the probability that the user makes no typo.) We write $\pwprob(P)$ to denote the aggregate probability on a set $P \subseteq \PW$ of passwords.

For all $\pw \in \PW$, then, $\typoprob_\pw(\cdot)$  defines a probability space
over $\PW$. That is, $\typoprob_\pw(\pwtypo) \in [0,1]$ for any $\pwtypo$ and 
$\sum_{\pwtypo \in \PW} \typoprob_{\pw}(\pwtypo) = 1$. In practice, generally $\typoprob_\pw(\pw) > 0$, i.e., users will sometimes enter passwords correctly. Also, for many password pairs, $\typoprob_{\pw}(\pwtypo) \neq \typoprob_{\pwtypo}(\pw)$. For example, a user may mistype her password $\pw =$ ``Unlockme'' as $\pwtypo =$``unlockme'' as a result of accidentally failing to depress the SHIFT key, while a user whose password is $\pwtypo =$``unlockme'' is less likely to press SHIFT accidentally and capitalize the letter `u'.

Implicit in our model is the simplifying assumption that typos depend only on a user's password $\pw$ and
not, for example, on the user that typed them, the time of day, or other factors.
As we will see, this assumption simplifies operationalization of typo
tolerance models. As one example, modeling individual users' typo habits would require a server to record the user's typo history.
While higher-accuracy correction for the user might then be possible, this feature would, of course, result in a more complex system. It could also leak password information: Recoding the fact that a user fails to capitalize the first character in her password leaks the fact that that character is a letter.

\paragraph{Neighborhoods.} We define the neighborhood $\nh{\pw}$ of a password
$\pw$ as the set of passwords that a user with $\pw$ may type in attempting to
enter $\pw$. Thus $\nh{\pw} = \{\pwtypo \,|\, \typoprob_{\pw}(\pwtypo) > 0\}$. 

Some passwords in $\nh{\pw}$ may have very low associated probabilities,
reflecting typos that rarely occur in practice. In this paper, we focus on and
characterize common typos, so a more restrictive definition of neighborhoods is
helpful. We define a {\em mutation} $\mut: \PW \rightarrow \PW$ as a (possibly
randomized) function that models a specific form of typo. For example, $\mut$
might flip the case of the first letter in $\pw$, modeling the form of typo in
the example above that transforms $\pw =$ ``Unlockme'' into $\pwtypo
=$``unlockme.'' 

Let $\mutset = \{\mut_0, \mut_1, \ldots, \mut_{a-1}\}$ denote a set of $a$
mutations, where $\mut_0$ is specially designated as the identity function
(modeling correct entry of a password). We let $\newnh{\pw} =
\{\mut_i(\pw)\}_{i=0}^{a-1}$ denote the {\em mutation-set neighborhood} induced
by the set $\mutset$ of mutations. This restricted definition of a neighborhood
is helpful in reasoning about common typos. We use the term ``neighborhood''
rather than ``mutation-set neighborhood'' where the context is clear.

\paragraph{Balls.} We define the {\em ball} of a password / typo $\pwtypo$ as
the set of passwords $\ball(\pwtypo) = \{\pw \,|\, \pw \in \PW \mbox{ and }
\typoprob_{\pw}(\pwtypo) > 0\}$.  Given a submitted password $\pwtypo$,
therefore, $\ball(\pwtypo)$ is the set of passwords $\pw$ that a user {\em might
have meant to type when actually entering $\pwtypo$}. 

A ball may be thought of as the inverse of a neighborhood: If $\pwtypo \in
\nh{\pw}$, then $\pw \in \ball(\pwtypo)$, and vice versa. Note that a ball may
lie outside the support of $\pwprob$: A password may never be registered by
users but still shows up as a typo.

Similarly, we may think of the inverse of a mutation as what we call a {\em
typo-corrector}, a (possibly randomized) algorithm $\tcf$ that takes as input a
password in $\PW$ and outputs a corrected password (also in $\PW$).  
For instance, $\tcf$ might correct the example mutation above in
which a user fails to capitalize a leading letter. In this case, if $\pw$ begins
with a letter $\ell \in L$, $\tcf(\pw)$  capitalizes $\ell$ and outputs the
result, while if $\ell$ is not a letter, $\tcf(\pw) \rightarrow \pw$. 

Just as some mutations may be especially common, making it helpful to reason in
terms of mutation sets, so typo-correctors may be especially helpful. In fact,
we focus on typo-correction strategies that employ just a few select
typo-correctors. We define a set of typo-correctors as $\tcfe = \{\tcf_0,
\ldots, \tcf_{\chbudget-1}\}$, where $\tcf_0$ is specially designated as the
identity function. We define the {\em typo-corrector ball}
$\newball_{\tcfe}(\pwtypo) = \{\tcf_i(\pwtypo)\}_{i=0}^{\chbudget-1}$.  We use
the term ``ball'' rather than ``typo-corrector ball'' where the context is
clear, and also write $\newball(\pw)$, dropping the subscript $\tcfe$ where
appropriate.

\paragraph{\em Remark:} While
$\pwtypo \in \nh{\pw} \Leftrightarrow \pw \in \ball(\pwtypo)$, of
course it is {\em not} the case that
$\pwtypo \in \newnh{\pw} \Leftrightarrow \pw \in
\newball_{\tcfe}(\pwtypo)$
for every $\mutset$ and $\tcfe$. Moreover, in practice some typos /
mutations that occur relatively frequently in practice are hard to
correct. This is true, for example, of the mutation $\mut$ that omits
the last letter of a password $\pw$ (which was among the top four most
common typos observed in our MTurk experiments, as reported
below). There is no effective general way to correct this typo, e.g.,
no deterministic typo-corrector $\tcf$ that corrects this typo for all
passwords.

\begin{figure*}[t]
\centering
\begin{minipage}[b]{.35\textwidth}
  \includegraphics[width=1\textwidth]{images/neighborhoods.pdf}\vspace{-7mm}
\caption{Neighborhoods for passwords ``Password!'' and ``Password1'' under example mutation set $\mutset$.}\label{fig:neighborhoods}
\end{minipage}\qquad
\begin{minipage}[b]{.35\textwidth}
 \includegraphics[width=1\textwidth]{images/ball.pdf}\vspace{-7mm}
\caption{Ball for password ``password1'' under example typo-corrector set $\tcfe$.}\label{fig:ball}
\end{minipage}
\end{figure*}

\paragraph{Example.} We give a simple example for a mutation set $\mutset = \{\mut_0, \mut_1, \mut_2\}$ such that $f_1$ flips the capitalization of the first letter of a password and $f_2$ changes a terminal symbol to a number,  and a typo-corrector set $\tcfe = \{\tcf_0, \tcf_1, \tcf_2\}$ where $\tcf_1$ and $\tcf_2$ that respectively fix $\mut_1$ and $\mut_2$. Figure~\ref{fig:neighborhoods} show neighborhoods for passwords ``Password!'' and ``Password1,'' while Figure~\ref{fig:ball} depicts the ball for ``password1.'' Note that the two neighborhoods include ``password1'' in their intersection, which is why ``Password!'' and ``Password1'' are both in the ball for ``password1.''

\subsection{Password checkers}
A password checker consists of two algorithms:

\begin{itemize}
\item $\register$ is a (possibly randomized) password registration function. It takes as input a password $\pw$ and outputs a string $s$ that may, for example, be the output of a password hashing scheme like scrypt. 
\item $\checker$ is a (possibly randomized) password verification function. It takes as input a password $\pwtypo$ and a stored string $s$, and outputs a boolean value, either $\accept$ or $\reject$.
\end{itemize}

More generally one might return a value
in $[0,1]$ which acts as a confidence in how close is $\pwtypo$ to the
registered password. This would be useful in systems that combine multiple
contextual authentication signals to make an authentication decision. We will
focus on the boolean case, but our techniques extend in natural ways
to confidence values (e.g., by returning $\typoprob_\pw(\pwtypo)$).  A
password checker is deterministic if $\checker$ is deterministic.

We measure utility of a typo-tolerant checker by the probability that
the checker outputs true for entered passwords, including some
typos. Formally, the acceptance utility is equal to
$\utility = \Pr[\ACC(\checker)\Rightarrow \true]$, where $\ACC$ is the random
variable defined by the output of the pseudocode experiment described
in \figref{fig:acc-security}. There $\getpwp$ means sampling from the
set according to $\pwprob$, and $\gettypo$ means sampling from the set
according to $\typoprob_\pw$.  The game is (implicitly) parameterized
by the scheme algorithms, the password distribution $\pwprob$, and the
typo model $\dist$.

\begin{figure}[t]
\center
\fpage{.20}{
   \underline{$\ACC(\checker)$}\\
   $\pw \getpwp \PW \next \pwtypo \gettypo \PW$\\
   $s \getsr \register(\pw)$ \\
   $b \getsr \checker(\pwtypo,s)$\\
  Return  $b$
}\hspace{0.1in}
\fpage{.20}{
\underline{$\ATT(\checker,\advA, q)$}\\[1pt]
$\pw \getpwp \PW \next s \getsr \register(\pw)$\\
$\advA^{\CheckOr}$\\
Return $\win$\medskip

\underline{$\CheckOr(\pwtypo,s)$}\\[1pt]
$b \getsr \checker(\pwtypo,s)$\\
If $(b = \true)$ then \\
\myind $\win \gets \true$\\
Ret $b$
}
\caption{(\textbf{Left}) Experiment for defining acceptance utility
  for a typo-tolerant checking scheme $\register,\checker$.
  (\textbf{Right}) Security game for online guessing attacks.}
\label{fig:acc-security}
\end{figure}

An \emph{exact checker} is one for which $\checker(\pwtypo,s)$ never outputs one
when $\pwtypo \ne \pw$ yet $s\getsr \register(\pw)$.  In practice of course,
exact checkers actually have a non-zero, but cryptographically small probability
of false acceptance (for typical hash-function-based checkers, this small
probability is equal to the probability of having found a collision in the hash
function).  We will throughout ignore this false acceptance probability.  We
note that the acceptance utility of an exact checker will always be less than 1
assuming that there exist $\pw$ and $\pw'$ for which $\pwprob(\pw) > 0$ and
$\typoprob_\pw(\pw') > 0$. That is, the probability of typos is higher than zero.

\paragraph{Typo-tolerant checkers.} We will primarily focus our 
attention on building typo-tolerant checkers that {\em relax} the checks made by an
existing exact construction. Let $\exregister, \exchecker$ be the algorithms of
an exact checker. Then an associated relaxed checker $\checker$ has the same
registration algorithm $\register = \exregister$, but is not exact, i.e., $\checker \neq \exchecker$.
Specifically, our approach will be to design a function $\register$ that individually checks each of {\em multiple candidate
passwords} in a set $S$ using $\exchecker$, where elements of $S \subseteq \ball(\pwtypo)$ represent possible typo corrections to a submitted password $\pwtypo$.

Relaxing an exact checker is a desirable approach to typo-tolerance
for two main reasons. First, it means that modifying a system to
become typo-tolerant can be done just by deploying a new checking
algorithm that works with previously registered passwords.  For
example registration may use a password hashing scheme like
scrypt~\cite{percival2015scrypt} or argon2~\cite{biryukov2015argon}, or use
a password onion construction that combines password hashing with an
off-system crypto service~\cite{everspaugh2015pythia}.

Second, from a security perspective, it implies that  compromise of the password
storage has the same effect as for the legacy system.  In particular brute-force
cracking attacks given~$s$ will work precisely as well as before. Of course,
should the typo-tolerant $\checker$ algorithm be very complex to implement, it might increase the likelihood of software implementation
vulnerabilities.  For this reason, we consider simple-to-implement relaxed checkers.
We emphasize that security could still be diminished by the adoption of relaxed checking, due to remote
brute-force attacks, a subject we will investigate in detail below.

Of course, relaxing an exact checker does come with limitations. We must run $\exchecker$ against each of the elements of the candidate set  $S \subseteq \ball(\pwtypo)$ until a match is found or none is found and $\pwtypo$ is rejected.\footnote{We note that this can be
  viewed simply as the standard brute-force construction of an error
  correction code from an error detection code.} As $\exchecker$ is designed to be
computationally expensive to thwart offline brute-force guessing attacks, this process may be slow if $|S|$ is large. We will show that one can significantly increase acceptance utility using a small set $S$ and thus few invocations of
of $\exchecker$ beyond the single one necessary to achieve perfect
correctness.

In \apref{sec:secure-sketches} we discuss some natural approaches to
typo-tolerant systems that use cryptographic constructions such as
secure sketches, but note that unfortunately these do not appear to
provide sufficient security in the face of server compromise.

\paragraph{A utility-optimized relaxed checker.} Let $pw$ denote the true password selected by a user. Then for a  submitted password (potential typo) $\pwtypo$, it is the case that $\Prob{pw = \pw \,|\, \pwtypo} = (\typoprob_{\pw}(\pwtypo) \pwprob(\pw)) / (\sum_{\pw'} \pwprob_{\pw'}(\pwtypo))$. Thus, for a given $\pwtypo$, see that $\Prob{pw = \pw \,|\, \pwtypo} \varpropto \typoprob_{\pw}(\pwtypo)\pwprob(\pw)$.

Suppose, then, that the checker designer has perfect knowledge of the distributions $\pwprob$ and $\typoprob$. 
Given a potential typographical error, i.e., $\pwtypo \gettypo \PW$, the following procedure to recover $\pw$ is possible. Order the set $\ball(\pwtypo)$ by nonincreasing value of $\typoprob_{\pw}(\pwtypo)\pwprob(\pw)$ and successively test each element of $\ball(\pwtypo)$ using $\exchecker$. If
$\exchecker$ finds a match, output $\true$ and stop, else continue
until $\ball(\pwtypo)$ is exhausted and output $\false$. Informally, this procedure is utility-optimal for a relaxed checker in the sense that, given $\pwtypo \gettypo \PW$: (1) It achieves $\utility = 1$ and (2) Among all possible relaxed checking procedures with $\utility = 1$, it achieves the minimum expected number of invocations of $\exchecker$.

Unfortunately, this procedure is impractical for a number of reasons. First, it requires a potentially huge number $|\ball(\pwtypo)|$ of 
invocations of $\exchecker$. Second, a checker designer in practice may not know $\pwprob$ and $\typoprob$ exactly, and may find it hard to estimate them for low-probability passwords or for all elements of $\ball(\pwtypo)$ (which may be large). Finally, as we have noted, there is a tension between utility and security in typo-tolerant password checking; it is not appropriate to optimize utility alone.

\paragraph{Our strategy: Relaxed checking with typo-correctors.} Given the (intentionally imposed) high computational burden of invoking $\exchecker$, achieving a practical checker design requires that only a relatively small set of candidate passwords be checked using $\exchecker$. We focus in this paper on the simple strategy of using a fixed set $\tcfe$ of typo-correctors in relaxed checking. When a password $\pwtypo$ is submitted, $\checker$ uses $\exchecker$ to test the sequence of passwords $\tcf_0(\pwtypo), \tcf_1(\pwtypo), \ldots, \tcf_{\chbudget-1}(\pwtypo)$ in order. If a match is found, $\checker$ outputs $\true$ and halts. If all checks fail, it outputs $\false$. This approach generalizes industry practice (\`{a} la Facebook, Vanguard, etc., as noted above) and, as we shall show, achieves a good balance between utility and security. 

\subsection{Security of relaxed checkers}
\label{subsec:security}
As discussed above, since we focus on relaxed checkers, attacks due to
compromise of an authentication server, such as offline cracking
attacks, are not affected by the change to typo-tolerance. The
critical question, then, is the effect of typo-tolerance on online
guessing attacks. 

Let us precisely define the notion of an online attack. In \figref{fig:acc-security} we give a simple guessing game
played between an adversary $\advA$ and a checker. The game $\ATT$ is
implicitly parameterized by $\pwprob$ and the checker
$\register,\checker$. The advantage of the adversary $\advA$ in
guessing the password is measured as
$\epsilon_\advA = \Pr[\ATT(\advA)\Rightarrow\true]$.  This security
game models a vertical attack, where the attacker tries to compromise
a randomly chosen user account; changing this security game to model
horizontal attacks is straightforward.  

First let us derive a bound on the optimal guessing attack given a
certain query budget~$q$.  Recall that if the checker is exact, we
have that $\epsilon_\advA \le \lambda_q$ for any $\advA$ making at
most $q$ queries where $\lambda_q = \sum_{i=1}^q \pwprob(\pw_i)$ 
for $\pw_1, \pw_2, \ldots$ in order of decreasing probability, i.e.,
that $\pwprob(\pw_i) \ge \pwprob(\pw_j)$ for $i < j$.  

Given a relaxed checker, in submitting a password guess $\pwtypo$, an
attacker induces checking on {\em all of the passwords in
  $\newball(\pwtypo)$}. We assume that given a $\pwtypo$, the attacker
can compute $\newball(\pwtypo)$. Intuitively, then, the optimal
strategy for an attacker for the game $\ATT(\advA,q)$ is to
iteratively guess the password $\pwtypo$ whose ball
$\newball(\pwtypo)$ has the largest aggregate probability over
passwords in $\PW$ not yet checked during the game.

More precisely, the strategy is as follows. Initialize a set
$P \gets \PW$ of possible passwords. Then repeat the following until
the query budget $q$ is exhausted. Guess a password $\pwtypo$ that
maximizes $\pwprob(\newball(\pwtypo) \,\cap\, P)$. If the query
succeeds, then the game is won; otherwise set
$P \gets P \setminus \newball(\pwtypo)$ and repeat.

Let the probability of success of this attacker be denoted by
$\fuzzlambda_q$. For $q=1$, we simply have
$\fuzzlambda_1 = \argmax_{\pwtypo \in \PW}
\pwprob(\newball(\pwtypo))$.
In other words, for $q=1$, an optimal attacker simply guesses the
password $\pwtypo$ whose ball has the highest aggregate probability,
i.e., for which $\pwprob(\newball(\pwtypo))$ is maximized. This
guessing strategy is analogous, in an exact-checking setting, simply
to guessing the most probable password. We observe that
$\fuzzlambda_1$ as defined here coincides conceptually with the fuzzy
min-entropy notion of Fuller et al.~\cite{fuller2014fuzzy}.

It is easy to see that this is optimal, meaning
$\epsilon_\advA \le \fuzzlambda_q$ for any $\advA$ making at most $q$
queries.  We can prove this in the following way. The optimal set of
$q$ guesses also contains optimal set of $q-1$ guesses for any
$q\ge 1$, so in other words. We have shown that $\fuzzlambda_1$ is
optimal. The, only thing remains to show is that if
$\fuzzlambda_{i-1}$ is optimal then, $\fuzzlambda_i$ is also optimal
for any $i\le q$. The attacker, at $i$-th iteration picks the password
that increases its success probability maximally, and as we assumed,
the previous $i-1$ guesses were optimal, i.e., $\fuzzlambda_{i-1}$ is
maximum advantage that an attacker can obtain with $i-1$ queries,
$\fuzzlambda_{i}$ will also be maximum. 

For this algorithm the time complexity is of the order of the size of
set $\PW$. We propose an alternative approach to restrict the search
space and improve the time complexity. Please see~\apref{sec:faster-attack}.

\paragraph{Worst-case security loss.} 
Let us start with the intuition for why typo-tolerance represents a potential 
security risk.  Assume for simplicity that $\pwprob$ is uniform over $\PW$ and that $\checker$ is such that
$|\newball(\pwtypo)| = \chbudget$ for all $\pwtypo$. Then moving to the typo-tolerant checker can increase the
probability of success of an online brute-force attacker (whether horizontal or
vertical) by a factor of $\chbudget$. Formally, 
$\fuzzlambda_q = \chbudget \lambda_q$ for all $q$ such that there exist $q$ disjoint balls.

It is tempting to conclude that typo tolerance will \emph{always} result in a
factor $\chbudget$ decrease in security. But this conclusion is too hasty:
$\pwprob$ is not uniform in reality and, in particular, passwords with high
mass are sparse in the universe $\PW$. Sparsity matters since a high
$\fuzzlambda_q$ depends intimately on finding passwords that are typos of many
high-probability passwords. 
%Let us look at another artificial example where there is in
%fact \emph{no} security loss. 

%\paragraph{The crossover theorem.} 
\paragraph{Free corrections theorem.} 
%While the artificial example above gives a worst-case bound on security loss, it
%is quite conservative and bears little on security in practice: human-chosen 
%passwords are not uniformly random. 
We would ideally like to have typo-tolerant checkers that
enjoy \emph{free corrections}. This means that  
its security is equivalent to the security of an exact checker.  
It is easy to come up with artificial distributions 
which admit free corrections of all typos. Specifically a distribution for
which no password's neighbor is in the ball of another password. 

Unfortunately, real password distributions do not enjoy this property. For
example, in the RockYou password leak and taking $\topfive$ as the set of
corrections to apply, one has significant overlap even amongst the  top 50
passwords. We therefore ask: for the password distributions seen in practice, 
can one achieve checkers with free corrections? The answer is yes as formalized
by the following theorem.

%hasIn particular, when no password's 
%We show that for most natural distributions one can get corrections ``for free'', at least when it
%comes to doing so without security loss. In words, the theorem below states that
%one can always build a typo-tolerant password checker such that acceptance
%utility increases over exact checking, yet the optimal advantage of an online
%guessing adversary is only as good as it would be against exact checking. 


\begin{theorem}[\textbf{Free corrections}] Fix $q > 0$, some password distribution
$\pwprob$ with support $\PWset$, 
a typo distribution $\dist$, and an exact checker $\exchecker$.
If there exists $\pw \in \PWset$ with a neighbor $\pwtypo$ for which
$\pwprob(\ball(\pwtypo)) < \pwprob(\pw_q)$,  then there exists a typo-tolerant
checker~$\checker$
for which $\ACC(\checker) > \ACC(\exchecker)$ and $\lambda_q = \fuzzlambda_q$. 
\end{theorem}


\noindent\emph{Proof:}
Consider the following typo-tolerant checker. Upon input $\pwtypo$, it first
checks whether the mass of the ball $p(\ball(\pwtypo)) > p_q$ where recall that $p_q$ is the
probability mass of the $q\thh$ most probable password (under
distribution~$\pwprob$). If it is larger, then only check $\pwtypo$ and do not
attempt typo correction. If instead the mass of the ball is smaller than $p_q$,
then check all passwords in the ball and accept the password if any match. Let
$\checker$ be the checker just described. 

Now we must prove two things about $\checker$. First that acceptance utility is
higher than just exact checking.  By assumption we have that the there exists
some password $\pw \in \PWset$ that has a neighbor $\pwtypo$ for which $0 <
p(\ball(\pwtypo)) < p_q$. Because this ball's mass is less than $p(\pw_q)$,
 $\checker$ will accept $\pwtypo$ while the exact checker will not. 
Thus $\ACC(\checker) \ge \ACC(\exchecker) + \pwprob(\pw)\cdot \dist_\pw(\pwtypo)$. 

%be that it includes a password $\pw \in \ball(\pwtypo)$ with $\pw \ne \pwtypo$
%and $\pwprob(\pw) > 0$.  This password will also have $\pwtypo$ corrected under
%$\checker$, since $\ball(\pwtypo) < p_q$. Noting that the exact checker will
%never correct this typo, we have that 



Now we prove the other part, that $\lambda_q = \fuzzlambda_q$. Assume for
contradiction that instead ... \fixme{Rahul you want to fill this in?}
\qedsym\\

A few comments are in order.  As $q$ increases, the mass of the neighbor's ball
must decrease for the proof to work. For relevant ranges of $q$ (e.g., $q \le
1000$), this condition will however be met for the highly-skewed password
distribution seen in practice. \tnote{do we want some empiricism here?} An
example setting that does not meet the conditions of the theorem is the
artificial one given above that has uniformly distributed passwords. %We note
%that this assumption is not strictly necessary for free corrections: one can
%give (again artificial) distributions which can have free corrections but do not
%meet the conditions. 

The typo-tolerant checker given in the proof may be inefficient (should balls be
very large), but it is easy to see that the checker can be made efficient by
only handling an efficiently checkable portion of any ball.  As long as that
portion contributes to improving acceptance utility, one still gets free
corrections.

A final issue is that the construction within the proof would be, as presented,
subject to timing attacks that leaks partial information about what password a user
entered to network attackers. This can be fixed easily by implementing the typo
checker in a constant-time fashion.

\iffalse
In the special case that no two password neighborhoods overlap and thus $|\ball(\pwtypo)| = 1$ for all $\pwtypo \in \PW$, it is possible to design a scheme $\checker$ with no security loss relative to the exact checker $\exchecker$ and with $\utility = 1$.

For any scheme $\checker$ (with $\utility \leq 1$), when no ball contains two passwords in the support of $\pwprob$, there is no security loss. It is easy to see that this is the case: Any password $\pwtypo$ submitted by an attacker results in at most one invocation of $\exchecker$ on a password that a user might select. In fact, there is no security degradation provided that $\fuzzlambda_q = \chbudget \lambda_q$. This much weaker condition is satisfied, for example, if the $q$ balls with the highest aggregate probabilities have the same probabilities as the $q$ most probable passwords. 

Intuitively, balls will not contain two distinct passwords $\pw, \pw'$ provided that these passwords do not have overlapping neighborhoods, a condition that occurs when the passwords are sufficiently ``distant'' from one another. Specifically, there should be no typo that maps $\pw$ onto password $\pwtypo$ and other typo that also maps $\pw'$ onto $\pwtypo$ such that $\checker$ corrects both typos. High lexicographic distance often achieves this condition in practice.

\iffalse
no two neighborhoods overlap,
typo tolerance comes at no security loss. In detail, consider a
distribution $\pwprob$ and typo model $\dist$ such that for any
$\pw,\pw'\in \PW$ with $\pwprob(\pw) > 0$ and $\pwprob(\pw') > 0$
there does not exist a $\pwtypo$ such that both
$\similar_\pw(\pwtypo) >0 $ and $\similar_{\pw'}(\pwtypo)> 0 $.  In
words, this just means that there exist no points in common between the
neighborhood of any $\pw$ and $\pw'$.  Then one can in theory build a
checker that achieves acceptance utility 1 and for which
$\fuzzlambda_q = \lambda_q$ for all~$q$.
\fi

This observation was made (using less formal reasoning) in prior
treatments on computer-generated
passwords~\cite{shay2012correct,techreport}, where one can have the
system arrange for all chosen passwords to be sufficiently ``distant''
from one another with respect to a reasonable typo model. But for
user-generated passwords it is unlikely that one gets such a
situation.

So we have two extremal, but artificial, examples of
security. Security loss could be as bad as a factor of $\chbudget$. It
could also come at no loss. The truth will be somewhere in between,
and here we must turn to empiricism. To start, we do a brief
exploration of the structure of real user passwords under a typical
distance measure, Damerau-Levenshtein distance.

\fixme{We will consider heuristics that enforce the condition that a correction ball contains only one  high-probability password. A simple one is this: If a known high-probability password is encountered while iterating through typo-correction functions, halt.}

\paragraph{Damerau-Levenshtein distance} Damerau-Levenshtein distance
is a widely used distance metric for spelling errors. This metric
evolved from the work by Levenshtein~\cite{levenshtein1966binary} and
Damerau~\cite{damerau1964technique}.  DL distance between two strings
is the minimum number of edit operations required to convert one
string into the other. Here allowed edit operations are insertion,
deletion, substitution and transposition.  Using simple dynamic
programming we can efficiently calculate DL distance between two
strings.  For our analysis we have used case independent DL distance,
where we compute the DL distance after converting both the string to
lower case.


\subsection{Sparsity of 1-DL Typos in Rockyou}
\fi




\tnote{There is a bunch of stuff commented out in latex below and in model.tex}

\iffalse

\bigskip
\bigskip
\bigskip

There are three main component of a typo tolerant password checking system. 
First, a fuzzy equality metric, that tells what is the probability that two
strings could be typo of each other.


\paragraph{Typo-tolerant password checking.}  
Typo tolerant password checking is the accepting passwords that are withing some
distance from the original password. Here, the distance is a function that maps
a pair of strings onto a real number.

\rcnote{
\begin{itemize}
\item Define {\em fuzzy-guess rank}, and {\em fuzzy-$\beta$ success rate}.
\item Formal definition of typo tolerance? \\
  The function {\tt fuzzy\_equal($w, \hat{w}$)} is an abstract function
  implemented by the servers that decides whether $\hat{w}$ can be treated as a
  proxy for $w$ for authentication or not.  In general this is just an equality
  checking, we can extended it to more flexible checks, e.g., treats the strings
  as equal if the DL-distance between the strings are under some limit.
\end{itemize}
}




\rcnote{random thoughts follow..}  However for typographical error correction we
have seen (shown in later section of the paper) several issues with simple edit
distance. Firstly, ``password'' and ``PASSSWORD'', differ by DL distance of 8,
and ``password'' and ``mypass12'' are also differ by DL distance 6, but it is
easy to see that ``PASSWORD'' is a more likely to be a typo of ``password'' than
``mypass12''. The solution to this problem is using, weighted edit distance,
where every edit operation may incur different cost instead of unit cost. So, we
can set substitution of a capitalized version of a character to lower weight,
while substitution of a completely different character to a higher weight. But
in weighted edit distance, the problem is how to decide on an optimal weight
given a set of word-pairs that are likely to be typo of each other. There are
approximate solution for this problem~\cite{hauser2007unsupervised}.

We propose a new approach for this problem. Most of the typo occurs due to
pressing different sequence of keys than what it should be. Given a string and a
keyboard layout we can find the optimal sequence of key presses that is required
to type that string. Now to find the distance between two strings we can convert
both the strings into two key press sequences and then compute the DL distance
on those sequences. E.g., ``PASSWORD'' will be converted to
``\caps -p-a-s-s-w-o-r-d'', and ``password'' will be converted to
``p-a-s-s-w-o-r-d''. So, distance between ``PASSWORD'' and ``password'' will be
only 1, while distance between ``password'' and ``mypass12'' will be 8. Also,
``AnyPass1!'' will be translated to ``\shift -a-n-y-\shift -p-a-s-s-1-\shift -1''.
The password ``PASSword123!@\#'' will be translated to ``\caps -p-a-s-s-\caps
-w-o-r-d-1-2-3-\shift -1-\shift -2-\shift -3''
\rcnote{This idea may be useful, wrote down as free writing. Will give deeper
  thought later, as for now, case independent edit distance is enough.}

\paragraph{Preliminary analysis on RockYou}
To understand the effect of typo tolerance on security, we performed a small
experiment with the publicly available password leaks, such as RockYou, phpBB,
Yahoo and Myspace.  In this experiment, for each of the leaks, we assume the
accounts in the database of the authentication server is the same as one the
leak, and the server accepts a password against an user account if the entered
password is withing 1 DL-distance of the original password.  This is a simple
model for typo tolerance, and we are intentionally leaving the details of how
one can implement this to focus on the issue of security.


We define ball $B_{w}$ of a password as the set of passwords that will be
accepted the authentication server in place of $w$.  Now if we assume that
our hypothetical authentication server is populated with the RockYou user
accounts, then for each passwords in RockYou the server accepts the ball of that
password.  In this case, the goal of an attacker is to find the set of strings
(these strings do not necessarily belong to the set of passwords in RockYou),
whose cumulative probability mass of the balls they belong to is maximum.


A naive approach to create all password variants for each ball will be super
inefficient.  The size of the ball of a password of length $\ell$ over the
alphabets $\Gamma$, can be as much as \fixme{$4*|\gamma|*\ell$}. Without
creating all these balls, optimal guesses can not be found and we cannot compute
the fuzzy-$\beta$ success rate.





[Discuss speedup for breadth-first guessing attack]


\fi







%%% Local Variables:
%%% mode: latex
%%% TeX-master: "main"
%%% End:

