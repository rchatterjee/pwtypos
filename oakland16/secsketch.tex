%%%%%%%%%%%%%%%%%%%%%%%%%%%%%%%%%%%%%%%%%%%%%%%%%%%%%%%%%%%%%%%%%%%%%%%%%%%%%%%%
\subsection{Secure Sketches}
\label{app:secsketch}
Secure sketches and fuzzy extractors, explored by Dodis et
al.~\cite{dodisetal:2004,1543961}, are designed to generate
consistent, cryptographically strong keys from noisy secrets, such as
biometric data. They may also be applied to passwords, as
typographical errors in passwords can be modeled as noise.
Dodis~et~al.~proposed two ways to construct secure sketches for the
edit-distance metric space; see Section 7
of~\cite{dodisetal:2004}. They show how to use a low-distortion
embedding for the edit-distance metric given by Ostrovsky and
Rabani~\cite{ostrovsky2007low}, and also describe a relaxed embedding
for the edit-distance metric using $c$-shingles. The security losses
for these constructions, as given in Proposition 7.2 and Theorem 7.5
of~\cite{dodisetal:2004}, are
$t(\log F)2^{O\left(\sqrt{\log(n\log F)\log\log(n\log F)}\right)}$ and
$\lceil{n\over c}\rceil\log(n-c+1)-(2c-1)t\lceil\log(F^c+1)\rceil$
respectively. Here, $n$ is the size of the password, $F$ is the
alphabet size, $t$ is the number of errors/edits tolerated, and $c$ is
a construction parameter denoting the size of the shingles. In our
setting, typical values would be $n=8$, $t=1$, and $F=96$. The value
of $c$, according to Theorem 7.4, should be 1 in our setting (and the
loss is an increasing function in $c$). Given these parameters, the
entropy loss of the two secure sketches would be $\approx 91$~bits and
$\approx 31$~bits respectively. The min-entropy of real world password
distributions is only about $\le 8$~bits~\cite{bonneau12}. Thus known
constructions provide no security guarantees in our context, and
providing proven constructions that do would seem to require new
techniques.




%% \begin{figure*}[t]
%%   \centering
%%   \small
%%   \begin{tabular}[t]{p{2.0in}lrrrr}
%%     \toprule
%%     \multirow{2}{*}{\textbf{Typo type}}     & \multirow{2}{*}{\textbf{Corrector}} & \multicolumn{3}{c}{\textbf{\% of typos}}\\\cline{3-6}
%%     &&{\bf general} & \pbox{1in}{\bf general\\(adjusted)} & \pbox{.8in}{\bf touchscreen \\device} & {\bf (New) \% of typos}\\\midrule
%%     Case of all letters flipped & \swcall & 30.9\% & 9.5\% & 8.3\% & 17.8\%\\
%% %    \midrule
%%     Case of first character flipped & \swcfirst &3.8\% & 4.6\% & 4.8\% & 7.8\% \\
%% %    \midrule
%%     Added extra character at the end & \rmlast &3.5\% & 4.6\% & 6.6\% & 3.7\%\\
%% %    \midrule
%%     Added extra character at the front & \rmfirst &  1.0\% & 1.3\% & 1.1\% & 1.1\%\\
%% %    \midrule
%%     Missed shift key for the last symbol &  \dtoslast & 0.2\% & 0.2\% & 0.0\% & 0.1\%\\
%%     % \midrule
%% %    \midrule
%%     Proximity errors & n/a & 17.3\% & 22.7\%  & 29.6\% & 19.0\%\\
%%     %Character replaced w/ nearby character on US keyboard 
%% %    \midrule
%%     Transcription errors & n/a & 2.3\% & 3.1\% & 3.4\% & 3.0\%\\
%% %    \midrule
%%     Other errors & n/a & 41.0\% & 54.0\% & 45.7\% & 46.5\%\\ 
%%     % Upper case to Title case and vice verse  &  \upncap & 0.2 & 0.3\%\\
%%     % \rownumber. & \stodlast & Last character symbol-to-number & 1\%\\
%%     % Change the shift state of the last non-letter character & \swslast& 0.3\%\\
%%     \bottomrule
%%   \end{tabular}
%% % \hspace{0.2in}
%% %   \begin{tabular}[t]{lr}
%% %     \toprule
%% %     \textbf{Correctors} & \textbf{\% of typos}\\\midrule
%% %     \swcall & 8.3\%\\
%% %     \swcfirst & 4.8\%\\
%% %     \rmlast & 6.6\%\\
%% %     \rmfirst & 1.1\%\\
%% %     \dtoslast & 0.0\%\\
%% %     Proximity errors & 29.6\%\\
%% %     Transcription errors & 3.4\%\\
%% %     Other errors  & 45.7\%\\\bottomrule
%% %   \end{tabular}
  
%%   \caption{The top kinds of typos observed in our collected data from
%%     the MTurk experiment over 100,000 passwords drawn from Rockyou,
%%     out of which 5,554 were mistyped.  The column labeled corrector
%%     identifies a correction function that can be used to catch the
%%     associated class of typos, when possible.  The right three columns
%%     are the percentage of all typos that were of the indicated type.
%%     The column under general (adjusted) is the one we obtained after
%%     discounting the caps lock errors that propagate over other
%%     passwords in same HIT.  This brings down the total number of mistyped
%%     password to 4,125. The right most column shows the percentage of
%%     typos broken down by their types that we saw in the data collected
%%     from touchscreen devices.  We see 2,075 typos among 23,098 typing
%%     events. }
%% \label{fig:top10-typo}
%% \label{fig:top-typo-mobile}
%% \end{figure*}





%%% Local Variables:
%%% mode: latex
%%% TeX-master: "main"
%%% End:


