%%%%%%%%%%%%%%%%%%%%%%%%%%%%%%%%%%%%%%%%%%%%%%%%%%%%%%%%%%%%%%%%%%%%%%%%%%%%%%%%
\section{Modeling Typo Tolerance and its Security}
\label{sec:model}

\def\E{{\mathcal{E}}}

Authentication failure due to small mistake in typing a password is
frustrating. We want to design an authentication system that will accept a
password even if it is off by some small typos. In this section we shall
formalize the notion of typo tolerance. The basic idea is when the entered
password is not correct, the authentication server will try to find few possible
transforms of the that password and check if any of them is the correct password,
and if the server can find possible correct password within the acceptable
mutations, it will let the user login without asking the user to retype the
password. Obviously accepting variants of password will lead to degradation in
security. But we claim, that we can find an optimal set of the mutations that
offers higher typo tolerance to probable typos at the cost of minimal loss in
security. This will not only increase users' experience with the website, it
will increase the amount of time a user stays online in the service. We shall
explain how we measure the security and how we can estimate the loss in security
in section~\ref{sec:security}.

Let $\Gamma$ be the set of alphabets allowed by most of the password policies,
and all passwords are strings over $\Gamma$.  We call the entered or typed
password as $t_{w}$, and the stored password as $s_w$.  We assume, the passwords
are not stored in plain text, rather the passwords are properly salted and
hashed with a strong cryptographic hash function, and the hash digest is stored
into the database. So, we can only check the equality or inequality of an
entered password with a stored password, by salting and hashing the entered
password accordingly.  For simplicity of presentation, we shall not use hash
function, and when we say $t_w=s_w$ (or $t_w\ne s_w$), we actually mean
$h(s_w)=h(t_w)$ (or $h(s_w)\ne h(t_w)~)$, where $h$ is a cryptographically
strong one way hash function. On receiving $t_w$ for some user $u$, the server
will first retrieve the corresponding $s_w$ for the user and then check,
$t_w\stackrel{?}{=}s_w$, if the check fails, the server will apply a set of
transforms or edits on $t_w$, and try to see if any of the mutated string
matches with $s_w$. The authentication succeeds if either $t_w$ or any of its
transforms matches with $s_w$.

Transform is a deterministic function $e$ that maps a string in $\Gamma^*$ to a
string $\Gamma^*$. Let $\E$ be set of all possible transforms, and the server
only implements a small subset of these edits, which we call $E$.  Typo ($\tau$)
is also similar to transforms.  A noisy channel model of typing is following:
typing a password is equivalent of choosing a $\tau$ from $\E$ and applying that
on the actual password ($s_w$) to produce $t_w$. The server wants to revert that
noise, and recover the original password $s_w$.

We need to keep the set $E$ small for two reasons. Firstly, the number of hash
computation is linear in size of $E$ and hence large $E$ will impose higher
computational cost on the authentication server. Second, and the bigger reason,
is more edits implies it will be easier for an attacker to guess the password
for a user account. Also, $\beta$-success rate will increase for an attacker
with the same list of guesses.

Our goal is to find an optimal $E$ that \rcnote{left here..}




%%% Local Variables:
%%% mode: latex
%%% TeX-master: "main"
%%% End:
