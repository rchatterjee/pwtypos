%%%%%%%%%%%%%%%%%%%%%%%%%%%%%%%%%%%%%%%%%%%%%%%%%%%%%%%%%%%%%%%%%%%%%%%%%%%%%%%%
\subsection{Proofs}
\label{app:proofs}

We restate and then prove the free correction theorem from~\secref{sec:free-thm}. 

\setcounter{theorem}{0}
\begin{theorem}[\textbf{Free Corrections Theorem}] Fix some password
  distribution $\pwprob$ with support $\PW$, a typo distribution
  $\typoprob$, $0<q<|\PW|$ and an exact checker $\exchecker$.
%If there exists $\pw \in \PW$ with a neighbor $\pwtypo$ for which
%$\pwprob(\ball(\pwtypo)) < \pwprob(\pw_q)$,  
Then for \opchecker with any set of correctors $\typoset$, it 
holds that $\fuzzlambda_q = \lambda_q$. 
\label{thm:free_corr2}
\end{theorem}

\noindent\emph{Proof:}
Let $\hat{S}$ be the optimal set of $q$ strings which maximizes the
total acceptance rate for the given checker $\opchecker$.  (Note that
the order in which the queries are made does not change the success
probability.)  Let
$\ball(S) = \mathop{\cup}_{\pwtypo\in S} \ball(\pwtypo)$ for some set
$S$ of strings in $\strings$.  Recall that
$\lambda_q = \sum_{i=1}^{q} \pwprob(\pw_i)$ is the sum of the
probabilities of $q$ most probable passwords in $\PW$. On the other
hand, $\fuzzlambda_q = \pwprob\left(\ball(\hat{S})\right)$.  The above
holds because $\ball(\pwtypo)$ is the set of passwords checked by
\opchecker for a given string $\pwtypo$.


The checker $\opchecker$ ensures that the cumulative probability mass of
any ball is less than or equal to $\pwprob(\pw_q)$ whenever the size
of the ball is more than 1, but, if the size is one, the cumulative
probability can be more than $\pwprob(\pw_q)$.  So, we split $\hat{S}$
into two distinct groups $\hat{S}_1$ and
$\hat{S}_{>1}$, where $\hat{S}_1$ is the set of all strings in $S$
whose ball sizes are exactly one, and
$\hat{S}_{>1} = \hat{S}\setminus \hat{S}_1$.  We can claim following two
inequalities.
\begin{align}
  \label{eq:1}
  \pwprob\left(\ball(\hat{S}_1)\right) &\le \sum_{i=1}^{|\hat{S}_1|}\pwprob(\pw_i)\\
  \label{eq:2}
 \pwprob\left(\ball(\hat{S}_{>1})\right)&\le |\hat{S}_{<1}|\pwprob(\pw_q)  \le \sum_{i=|\hat{S}_1|+1}^{q}\pwprob(\pw_i)
\end{align}
Equation~\eqref{eq:1} is true because the
$|\ball(\hat{S}_1)|=|\hat{S}_1|$, and the right hand side is the
highest cumulative probability that any set of that size can achieve under
$\pwprob$.  Equation~\eqref{eq:2} is true because of the facts that $\pwprob(\ball(\pwtypo))\le\pwprob(\pw_q)$  for all $\pwtypo\in\hat{S}_{>1}$, and $\pwprob(\pw_i)\ge\pwprob(\pw_q)$ for all $i\ge q$. 
So, by a union bound over $\ball(\hat{S}_{>1})$,  we can achieve that inequality. 
We can add the two inequalities to obtain our  desired result.
\bnm \pwprob\left(\ball(\hat{S})\right) = \pwprob\left(\ball(\hat{S}_1)\right)  + \pwprob\left(\ball(\hat{S}_{>1})\right) \le  \sum_{i=1}^q\pwprob(\pw_i)
\enm

% \noindent\emph{Case 1:}
% If $\hat{S} = \PW_q$, then by construction of $\opchecker$ every ball
% around strings in $\PW_q$ will only contain the string itself.
% So, $\cup_{\pwtypo\in\hat{S}}\ball(\pwtypo) = \hat{S} = \PW_q$, and $\fuzzlambda_q = \lambda_q$.\\

% \noindent\emph{Case 2:}
% Now, consider the case when $\hat{S} \ne \PW_q$, then there must exist
% at least one $\pwtypo\in\hat{S}\setminus\PW_q$, because
% $|\PW_q| = |S| = q$. If the ball around $\pwtypo$ contains more than
% one password, then $\opchecker$ ensures that the weight of that ball
% is $\le \pwprob_q$. This argument is true for all
% $\pwtypo\in\hat{S}\setminus\PW_q$. 


% this string $\pwtypo$, the ball $\ball{\pwtypo}$
% might contain more than one passwords, in that case their cumulative
% sum will be less than $\pwprob_q$.


% Applying a union bound we have
% that
% $\fuzzlambda_q \le \sum_{\pwtypo\in \hat{S}}\pwprob(\ball(\pwtypo))$.
% Now, because \opchecker enforces that it only corrects as many
% passwords as have cumulative mass up to $p(\pw_q)$, we have that
% $\pwprob(\ball(\pwtypo)) \le \max\{\pwprob(\pwtypo), p_q\}$.  Hence,
% \bnm \sum_{\pwtypo\in \hat{S}}\pwprob(\ball(\pwtypo)) \le
% \max\{\sum_{\pwtypo\in \hat{S}}\pwprob(\pwtypo),\, qp_q\} \enm As per
% our assumption, if $\fuzzlambda_q>\lambda_q$, \bnm
% \max\{\sum_{\pwtypo\in \hat{S}}\pwprob(\pwtypo),\, qp_q\}>\lambda_q
% \enm Here we arrive at a contradiction because neither of the terms on
% the left hand side can be bigger than $\lambda_q$. First, by
% definition, $\lambda_q$ is the set of $q$ strings which has largest
% cumulative probability under the probability distribution $\pwprob$,
% and so $\lambda_q\ge \sum_{\pwtypo\in S}\pwprob(\pwtypo)$, for any
% size-$q$ subset of $\PW$.  Second,
% $\lambda_q = \sum_{i=1}^q \pwprob(\pw_i) \ge q\cdotsm\pwprob(\pw_q)$
% because $\pwprob(\pw_i) \ge \pwprob(\pw_q)$ for all $i\le q$.  This
% proves that $\fuzzlambda_q\le \lambda_q$.

To show strict equality, simply observe that an attacker against \opchecker 
can always choose the $q$ most probable passwords to guess
and achieve a success rate of $\lambda_q$. Thus $\fuzzlambda_q=\lambda_q$.
\qedsym\\

\setcounter{theorem}{1}
\begin{theorem} Fix $q > 0$, a distribution pair
  $(\pwprob,\typoprob)$, and a corrector set $\typoset$. Define
  \opchecker to work over $\typoset$ and let $\checker$ work for a set
  of correctors $\typoset' \subseteq \typoset$. If
  $\secloss_q(\checker) = 0$, then
  $\utility(\checker) \le \utility(\opchecker)$.
\end{theorem}
\def\Bt{\tilde{B}}
\noindent\emph{Proof:} 
First recollect utility of any checker $\checker$ is defined as
\begin{align*}\label{eq:th3eq1}
\utility(\checker) &= \Pr[\ACC(\checker)\Rightarrow \true]\\
  & = \sum_{\pwtypo\in\strings}\sum_{\pw\in\ball(\pwtypo)}\pwprob(\pw)\cdot\typoprob_{\pwtypo}(\pw),
\end{align*}
where $\ball(\pwtypo)$ is the ball of $\pwtypo$ under $\checker$.

Let assume for contradiction that there exists a checker $\checker$
which uses only the correctors in $\typoset$, achieves a
$\secloss_q(\checker)=0$ and still beats the $\opchecker$ in utility,
that is, $\utility(\checker)>\utility(\opchecker)$. Let denote a ball
of $\opchecker$ by $B(\cdot)$ and a ball of $\checker$ by
$\Bt(\cdot)$.  So, if $\utility(\checker)>\utility(\opchecker)$, then
there exists at least one $\pwtypo\in\strings$ such that
\begin{equation}
  \label{eq:th3eq1}
  \sum_{\pw\in \Bt(\pwtypo)}\pwprob(\pw)\cdot\typoprob_{\pw}(\pwtypo) >\sum_{\pw\in B(\pwtypo)}\pwprob(\pw)\cdot\typoprob_{\pw}(\pwtypo).
\end{equation}
% If
% $\pwprob(\pwtypo)>0$, then $\pwtypo$ must be in $B(\pwtypo)$ and
% $\Bt(\pwtypo)$ to meet the correctness criteria.
Now, by construction, the optimal checker $\opchecker$ selects the
$\ball(\pwtypo)$ that maximizes the utility under the constraint that
no ball of size 1 has higher cumulative mass than $\pwprob(\pw_q)$.
Here by utility we mean the sum
$\sum_{\pw\in\ball(\pw)}\pwprob(\pw)\cdot\typoprob_{\pw}(\pwtypo)$. (See
Eqn.~\ref{eq:constraints}.)  The checker $\checker$ can achieve higher
utility only if it violates one of the two constraints
in~\eqref{eq:constraints}.  The first constraint, required for
completeness, is inviolable.  The second constraint determines
security; if $\pwprob(\Bt(\pwtypo))>\pwprob(\pw_q)$ when
$|\Bt(\pwtypo)|>1$, then the security loss $\secloss_q(\checker)>0$
according to Lemma~\ref{th:secloss}. Thus there cannot exist any
$\pwtypo\in\strings$ fulfilling Eqn.~\ref{eq:th3eq1}. Thus
the assumption $\utility(\checker)>\utility(\opchecker)$ is false.
\qedsym


\setcounter{theorem}{3}
\begin{lemma}
\label{th:secloss}
For any password and typo distribution pair ($\pwprob$, $\typoprob$), checker $\checker$, and parameter
$0<q<|\PW|$, if there exists a string $\pwtypo\in\strings$,
s.t. $|\ball(\pwtypo)|>1$ and
$\pwprob(\ball(\pwtypo))>\pwprob(\pw_q)$, then 
$\secloss_q > 0$.
\end{lemma}

\noindent\emph{Proof:}  Security loss $\secloss_q>0$ implies that
$ \fuzzlambda_q > \lambda_q$. Let $\PW_q = \{\pw_1, \ldots, \pw_q\}$ and recall that
$\lambda_q = \pwprob(\PW_q)$.  Recall too that:
\bnm \fuzzlambda_q = \max_{S\subseteq\strings} \pwprob(\ball(S)), \,\,\mbox{ where }|S|=q.\enm
% We shall prove this theorem by constructing a set $S$ of size $q$ such
% that $\pwprob(\ball(S))>\pwprob(\PW_q)$.
First set $S\leftarrow
(\PW_q\setminus\ball(\pwtypo))\cup\{\pwtypo\}$.
Clearly $\fuzzlambda_q\ge \pwprob(\ball(S))$.  If we look at the union
of balls of the strings in the set $S$,
\[
  \ball(S) \supseteq \PW_q\cup\ball(\pwtypo)
\]
\[  
  \Rightarrow\pwprob(\ball(S)) \ge \pwprob(\PW_q) + \pwprob(\ball(\pwtypo)\setminus\PW_q) 
\]
Now, if $\ball(\pwtypo)\setminus\PW_q\ne\phi$, then clearly $\fuzzlambda_q\ge p(\ball(S))>\pwprob(\PW_q)$, and so $\secloss_q>0$. 

If $\ball(\pwtypo)\setminus\PW_q=\phi$, then $|S| < q$, as
$|\ball(\pwtypo)|>1$. Thus as long as there exists a password
$\pw'\in \PW\setminus S$ such that $\pwprob(\pw')>0$, we can add
$\pw'$ to $S$, resulting in $\pwprob(\ball(S)) > \pwprob(\PW_q)$. This
concludes the proof.  \qedsym




%%% Local Variables:
%%% mode: latex
%%% TeX-master: "main"
%%% End:
