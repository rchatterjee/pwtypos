%%%%%%%%%%%%%%%%%%%%%%%%%%%%%%%%%%%%%%%%%%%%%%%%%%%%%%%%%%%%%%%%%%%%%%%%%%%%%%%%
\section{Background and Related Work}
\label{sec:background}


\paragraph{Password checking and threats.} 
Traditional password-based authentication systems work as follows. A
user chooses a username and a password at registration time.  For subsequent
logins, the user submits their username and password.  Using some stored representation of the password (e.g., a salted hash), a password checking
scheme determines whether the submitted password matches the registered one.
Login is allowed only if the equality check passes. A more formal
treatment appears later in \secref{sec:formal}. 

In terms of security, two main threats arise in the context of password checking
systems. The first is online guessing attacks in which the attacker can submit
guesses to the checking system via the standard interface. The attacker might
target a particular login (a vertical attack), or try popular passwords against
multiple accounts (a horizontal attack). Here the system can employ various
countermeasures to mitigate online attacks, such as slowing down how quickly
responses are returned, locking accounts after a certain number of queries per
unit time, and using anomaly detection mechanisms to flag requests as
unauthentic based on contextual information.  The second main threat is leakage to attackers of
password hash databases due to compromise of authentication systems or
accidental data disclosure. These attackers can mount offline
brute-force attacks in an attempt to crack the passwords. Our focus will be on
online brute-force attacks as, looking ahead, our checkers will be compatible
with existing password storage schemes and thus no alter security with respect to offline
attacks. We discuss more in \secref{sec:formal}.

%In addition to offline brute-force attacks, in which one assumes the attacker
%has compromised the stored password hashes, there also exists the threat of
%online brute-force attacks. Here the attacker can only submit guesses to the
%password checking system (e.g., over the network). In a \emph{horizontal online
%attack}, the adversary tries the $\beta$ most probable passwords across a large
%number of accounts. For example, submitting the always popular password 123456 against every
%account.  In a \emph{vertical online attack}, the adversary instead queries  just
%against a single target account. In either case, these are significantly more
%expensive to the attacker than offline attacks, and as well the authentication
%system can make use of rate-limiting (slowing down how long checks
%take)~\cite{..}, locking accounts after a number of incorrect guesses,  and
%various anomaly detection mechanisms that take advantage of side-information
%about the login attempt (IP address, time of day,  the user agent string for web
%login, etc.). 
%Here Bonneau suggests using $\beta$-success rate as a metric of
%difficulty. For horizontal attacks, it gives the expected fraction of
%compromised accounts when using $\beta$ guesses per account, and for vertical
%attacks, the probability of compromising an account using $\beta$ guesses.


\paragraph{Typos in user-selected passwords.} A number of prior
measurement studies reveal the tendency of users to choose weak
passwords~\cite{bonneau2014towards,florencio2007lsw,Morris:1979:Password}
with highly skewed, heavy-headed distributions (i.e., a relatively
small number of passwords are chosen by a large number of people).
The most-stated reason is that ease-of-memorability guides users to
simple, common passwords.
%Another aspect
%of usability is, however, the potential difficulty of typing more complex
%passwords. 
While memorability is clearly a
critical aspect of usability, users may also be
reluctant to choose more complex passwords 
because entering them is difficult, error-prone, and slow. The problem may be
exacerbated by various input device form factors, e.g., mobile phone touch
keyboards.

Few works have measured the difficulty of correctly
entering user-chosen passwords. Keith et al.~\cite{keith2007usability} measured
the usability of user-selected passphrases in comparison to passwords
for a cohort of 56 undergraduate students. 
%  They delineated between
%login failures due to memory errors and typographical errors by
%setting a threshold on the ratio of the Levenshtein
%distance~\cite{levenshtein1966binary} to the password length of 0.125
%(1 typo in an 8-character password), and 
They showed that 2.2\% of entries of user-chosen passwords had a typo
(defined by thresholding via Levenshtein distance), and the rate of
typos roughly doubles for more complex passwords (at least length 7,
one upper-case, one lower-case, one non-letter). A follow-up study by
the same authors also revealed a typo rate of roughly 2\% with another
small corpus of students~\cite{keith2009behavioral}.  Their studies,
being of small scale, may not generalize to other settings, and the
authors do not analyze the types of errors subjects made.

Mazurek et al.~\cite{mazurek2013measuring} hypothesize that users may pick
weaker passwords because they are simpler to type and that more
complex passwords are harder to type. Via large-scale measurements of
a university authentication system, they show that login errors are
correlated with stronger passwords. However, they do not analyze
the nature of errors, i.e., whether they were in fact typos, typing in the entirely
wrong password, or some other problem.


\paragraph{Server-side hashing changes.} In theory a secure sketch
  ~\cite{dodisetal:2004} could be used to correct some typos in the server
  side. However, the proven bounds for existing constructions are too weak to provide
meaningful protection for our setting (in which entropy is quite
  low). More details are given in~\apref{app:secsketch}. Mehler and
Skiena~\cite{mehler2006improving} propose to allow controlled collisions in
password hashing so that, with high probability, passwords with a transposition
or substitution error hash to the same value. Both such approaches to
typo-tolerant techniques are not backwards-compatible with existing password
storage and also will degrade offline attack security.


%They measured typo rates of 2.21\% for user-chosen passwords (no
%requirements), 4.76\% for user-chosen strong passwords (at least length 7, one
%upper-case, one lower-case, one non-letter), and 19.95\% for user-selected 
%passphrases (at least length 15, one upper-case, one-lower case, one
%non-letter).
%For those failures marked as typos by this threshold ratio, they saw
%that passphrases exhibited a significantly higher typo rate than shorter
%passphrases. 

%In a follow-up work, the same authors performed a study using 52 undergraduate
%students to investigate the usability of at least length 16 user-selected
%passphrases that were suggested to consist of a sequence of words. They show
%that the typographical errors are about 2\% of login attempts,  and conclude that allowing users to
%choose passphrases for which typing is similar to typing normal text (the
%so-called word-process model) benefits typability. Note that they classify login
%failures as typographical errors via manual inspection of the contents of the
%attempt.



\paragraph{Typos in passphrase systems.}
Shay et al.~\cite{shay2012correct} perform a study of system-generated
passwords that are chosen uniformly for a user, chosen uniformly to be
pronounceable, or chosen uniformly among CorrectHorseBatteryStaple-type
passphrases~\cite{xkcd} of various word lengths. They measure typos and
investigate the correlation between password/passphrase length and typing errors,
and investigate simple typo-tolerance strategies such as ignoring case
completely and, for passphrases, combining words from a dictionary whose strings
have large pairwise Damerau-Levenshtein
distance~\cite{levenshtein1966binary,damerau1964technique}, which, in turn,
enables correction by comparison with the dictionary.  This latter suggestion is
originally due to Bard~\cite{bard2007spelling}. Later, Jakobsson and
Akavipat~\cite{jakobsson2012rethinking} suggest similar dictionary-checking-based error correction in what they call fastwords.
%But all of this was in the context of system-generated passwords, and so their
%measurements do not apply for user-selected passwords nor do their 
%error-correction approaches work in our setting.
%by Schechter et al.~\cite{schechter:2010:pen} for conventional passwords. 
%Fastwords take advantage of typo-correction
%on the client side, and build in some tolerance to word ordering errors on the
%server side. 
In contrast to the above works, we focus on arbitrary user-chosen passwords, so these previous measurements and mechanisms unfortunately do not apply to our setting.

%Frykholm and Juels~\cite{frykholm2001error} suggest a system for error-tolerant
%passwords based on remembering some fraction of previously chosen passwords.

%\fixme{Sort out discussion of~\cite{mazurek2013measuring, shay:2010:encounter}}
%Bard~\cite{bard2007spelling} and Shay et al.~\cite{shay2012correct} suggest that 
%password authentication systems that generate passphrases for users 
%might improve usability by building into these systems some error-correction
%capabilities. For example Bard gives a mechanism for 
%passphrases that uses a dictionary with words with only high pairwise
%Damerau-Levenshtein~\cite{damerau1964technique,levenshtein1966binary} 
%edit distance. Such approaches cannot work for user-chosen passwords.


\paragraph{Typo-tolerant checking in industry.} 
There have been several examples of major websites accepting slightly
incorrect versions of user-chosen passwords.  Facebook as early as
2011 accepted the correct password, the password with all letters'
cases flipped, or the password with the first letter's case flipped (if
it is indeed a letter)~\cite{muffett15,zdnet2011Facebook}.  These two
modifications correspond to errors resulting from leaving the caps
lock on (or off) or the tendency of (particularly) mobile phone
keyboards to automatically capitalize the first entered character. Early password authentication mechanisms at Amazon
allegedly ignored case and any characters beyond the eighth
position due to a bug~\cite{ghacks2011Amazon}. 
Users of Vanguard (an investment management company) reported that the answers to
security questions could have typographical errors and still be
accepted~\cite{vanguard15i}.

These companies faced significant backlash in the media and from some security
professionals~\cite{vanguard15i,ghacks2011Amazon,zdnet2011Facebook}. The
assumption underlying the criticism seems to be that accepting any variant of a
password will necessarily speed up online guessing attacks.

\paragraph{Open questions.} To summarize, before our work there was no
information available about the kinds of typos that burden users typing user-selected passwords and
whether typo-tolerant password checking systems are achievable without
degrading security. We seek to answer these questions here.

%But how much this speeds up attacks, if at all, has
%never been investigated (at least in a public forum), and it could be that ---
%perhaps counter-intuitively --- tolerating some number of typos both 
%increases usability and security. \tnote{Going to need to massage this last
%paragraph, it is currently poor}
  
  

%%% Local Variables:
%%% mode: latex
%%% TeX-master: "main"
%%% End:
