%%%%%%%%%%%%%%%%%%%%%%%%%%%%%%%%%%%%%%%%%%%%%%%%%%%%%%%%%%%%%%%%%%%%%%%%%%%%%%%%
\section{Introduction}
\label{sec:intro}

Despite repeated calls for their demise (cf.~\cite{bhos12}),
human-chosen passwords remain the primary form of user authentication
on the Internet.  A long line of investigation has shown that
passwords are easily predicted by attackers
(cf.~\cite{florencio2007lsw,bonneau12b,ur2015added}), that strength meters
offer limited improvements to security~\cite{ur2012does},
that password expiration does not increase
security~\cite{zhang:2010:security}, and that 
users have a hard time remembering complex
passwords~\cite{shay2012correct,shay2014can,bonneau2014towards,ur2012does}.

A handful of works have pointed out that complex, user-chosen
passwords are not only more difficult to remember, but also more
difficult to
type~\cite{keith2007usability,keith2009behavioral,shay2014can}.  But
these studies are quite limited, investigating neither the prevalence
nor form of typos across a wide user base.  Additional anecdotes arise
in industry, where a few web services seem to intentionally allow a
small set of
typos~\cite{muffett15,zdnet2011Facebook,vanguard15v,vanguard15i}.
Facebook currently accepts a password whether or not the user
capitalizes the first letter of their password (assuming it starts
with a letter), and whether or not they have the caps lock on. But no
information about why they do this has been published, and, more
importantly, whether this degrades security is unclear.

%Additionally, Facebook currently accepts a submitted
%password whether or not the user capitalizes the first letter of
%their password (assuming it starts with a letter), and whether or not they have
%the caps lock on~\cite{muffett15}. 

%Such typo-tolerance 
%but
%to date there have been no studies focused on the extent and nature of typographical errors
%(typos) and whether password login systems can be augmented to help users with them.
%usability such as the c with notable
%exceptions~\cite{keith2007usability,keith2009behavioral,shay2014can,techreport} discussed later, 

We provide the first detailed treatment of password typos.  We start
by measuring empirically the rates and nature of typos made by users.
We perform preliminary experiments with Amazon Mechanical Turk (MTurk)
in which we task human workers with transcribing passwords drawn from
the RockYou password leak.\footnote{We submitted our experiment design
  to our IRB, but received an exemption for lack of collecting any
  PII.}  This does not perfectly model password entry (among other
reasons, because the passwords were not the workers' own), but allows
us to collect over 100,000 submissions in short order across thousands
of workers. Our experiment provides important, basic insights into
common typographical errors. We find that a large number are proximity
errors (hitting a key near the intended one). Several other common
ones are what we call ``easily-correctable'' typos: they can all
be corrected by simple functions applied to the submitted, incorrect
password. Examples of the latter include accidentally hitting the caps lock, implementing incorrect first-letter
capitalization, adding a character to the front or end of a password,
and missing the shift key when entering a symbol at the end of a
password. These easily-correctable typos account for 20\% of the typos
observed in our MTurk study, an observation that serves as a key basis
for our work.

%Beyond these, revealing that caps lock and first-letter capitalization accounts for
%15\% of typos made and other common errors include adding a character to the end of
%the password , adding one to the beginning, and symbol-to-number mistakes.  We
%use these results to enumerate common typos that can be easily corrected, the
%capitalization errors just mentioned are all examples.

Armed with correction functions for easily-correctable typos, we
instrument Dropbox's production, Internet-scale login infrastructure.
This permits measurement of typo prevalence at scale without
changing the way Dropbox currently performs user authentication and with no
increased risk of exposure of passwords (i.e., we never store passwords or 
any information that could help in guessing them).
While we cannot
reveal the absolute number of requests seen during measurements for reasons of
confidentiality, we note that Dropbox has hundreds of millions
of customers and all user accounts were instrumented.  We first
perform a 24-hour measurement to identify login attempts involving the
easily-correctable, common typos surfaced in the MTurk study.  We find
that over 9\% of failed login attempts result from just one of three
easily-correctable typos (caps lock, first letter case, and adding a
character to the end).  We perform a subsequent 24-hour experiment to
analyze the impact of correcting just these top three typos. This
experiment reveals that~{\em 3\% of all users failed to login, but
  could have done so given correction of one of these three
  easily-correctable typos}. Many other users %that logged in
%successfully 
could have avoided multiple login attempts, significantly
decreasing the time required to login. In summary, our measurements suggest that easily-correctable
password typos represent a significant burden on users and businesses.

%result in a median of \fixme{15
%seconds} of delay between first and final successful login attempt, with
%\fixme{5\%} of users that made a typo being turned away for hours at a time. We
%emphasize that this study only checked a small number of typos, but even so they
%represent a painful burden on users.

All of this suggests typo-tolerance could bring substantial usability benefits.
What remains is to determine if such typo-tolerance necessarily degrades
security. The intuition would be that it does, because guesses might cover
multiple possible passwords. We show, however, that this intuition is flawed.
%The key question then becomes: Can this be
%partially alleviated by correcting easily-correctable typos on users' behalf
%without a substantial resulting security degradation.


We provide a formal framework that enables principled investigation. We define
\emph{typo-tolerant password checkers}, and among these a special class that we call 
\emph{relaxed checkers}. Relaxed checkers are systems that start with an existing exact
system (e.g., comparing salted bcrypt hashes or using an
encrypted password onion~\cite{everspaugh2015pythia}). The system is ``relaxed'' through a modification that additionally searches a small space of corrections  to the submitted password. This search allows easy deployment of
typo-tolerance, while ensuring that security in the face of server compromise
is as in the exact checking case (since stored values remain
unchanged). Thus we focus on analyzing online guessing attacks that
seek to maximize their probability of success 
by exploiting the extra typo checks.
%Now for remote guessing attacks, the adversary's goal is to find strings to submit
%that maximize its probability of successfully logging in. 
%%point characterize the optimal attacker's guessing strategy for any given password
%%distribution and typo correction mechanism.  
%It turns out this is an NP-hard problem reducible to the weighted max cover set
%problem, and the best approximation is to use a greedy algorithm that, in short,
%determines the sequence of guesses for which each ``ball'' of checked passwords maximizes the probability
%of the true password being checked. 


% a metric called the fuzzy $q$-success-rate, 
%which generalizes the $q$-success-rate
%metric~\cite{boztas1999entropies,bonneau12b}.
%and fuzzy
%min-entropy~\cite{fuller2014fuzzy}. 
%These measure the maximal probability of success in the first $q$ guesses against an account
%for typo-tolerant (resp.~exact) checkers. 
We prove a \emph{free corrections theorem}. It states that there exists an
optimal, fully secure typo-tolerant checker for any desired set of
corrections. Consequently: (1) The optimal remote attack up to some query
budget $q$ is
no more successful than the optimal attack against an exact checker and (2) No other
checking scheme can improve the utility of typo corrections while maintaining no
loss in security.  The key insight is that one can build a typo-tolerant checker
that forgoes corrections in the rare cases when doing so will allow checking for
multiple high-probability passwords.

Unfortunately, the optimal checker underlying the free corrections theorem must
be based on exact knowledge of the password distribution---an assumption unlikely to be 
realizable in practice. We
therefore explore the security of a number of practical typo-tolerant
checkers, such as always correcting the top three typos, checking corrections
only when they do not appear on a blacklist of common passwords, and a version
of the optimal checker that uses the RockYou password leak~\cite{rockyou:2009} to estimate the
password distribution. We then perform a number of simulations to show that
these typo-tolerant checkers improve usability while remote guessing attacks
improve negligibly, even in the worst case of attackers that somehow
know the precise password distribution. For real attackers that estimate the
distribution, our simulations suggest no
improvement in online guessing attacks. 

The contributions of this paper are the following:
\begin{newitemize}
\item We are the first to investigate the rate and nature of password
typos made by users via measurement studies using Mechanical
Turk and the production login infrastructure at Dropbox. Our work surfaces a small
set of easily-correctable typos, such as capitalization errors, that alone
prevent 3\% of users from logging in during the period of study. Correcting these few typos could therefore non-negligibly boost user access to the Dropbox service.
% during a 24 hour period.\devd{I think we can delete the "during a 24 hour period", since this is a summary.}
%
\item We introduce a formal framework for typo-tolerant password checkers
and prove a free corrections theorem that
establishes the existence, in theory, of an optimal typo-tolerant password
checker that has no loss in security over exact checking. 
%
% \rcnote{I think we should also add the optimal attack model as contribution}
\item We introduce a number of practical typo-tolerant password checkers that are
compatible with existing password storage systems. 
Simulations show that these checkers can achieve no degradation in security yet 
still significantly improve login success rates.
\end{newitemize}
Our focus is on web password ecosystems, but nothing about our 
techniques is unique to this setting. Our typo-tolerant checkers could easily
be integrated in other settings such as logging into a desktop or laptop
computer, though measurements are probably warranted to understand what are the best
typo corrections for these other settings. We leave this to future work.

\paragraph{Immediate impact of our work.}
In the course of this research, our results prompted Dropbox to deploy a caps
lock indicator on their website's password login interface. This indicator is
like the one already available in Apple OS X, but appears in all
browsers. Preliminary results suggest that it reduces caps lock errors by
about 75\%, and thus provides
significant benefit. Unfortunately, it does not eliminate caps-lock errors nor
assist with other sorts of common typos. So while our results in this paper have
already had a practical impact, we hope that they will also fuel further
advances in the mitigation of password typos.




%%% Local Variables:
%%% mode: latex
%%% TeX-master: "main"
%%% End:
