\subsection{Complexity and Typo Likelihood} 
\label{app:complexity}

Our MTurk experiments revealed a significant initial finding regarding the frequency
of typos. Typo rates in our study increased under the following three distinct
metrics relating to password complexity.

\paragraph{\em Lexical diversity in passwords:}
One might suspect that more lexically diverse passwords---ones that
include symbols, letters with different cases, numbers or some
combination thereof---would be more prone to typos.  We define four character classes: upper case letters,
lower case letters, digits, and symbols.  Now, based on how many of
the four classes of characters are present within a password we can
partition passwords into four buckets. For example, passwords
containing characters from only one of the four classes are binned as
bucket 1, passwords containing exactly two different classes of
characters are bucket 2, etc.  In our first sample of 100,000
passwords, there were very few lexically diverse passwords. RockYou has $<0.2\%$
passwords with characters from all of the four character classes.
So we sample with replacement 5,000 passwords for each bucket from
the empirical distribution of passwords in RockYou restricted to the
passwords corresponding to the bucket. We performed the same
typing experiment as described before but with new HITs created from these
newly sampled passwords. In the left graph of~\figref{fig:complexity-typo}, we present the percentage of passwords in each bucket that were mistyped.


%That said, trends we discuss
%below hold as well when restricting attention to just unique passwords. 

%\rcnote{Some discussion for this result. This is quite counter intuitive}

% Obviously this is biased by the way we sampled, which is
% according to RockYou --- more frequently typed passwords were typed more often.
% If one normalizes the frequencies of typos for a particular password 
% by the number of times it was typed, a different story emerges. This is shown in
% the right of \figref{fig:popularity-typo}. We warn, however, that most of these
% results are not significant in that most passwords were only typed a single
% time. 

%We can see that the bulk of passwords
%for the passwords grouped by their frequency of occurance in the 
%RockYou dataset. \tnote{Insert analysis}
%%%%%%%%%%%%%%%%Typing speed, time, length and typo %%%%%%%%%%%%%%%%%%%%%%%%%%%%%%%%%%%%%%%%%%
\begin{figure*}[t]
\gamesfontsize
  \begin{center}
%    \hspace{0.3in}
    \begin{tikzpicture}[scale=0.4]
      \begin{customlegend}[
        legend columns=2,
        legend style={
          align=left,
          draw=none,
          column sep=2ex,
          font=\fontsize{8}{4}\selectfont
        },
        legend entries={Samples~~~, Typos}]
        \addlegendimage{ybar, ybar legend, blue!50}
        \addlegendimage{mark=*,red!90}   
      \end{customlegend}
    \end{tikzpicture}
  \end{center}
  \vspace{0.01in}
  \centering
  \plotmturk{\complexitytypotable}{cmplcl}{xlabel={\large
      Lexical diversity}, ylabel={\large \% of samples in each
      bucket}}{totalsample}{ylabel={},
    ymax=0.20}{typo}{}{}
%   \hspace{-0.3in}
  \plotmturk{\lengthtypotable}{length}{xlabel={\large Password length}, ylabel={}}{totalsample}{ylabel={}, ymax=0.2}{typo}{}{}
 \plotmturk{\populartypotable}{freqrange}{xlabel={\large Popularity (RockYou frequency)}, ylabel={}}{totalsample}{ylabel={\large Probability of typo}, ymin=0.01, ymax=0.10}{typo}{}{}


%  \plotmturk{\typingtimetypo}{typingtime}{Avg.~time required to type (sec)}{totalsample}{}{typoa}{}{}{}  
%   \hspace{-0.3in}
 \caption{Three experiments showing typo frequency relative to various partitions of passwords into buckets. Bucket size is indicated on the left of each figure, and corresponding typo rates on the right. {\bf(Left)}   Passwords are partitioned into four buckets based on
  diversity of character types. For each bucket we report
    the percentage of samples (blue bars) that fall in that bucket and
    what fraction of those samples are mistyped (red line).
{\bf (Middle)} Passwords are categorized into buckets by increasing order of length.  {\bf(Right)} 
    Passwords are assigned to buckets by decreasing
    frequency (increasing unpopularity) in RockYou. Bucket frequency ranges are selected so that each bucket has roughly an equal
    number of samples. %The frequency ranges that we considered include 
   % \{$\le 1$, $2-11$, $12-211$, $\ge212$\}.   
  }
  \label{fig:complexity-typo}
  \label{fig:length-typo}
  \label{fig:popularity-typo}
\end{figure*}


\paragraph{\em Password length:}
We divide passwords into five groups based on their lengths, namely
$\le5$, 6--7, 8--9, 10--11, and $\ge12$. For each class, we compute
the percentage of samples that lie in that class, along with the
percentage of passwords in those samples that were mistyped.  In the
middle graph of \figref{fig:length-typo} we show these numbers for
each of the length groups. As one might expect, typo likelihood grows with password length. 

\paragraph{\em Password popularity:} We sort the list of sampled
passwords for our MTurk experiment based on their frequency counts in
the RockYou leak. (Ties were broken alphabetically.) We then split the
passwords into four buckets, adjusting their corresponding frequency
ranges to ensure that buckets are of roughly equal size. (Some
unevenness was unavoidable, as many passwords occur only once in the
Rockyou leak.) For each bucket, we present the number of mistyped
passwords in the right graph of \figref{fig:popularity-typo}.  We can
see the clear trend that passwords that are popular among RockYou
users are more likely to be typed correctly.  For example, passwords
used by more than 211 users are 1.5 times more likely to mistyped than
those used by only one user.


\paragraph{Discussion: password typing complexity.}  As noted above,
lexical diversity, length, and popularity are related
metrics. Inspection of the passwords within the various buckets used
in the charts of \figref{fig:popularity-typo} reveals that there is
significant overlap between them. As one example, 18\% of the
passwords with lexical-diversity bucket 4 {\em both} have length
$\ge 12$ {\em and} are unpopular ($f = 1$).

The three metrics together highlight different
aspects of the underlying and intuitive trend: some passwords are more
difficult to type than others. It appears, moreover, that typos are more likely to
surface in harder-to-guess passwords. Consequently, typo correction
could help encourage users to adopt stronger passwords by easing the
use of such passwords. We leave rigorous study
of this hypothesis to future work, but note that it offers further
potential motivation for our work.



%%% Local Variables:
%%% mode: latex
%%% TeX-master: "main"
%%% End:
