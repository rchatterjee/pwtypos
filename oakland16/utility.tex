\subsection{Utility improvement of the checkers}

%Typo corrector sets provide a checker with possible ways to try to
%correct a typo.  Though choosing a large number of correctors can enable 
%correcting a larger fraction of typos, it might also incur
%higher security loss with the heuristic checkers here. 
In \secref{sec:mturk} we found corrector sets whose utility was confirmed by our experiments with
Dropbox. It could be, however, that other choices
perform better for $\checkerbl$ and $\checkerapprox$. 
In order to assess utility, we build a simulation model as follows. 

Recall that in \secref{sec:formal} we defined a simple measure of
utility via the experiment $\ACC$. This measures the probability of
login success when a user chooses a password and later types it to log
in a single time. To measure this empirically, we must fix estimates
of $\pwprob$ and $\typoprob$. For the former we use the RockYou
distribution. For the latter, we will use the empirical distributions
found in the Dropbox setting (c.f.~\figref{dropbox-summary-one}).

\paragraph{User login model.} A user randomly selects a password $\pw$
from RockYou with replacement, and attempts to login by typing that
password. With probability $\alpha$ the user types the password
correctly, otherwise he mistypes the password, and the login attempt
is rejected.  The user after this can retry with a probability $\beta$
or give up trying. We simulate the Dropbox checker as one of our
heuristic checkers. The hope is if we can tune these two parameters
and generate typos using the empirical distribution we saw at Dropbox,
we should be able to regenerate the experiment results.  So, first we
runt the simulation with 100,000 random samples of user selected
passwords from RockYou, and set the Dropbox checker as $\exchecker$
with extra logging for failed logins.  We analyze the log as before
and compute the percentage of failed logins that can be corrected by
one of the top three correctors. We found that the above model with
$\alpha=0.67$ and $\beta=0.2$, the percentage of incorrect logins
corrected by the top three correctors becomes 8.78\% and percentage
increase in login is 3.13\%.  These numbers are close to what we saw
in reality, and a sanity check for our approach.

Given those two parameters we turn to evaluate our other heuristic
checkers in terms of there utility.  We analyzed the log using the
$\checkerbl$ and $\checkerapprox$ checker to infer the utilities of
those two checkers. We used simulation for the heuristic checkers
because,  it is safer to do so, and we can play with


\fixme{...} \tnote{need to explain how when calculating utility improvement, we
only need a subset of statistics needed for calculating utility, namely the
probability of not typo'ing or typo'ing in a way not covered by model}

The results of these simulations appear in \figref{fig:correctors-alpha}. 
We start with the utility improvement of any single checker function
used with $\checkerall$. This provides a sense of how much impact an individual
correction can have and appears in the left table of the figure. 
\fixme{As expected, most utility improvement comes from correcting
capitalization errors, with relatively less improvement coming from
correction of a mistakenly added first or final character, or a
number-to-symbol conversion.} 
Note that since the checkers within $\topfive$ are mutually exclusive, one can
calculate utility improvement of $\checkerall$ using $\toptwo$, $\topthree$, and
$\topfive$ just by adding the appropriate rows: \fixme{XX\%}, \fixme{yy\%}, and
\fixme{zz\%}, respectively.


\begin{figure}[t]
  \centering\gamesfontsize
  \begin{tabular}[t]{llr}
    \toprule
    \textbf{Scheme} & \textbf{Corrector(s)} & $\utilinc$ \\\midrule
    \multirow{5}{*}{$\checkerall$} & \swcall   &\\
     & \swcfirst &\\
     & \rmlast   &\\
     & \rmfirst  &\\
     & \dtoslast &\\
    \bottomrule
  \end{tabular}
    \begin{tabular}[t]{lcr}
    \toprule
    \textbf{Scheme} & \textbf{Corrector(s)} & $\utilinc$ \\\midrule
    \multirow{3}*{$\checkerbl$} & $\toptwo$ &\\
     & $\topthree$ &\\
     & $\topfive$ &\\\midrule
    \multirow{3}*{$\checkerall$} & $\toptwo$ &\\
     & $\topthree$ &\\
     & $\topfive$ &\\
    \bottomrule
  \end{tabular}

  \caption{Utility improvements for different corrector sets. \textbf{(Left)} 
    Utility of always applying a single corrector function from the top five set $\topfive$.
    \textbf{(Right)} Utility of blacklisting and approximate optimal for top
    two, three, and five correctors.}
    %utility improvement and loss ratios are given in percentage and in the final column
    %we report the ratio of utility improvement to security loss.}
    %The rightmost column represents the weight of a corrector computed
    %as ${\utilinc-1\over \secloss_{10^3}-1}$. }
  \label{fig:correctors-alpha}
\end{figure}

In the right table of \figref{fig:correctors-alpha} we give the blacklisting and
approximately optimal checkers' utility improvement for each of
$\toptwo$, $\topthree$ and $\topfive$.  \fixme{...}
A couple of trends are worth noting. First, the drop in utility improvement
moving from checking all typos and sometimes not checking due to blacklisting
(\checkerbl) or
too large of cumulative mass (\checkerapprox) is smaller than one might expect. 
For example, with $\toptwo$ utility improvement is \fixme{2.03\%, 1.93\%, and 1.83\% for the three
checkers}. The reason is in part because popular passwords are less likely to
mistyped (c.f., \figref{fig:popularity-typo}), and because of the steep drop off in
probability mass in real password distributions such as RockYou.
We will in subsequent analyses 
restrict attention to $\toptwo$ given its high utility improvement and low
security loss.  It is interesting to note, and possibly no coincidence, that
Facebook appears to implement $\checkerall$ for $\toptwo$. 

%Utility was defined in the last section; here we estimate it by
%\fixme{simulating the experiments using 
%the RockYou distribution for $\pwprob$ and for $\typoprob$ either the
%empirical frequencies of the typos associated to individual correctors as measured in the MTurk
%study (\figref{fig:correctors-alpha}) or in the Dropbox study
%(\figref{fig:dropbox-summary-one}, rightmost column).}

Finally in the simulations so far we used RockYou, and this is also used by
$\checkerapprox$ to estimate the distribution. If we change the simulations to
calculate utility using Myspace or phpBB, but $\checkerapprox$ still uses
RockYou as its estimate (and so is therefore slightly wrong), it still obtains
utility improvements of \fixme{YY\%} (Myspace) and \fixme{ZZ\%} (phpBB) for
$\toptwo$. 

 



\iffalse

%% DATA: singleton correctors
% >> RESULTS: rockyou
% name,typo_fix_rate,\fuzzlambda_beta
% rm-firstc,0.0134977053901,0.0239977804132
% n2s-last,0.00112480878251,0.00358329352536
% same,0.0,-3.20342650381e-16
% rm-lastc,0.084360658688,0.157886008343
% swc-first,0.224961756501,0.0226411233409
% swc-all,1.83456312427,0.0425743952525

  

\begin{figure}[t]
\centering\gamesfontsize
  \begin{tabular}[t]{lcrrr}
    \toprule
    \textbf{Checker} & \textbf{Set} & $\utilinc$ (MTurk) & $\utilinc$ (Dropbox) &  $\secloss_{q}$ (\%)\\
    \midrule
    \multirow{3}{*}{\checkerall} 
                     & $\toptwo$ &  0.59\%& 2.30\%& 4.5\%\\
                     & $\topthree$& 0.67\%& 3.00\%& 13.2\%\\
                     & $\topfive$ & 0.68\%& 3.19\%& 14.5\%\\
    \midrule
    \multirow{3}{*}{\checkerbl} 
                     & $\toptwo$ &  0.57\%&    & 2.9\%  \\
                     & $\topthree$ &  0.64\%&  & 8.2\%\\
                     & $\topfive$ &  0.65\%&   & 8.2\%\\
    \midrule
    \multirow{3}{*}{\checkerapprox} 
                     & $\toptwo$ &  0.52\%&  & 1.0\%\\
                     & $\topthree$ & 0.59\%& & 1.0\% \\
                     & $\topfive$ &  0.60\%& & 1.0\% \\

                     % \swcall   &1.835&3.0& 0.60\\
                     % \swcfirst &0.225&1.6& 0.14\\
                     % \rmlast   &0.084&11.6&$<0.01$\\
                     % \rmfirst  &0.013&1.7& $<0.01$\\
                     % \dtoslast &0.001&0.2& $<0.01$\\
    \bottomrule
  \end{tabular}
  \caption{Comparison of the checkers $\checkerall$, $\checkerbl$, and
    $\checkerapprox$ for three different corrector sets using the
    RockYou distribution over passwords and the MTurk and Dropbox
    measurements for typo probabilities.}
  \label{fig:compare-sets}
\end{figure}
\fi
%%(See \figref{fig:popularity-typo}.) The utility increase of $\checkerbl$
%and $\checkerapprox$ are $1.93\%$ and $1.83\%$ respectively. In the next
%section we shall see how these checkers perform in the dimension of
%security loss.



% Given a checker Chk we compute the utility of the checker in following way.
% 1. First, for a randomly sampled password \pw in RockYou,
% 2. Pick one of the correction according to their probabilities Fig 7.
% 3. If 'other' is picked, skip to step 1.
% 4. Otherwise, apply the inverse of that mutation on the sampled password, which will be our \typopw. If there are multiple options for the inverse, pick one randomly.
%     If \typopw = \pw, go to step 2
% 4. Run it


%%% Local Variables:
%%% mode: latex
%%% TeX-master: "main"
%%% End:
