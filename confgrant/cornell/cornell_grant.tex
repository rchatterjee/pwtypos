\documentclass[11pt,a4paper,notitlepage]{article}
\usepackage[margin=1in]{geometry}
\usepackage{color}
\usepackage{algorithm}

\newcommand{\comment}[1]{{\color{red}#1}}
\title{pASSWORD typos and\\ How To Securely Correct Them}
\author{Rahul Chatterjee}
\begin{document}
%\maketitle
\noindent{\bf\LARGE Application for Conference Grant} \\[4pt]
\begin{tabular}[t]{rl}
{\bf Name:} &Rahul Chatterjee\\
{\bf Conference:}& 37th IEEE Symposium on Security and Privacy, San Jose, CA\\
{\bf Dates:} &May 23-25, 2016\\
{\bf Title of the Paper:}& \mbox{pASSWORD tYPOS and How to Correct Them Securely.}
\end{tabular}
\vspace{0.2in}
\begin{abstract}
  We provide the first treatment of typo-tolerant password
  authentication for arbitrary user-selected passwords. Such a system,
  rather than simply rejecting a login attempt with an incorrect
  password, tries to correct common typographical errors on behalf of
  the user. Limited forms of typo-tolerance have been used in some
  industry settings, but to date there has been no analysis of the
  utility and security of such schemes.

  In studies conducted on Amazon Mechanical Turk and via
  instrumentation of the production Dropbox login infrastructure, we
  quantify the kinds and rates of typos made by users. Our experiments
  reveal that almost 10\% of login attempts fail due to a handful of
  simple, easily correctable typos, such as capitalization errors.  We
  show that correcting just a few of these typos would reduce login
  delays for a significant fraction of users as well as enable an
  additional 3\% of users to achieve successful login.

  We introduce a framework for reasoning about typo-tolerance, and
  investigate the seemingly inherent tension here between security and
  usability of passwords. We use our framework to show that there
  exist typo-tolerant authentication schemes that can get corrections
  for ``free'': we prove they are as secure as schemes that always
  reject mistyped passwords. Building off this theory, we detail a
  variety of practical strategies for securely implementing
  typo-tolerance.
\end{abstract}
\vspace{0.2in}
\noindent{\bf Relevance of the conference to my research.} \\ I am
going to present my paper titled ``pASSWORD tYPOS and How to Correct
Them Securely'' at the 37th IEEE Symposium on Security and Privacy
(also known as IEEE S\&P or Oakland S\&P).  IEEE S\&P is one of the
oldest conferences in the field of security and privacy, and
critically acclaimed for its quality of accepted research
publications. I am happy and thrilled to present my work in front of
the greatest researchers in the field, and get valuable feedback from
them. Additionally, I shall be one of the researchers representing
Cornell showcasing Cornell's leadership in the field of security and
privacy.


As a first year graduate student, attending this conference will give
me ample opportunity to meet with the most productive and talented
minds in the field. This will enable me to build relationships with
members of the vibrant community, and broaden my scope of
collaboration with other researchers.  To know and get known in the
research community is an essential part of graduate studies, and
attending a first tier conferences like IEEE S\&P will greatly help me
to do so.
\end{document}


%%% Local Variables:
%%% mode: latex
%%% TeX-master: t
%%% End:
