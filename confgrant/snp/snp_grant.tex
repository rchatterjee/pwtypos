\documentclass[11pt,a4paper,notitlepage]{article}
\usepackage[margin=1in]{geometry}
\usepackage{algorithm,amssymb,amsmath,algpseudocode}
\usepackage{lmodern}
\usepackage{hyperref}
\hypersetup{unicode=true,
            pdfborder={0 0 0},
            breaklinks=true}
\urlstyle{same}  % don't use monospace font for urls
\providecommand{\tightlist}{%
  \setlength{\itemsep}{0pt}\setlength{\parskip}{0pt}}
\usepackage{color}
\newcommand{\bnm}{\begin{newmath}}
\newcommand{\enm}{\end{newmath}}

\newcommand{\bea}{\begin{eqnarray*}}
\newcommand{\eea}{\end{eqnarray*}}




\newcommand{\bne}{\begin{newequation}}
\newcommand{\ene}{\end{newequation}}


\newenvironment{newmath}{\begin{displaymath}%
\setlength{\abovedisplayskip}{4pt}%
\setlength{\belowdisplayskip}{4pt}%
\setlength{\abovedisplayshortskip}{6pt}%
\setlength{\belowdisplayshortskip}{6pt} }{\end{displaymath}}

\newenvironment{neweqnarrays}{\begin{eqnarray*}%
\setlength{\abovedisplayskip}{-4pt}%
\setlength{\belowdisplayskip}{-4pt}%
\setlength{\abovedisplayshortskip}{-4pt}%
\setlength{\belowdisplayshortskip}{-4pt}%
\setlength{\jot}{-0.4in} }{\end{eqnarray*}}

\newenvironment{newequation}{\begin{equation}%
\setlength{\abovedisplayskip}{4pt}%
\setlength{\belowdisplayskip}{4pt}%
\setlength{\abovedisplayshortskip}{6pt}%
\setlength{\belowdisplayshortskip}{6pt} }{\end{equation}}


\newcounter{ctr}
\newcounter{savectr}
\newcounter{ectr}

\newenvironment{newitemize}{%
\begin{list}{\mbox{}\hspace{5pt}$\bullet$\hfill}{\labelwidth=15pt%
\labelsep=5pt \leftmargin=20pt \topsep=3pt%
\setlength{\listparindent}{\saveparindent}%
\setlength{\parsep}{\saveparskip}%
\setlength{\itemsep}{3pt} }}{\end{list}}


\newenvironment{newenum}{%
\begin{list}{{\rm (\arabic{ctr})}\hfill}{\usecounter{ctr} \labelwidth=17pt%
\labelsep=5pt \leftmargin=22pt \topsep=3pt%
\setlength{\listparindent}{\saveparindent}%
\setlength{\parsep}{\saveparskip}%
\setlength{\itemsep}{2pt} }}{\end{list}}

\newlength{\saveparindent}
\setlength{\saveparindent}{\parindent}
\newlength{\saveparskip}
\setlength{\saveparskip}{\parskip}


\newcommand{\Adv}{\mathbf{Adv}}
\newcommand{\AdvMI}[1]{\Adv^\mathrm{mi}_{#1}}
\newcommand{\AdvHEDIST}[1]{\Adv^\mathrm{dist}_{#1}}
\newcommand{\AdvMR}[1]{\Adv^\mathrm{mr}_{#1}}
\newcommand{\AdvMRCCA}[1]{\Adv^\mathrm{mr\textnormal{-}cca}_{#1}}
\newcommand{\AdvKR}[1]{\Adv^\mathrm{kr}_{#1}}
\newcommand{\AdvSAMPRAT}[1]{\Adv^\mathrm{dte\textnormal{-}ratio}_{#1}}
\newcommand{\AdvSAMPIND}[1]{\Adv^\mathrm{dte}_{#1}}
%\newcommand{\AdvDTE}[1]{\Adv^\mathrm{dte}_{#1}}

\newcommand{\decOracle}{\textbf{Dec}}

\newcommand{\negsmidge}{{\hspace{-0.1ex}}}
\newcommand{\cdotsm}{\negsmidge\negsmidge\negsmidge\cdot\negsmidge\negsmidge\negsmidge}

\def\suchthatt{\: :\:}

\newcommand{\Prob}[1]{{\Pr\left[\,{#1}\,\right]}}
\newcommand{\probb}[2]{{\Pr}_{#1}\left[\,{#2}\,\right]}
\newcommand{\Probb}[2]{\Pr[#1]}
\newcommand{\CondProb}[2]{{\Pr}\left[\: #1\:\left|\right.\:#2\:\right]}
\newcommand{\CondProbb}[2]{\Pr[#1|#2]}
\newcommand{\ProbExp}[2]{{\Pr}\left[\: #1\:\suchthatt\:#2\:\right]}
\newcommand{\Ex}[1]{{\textnormal{E}\left[\,{#1}\,\right]}}
\newcommand{\Exx}{{\textnormal{E}}}

\newcommand{\true}{\textsf{true}}
\newcommand{\false}{\textsf{false}}



\newcommand{\secref}[1]{\mbox{Section~\ref{#1}}}
\newcommand{\apref}[1]{\mbox{Appendix~\ref{#1}}}
\newcommand{\thref}[1]{\mbox{Theorem~\ref{#1}}}
\newcommand{\defref}[1]{\mbox{Definition~\ref{#1}}}
\newcommand{\corref}[1]{\mbox{Corollary~\ref{#1}}}
\newcommand{\lemref}[1]{\mbox{Lemma~\ref{#1}}}
\newcommand{\clref}[1]{\mbox{Claim~\ref{#1}}}
\newcommand{\propref}[1]{\mbox{Proposition~\ref{#1}}}
\newcommand{\factref}[1]{\mbox{Fact~\ref{#1}}}
\newcommand{\remref}[1]{\mbox{Remark~\ref{#1}}}
\newcommand{\figref}[1]{\mbox{Figure~\ref{#1}}}
\renewcommand{\algref}[1]{\mbox{Algorithm~\ref{#1}}}
%\newcommand{\eqref}[1]{\mbox{Equation~(\ref{#1})}}
% Have to use \renewcommand because exists already in amsmath
\renewcommand{\eqref}[1]{\mbox{(\ref{#1})}}
\newcommand{\consref}[1]{\mbox{Construction~\ref{#1}}}
\newcommand{\tabref}[1]{\mbox{Table~\ref{#1}}}


\newcommand{\getsr}{{\:{\leftarrow{\hspace*{-3pt}\raisebox{.75pt}{$\scriptscriptstyle\$$}}}\:}}
\newcommand{\getm}{{\:\leftarrow_{\mdist}\:}}
\newcommand{\getpw}{{\:\leftarrow_{\pwdist}\:}}
\newcommand{\gettypo}{{\:\leftarrow_{\pw}\:}}
\newcommand{\getd}{{\:\leftarrow_{\ddist}\:}}
%\newcommand{\getm}{{\:{\leftarrow{\hspace*{-3pt}\raisebox{.75pt}{$\scriptscriptstyle \mdist$}}}\:}}
\newcommand{\getk}{{\:\leftarrow_{\kdist}\:}}
%\newcommand{\getk}{{\:{\leftarrow{\hspace*{-3pt}\raisebox{.75pt}{$\scriptscriptstyle \kdist$}}}\:}}
\newcommand{\getp}{{\:\leftarrow_{p}\:}}

\newcommand{\pwlength}{k}
\newcommand{\pwprob}{p}
\newcommand{\getpwp}{{\:\leftarrow_{\pwprob}\:}}
\newcommand{\typoprob}{\tau}
\newcommand{\mut}{f}
\newcommand{\mutset}{F}
\newcommand{\tcf}{g}
\newcommand{\tcfe}{G}
\newcommand{\accept}{{\sf true}}
\newcommand{\reject}{{\sf false}}
\newcommand{\nh}[1]{N(#1)}
\newcommand{\newnh}[1]{N^*_\mutset(#1)}

\newcommand{\utilinc}{\mu}
\newcommand{\secloss}{\Delta}
\newcommand{\seclossg}{\Delta^\textnormal{greedy}}
\newcommand{\seclosso}{\Delta}
%\newcommand{\maxlambda}{\lambda^*}
%\newcommand{\maxfuzzlambda}{\tilde{\lambda}^*}
\newcommand{\fuzzlambda}{\lambda^\textnormal{fuzzy}}
\newcommand{\greedylambda}{\lambda^\textnormal{greedy}}
\newcommand{\msgspace}{{\cal M}}
\newcommand{\dist}{{d}}
\newcommand{\ball}{B}
\newcommand{\hatball}{\hat{B}}
\newcommand{\newball}{B^*}
\newcommand{\ballserver}{\tilde{B}}
\newcommand{\edist}[2]{{\delta(#1,#2)}}
\newcommand\error{e}
\newcommand\codewordspace{{\cal C}}
\newcommand\codebook{{C}}

\newcommand\checkm{{\sf check}}
\newcommand\checkval{s}
\newcommand\enco{{\sf encode}}
\newcommand\setup{{\sf setup}}
\newcommand\mindist{\mu}


\newcommand{\Hdot}{H(\mbox{ } \cdot \mbox{ }  , \mbox{ } \del)}


\newcommand{\ACC}{\textnormal{ACC}}


\newcommand{\ATT}{\textnormal{GUESS}}
\newcommand{\win}{\textsf{win}}
\newcommand{\CheckOr}{\textnormal{Check}}

\newcommand{\gamesfontsize}{\small}
\newcommand{\fpage}[2]{\framebox{\begin{minipage}[t]{#1\textwidth}\setstretch{1.1}\gamesfontsize  #2 \end{minipage}}}

\newcommand{\hpages}[3]{\begin{tabular}{cc}\begin{minipage}[t]{#1\textwidth} #2 \end{minipage} & \begin{minipage}[t]{#1\textwidth} #3 \end{minipage}\end{tabular}}

\newcommand{\hpagess}[4]{
        \begin{tabular}{c@{\hspace*{.5em}}c}
        \begin{minipage}{#1\textwidth}\gamesfontsize #3 \end{minipage}
        &
        \begin{minipage}{#2\textwidth}\gamesfontsize #4 \end{minipage}
        \end{tabular}
    }


\newcommand{\hfpages}[3]{\hfpagess{#1}{#1}{#2}{#3}}
\newcommand{\hfpagess}[4]{
        \begin{tabular}{c@{\hspace*{.5em}}c}
        \framebox{\begin{minipage}[t]{#1\textwidth}\setstretch{1.1}\gamesfontsize #3 \end{minipage}}
        &
        \framebox{\begin{minipage}[t]{#2\textwidth}\setstretch{1.1}\gamesfontsize #4 \end{minipage}}
        \end{tabular}
    }
\newcommand{\hfpagesss}[6]{
        \begin{tabular}{c@{\hspace*{.5em}}c@{\hspace*{.5em}}c}
        \framebox{\begin{minipage}[t]{#1\textwidth}\setstretch{1.1}\gamesfontsize #4 \end{minipage}}
        &
        \framebox{\begin{minipage}[t]{#2\textwidth}\setstretch{1.1}\gamesfontsize #5 \end{minipage}}
        &
        \framebox{\begin{minipage}[t]{#3\textwidth}\setstretch{1.1}\gamesfontsize #6 \end{minipage}}
        \end{tabular}
    }
\newcommand{\hfpagessss}[8]{
        \begin{tabular}{c@{\hspace*{.5em}}c@{\hspace*{.5em}}c@{\hspace*{.5em}}c}
        \framebox{\begin{minipage}[t]{#1\textwidth}\setstretch{1.1}\gamesfontsize #5 \end{minipage}}
        &
        \framebox{\begin{minipage}[t]{#2\textwidth}\setstretch{1.1}\gamesfontsize #6 \end{minipage}}
        &
        \framebox{\begin{minipage}[t]{#3\textwidth}\setstretch{1.1}\gamesfontsize #7 \end{minipage}}
        &
        \framebox{\begin{minipage}[t]{#4\textwidth}\setstretch{1.1}\gamesfontsize #8 \end{minipage}}
        \end{tabular}
    }

\newcommand{\hpagessss}[8]{
        \begin{tabular}{c@{\hspace*{.5em}}c@{\hspace*{.5em}}c@{\hspace*{.5em}}c}
        \begin{minipage}[t]{#1\textwidth}\gamesfontsize #5 \end{minipage}
        &
        \begin{minipage}[t]{#2\textwidth}\gamesfontsize #6 \end{minipage}
        &
        \begin{minipage}[t]{#3\textwidth}\gamesfontsize #7 \end{minipage}
        &
        \begin{minipage}[t]{#4\textwidth}\gamesfontsize #8 \end{minipage}
        \end{tabular}
    }
\newcommand{\hpagesss}[6]{
        \begin{tabular}{c@{\hspace*{.5em}}c@{\hspace*{.5em}}c@{\hspace*{.5em}}c}
        \begin{minipage}[t]{#1\textwidth}\gamesfontsize #4 \end{minipage}
        &
        \begin{minipage}[t]{#2\textwidth}\gamesfontsize #5 \end{minipage}
        &
        \begin{minipage}[t]{#3\textwidth}\gamesfontsize #6 \end{minipage}
        \end{tabular}
    }


\newcommand{\vecw}{\mathbf{w}}
\newcommand{\R}{\mathbb{R}}
\newcommand{\N}{\mathbb{N}}
\newcommand{\Z}{\mathbb{Z}}
\newcommand{\load}{L}
\newcommand{\coll}{\mathsf{Coll}}
\newcommand{\nocoll}{\overline{\mathsf{Coll}}}


\newcommand{\Img}{\textsf{Img}}

\def \mspace {{\cal{M}}}
\def \mspacebot {{\cal{M}_\bot}}
\def \sspace {{\cal{S}}}
\def \slen {{s}}
\def \kspace {{\cal{K}}}
\def \kspacesize {{m}}
\def \mspacesize {{n}}
\def \kdict {D}
\def \dictsize {d}
\newcommand{\kdist}{p_k}
\newcommand{\mdist}{p_m}
\newcommand{\alldist}{\rho}
% \newcommand{\pwdist}{\rho}
\newcommand{\ddist}{\rho_{dec}}
\newcommand{\PWset}{{\cal P}}
\newcommand{\PWsetvec}{\vec{\cal P}}
\newcommand{\PWvec}{\vec{P}}
\newcommand{\domvec}{\vec{D}}
\newcommand{\humanornot}{\vec{h}}
\newcommand{\dom}{\textsf{dom}}
%\def \kdist {{\kappa}}
%\def \mdist {{\mu}}
%\def \ddist {{\delta}}
\def \pspace {{\cal{P}}}
\def \mpspace {{\cal{MP}}}
\def \cspace {{\cal{C}}}
\def \key {K}
\def \msg {M}
\def \seed {S}
\def \ctxt {C}
\def \ctxtpart {C_2}
\newcommand{\genprime}{{\textsf{GenPrime}}}
\newcommand{\isprime}{{\textsf{IsPrime}}}
\newcommand{\LeastLesserPrime}{{\textsf{PrevPrime}}}

\newcommand{\wallet}{w}



\newcommand{\DTE}{{\textsf{DTE}}\xspace}
\newcommand{\encode}{{\textsf{encode}}\xspace}
\newcommand{\decode}{{\textsf{decode}}\xspace}

\newcommand{\DTEsingle}{{\textsf{1PW}}\xspace}
\newcommand{\encodesingle}{{\textsf{1PW-encode}}\xspace}
\newcommand{\decodesingle}{{\textsf{1PW-decode}}\xspace}

\newcommand{\DTEuniform}{{\textsf{UNIF}}\xspace}
\newcommand{\encodeuniform}{{\textsf{U-encode}}\xspace}
\newcommand{\decodeuniform}{{\textsf{U-decode}}}

\newcommand{\DTEng}{{\textsf{NG}}\xspace}
\newcommand{\encodeng}{{\textsf{NG-encode}}\xspace}
\newcommand{\decodeng}{{\textsf{NG-decode}}\xspace}

\newcommand{\DTEpcfg}{{\textsf{PCFG}}\xspace}
\newcommand{\encodepcfg}{{\textsf{PCFG-encode}}\xspace}
\newcommand{\decodepcfg}{{\textsf{PCFG-decode}}\xspace}

\newcommand{\DTErss}{{\textsf{RSS}}\xspace}
\newcommand{\encoderss}{{\textsf{RSS-encode}}\xspace}
\newcommand{\decoderss}{{\textsf{RSS-decode}}\xspace}

\newcommand{\DTEindep}{{\textsf{MPW}}\xspace}
\newcommand{\encodeindep}{{\textsf{MPW-encode}}\xspace}
\newcommand{\decodeindep}{{\textsf{MPW-decode}}\xspace}

\newcommand{\DTEindepng}{{\textsf{MPW-NG}}\xspace}
\newcommand{\encodeindepng}{{\textsf{MPW-NG-encode}}\xspace}
\newcommand{\decodeindepng}{{\textsf{MPW-NG-decode}}\xspace}

\newcommand{\DTEindeppcfg}{{\textsf{MPW-PCFG}}\xspace}
\newcommand{\encodeindeppcfg}{{\textsf{MPW-PCFG-encode}}\xspace}
\newcommand{\decodeindeppcfg}{{\textsf{MPW-PCFG-decode}}\xspace}

\newcommand{\DTEsub}{{\textsf{SG}}}
\newcommand{\encodesub}{{\textsf{SG-encode}}}
\newcommand{\decodesub}{{\textsf{SG-decode}}}

\newcommand{\dkamf}{{\textsf{KAMF}}}
\newcommand{\dkamfplus}{{\textsf{KAMFP}}}


\newcommand{\decodekamf}{{\textsf{KAMF-decode}}}
\newcommand{\decodekamfplus}{{\textsf{\KAMFP-decode}}}





\newcommand{\DTEis}{{\textsf{IS-DTE}}}
\newcommand{\encodeis}{{\textsf{is-encode}}}
\newcommand{\decodeis}{{\textsf{is-decode}}}
\newcommand{\DTErej}{{\textsf{REJ-DTE}}}
\newcommand{\encoderej}{{\textsf{rej-encode}}}
\newcommand{\decoderej}{{\textsf{rej-decode}}}


\newcommand{\DTErsarej}{{\textsf{RSA-REJ-DTE}}}
\newcommand{\encodeRSAREJ}{{\textsf{rsa-rej-encode}}}
\newcommand{\decodeRSAREJ}{{\textsf{rsa-rej-decode}}}
\newcommand{\DTErsainc}{{\textsf{RSA-INC-DTE}}}
\newcommand{\encodeRSAINC}{{\textsf{rsa-inc-encode}}}
\newcommand{\decodeRSAINC}{{\textsf{rsa-inc-decode}}}
\newcommand{\DTEunf}{{\textsf{UNF-DTE}}}
\newcommand{\DTEnunf}{{\textsf{NUNF-DTE}}}


%\newcommand{\encodeis}{{\textsf{encode}_{\textrm{is}}}}
%\newcommand{\decodeis}{{\textsf{decode}_{\textrm{is}}}}
\newcommand{\rep}{\textsf{rep}}
\newcommand{\isErr}{\epsilon_{\textnormal{is}}}
\newcommand{\incErr}{\epsilon_{\textnormal{inc}}}
\def \enc {{\textsf{enc}}}
\def \dec {{\textsf{dec}}}
\def \SEscheme {{\textsf{SE}}}
\def \HEscheme {{\textsf{HE}}}
\def \CTR {{\textsf{CTR}}}
\def \encHE {{\textsf{HEnc}}}
\def \HIDE {{\textsf{HiaL}}}
\def \encHIDE {{\textsf{HEnc}}}
\def \decHIDE {{\textsf{HDec}}}
\def \decHE {{\textsf{HDec}}}
\def \encHEt {{\textsf{HEnc2}}}
\def \decHEt {{\textsf{HDec2}}}

\newcommand{\myind}{\hspace*{1em}}
\newcommand{\thh}{^{\textit{th}}} % th
\newcommand{\concat}{\,\|\,}
\newcommand{\dotdot}{..}
\newcommand{\emptystr}{\varepsilon}


\newcommand{\round}{\textsf{round}}

\newcommand{\alphabar}{\overline{\alpha}}
\newcommand{\numbinsbar}{\overline{b}}
\newcommand{\numballs}{a}
\newcommand{\numbins}{b}

%\def \encHE {{\sf{enc}^{HE}}}
%\def \decHE {{\sf{dec}^{HE}}}
%\def \encHEt {{\sf{enc}^{HE2}}}
%\def \decHEt {{\sf{dec}^{HE2}}}
\def \idealHE {{\mathcal{HE}}}
\def \IEnc {{\mathbf{\rho}}}
\def \IDec {{\mathbf{\rho^{-1}}}}
\def \OEnc {{\mathbf{Enc}}}
\def \ODec {{\mathbf{Dec}}}
\newcommand{\SimuProc}{\mathbf{Sim}}
\newcommand{\ROProc}{\mathbf{RO}}
\newcommand{\PrimProc}{\mathbf{Prim}}
\def \stm {g}
\def \istm {\hat{g}}
\def \kts {{f}}
\def \lex {{\sf lex}}
\def \part {part}
\def \kd {{\sf{kd}}}
\def \msgdist {{d}}
\def \keydist {{r}}
\def \ind {{\sf{index}}}
\def \kprf {z}
\def \adv {{\cal A}}
\def \pwds {u}
\newcommand{\mpw}{mpw} 
\newcommand{\pw}{{w}} 
\newcommand{\pwtypo}{\tilde{w}} 
\newcommand{\PW}{\mathcal{PW}}
\newcommand{\strings}{\mathcal{S}}
\newcommand{\pwvec}{\vec{\pw}} 
\newcommand{\vecx}{\vec{x}} 
\def \tokens {v}
\def \calP{{\cal P}}
\def \template{{\cal T}}
\def \vaultset{{\cal V}}
\def \ext {{\sf ext}}
\def \offset {\delta}
\def \maxweight {\epsilon}
\def \advo {{\cal A}^{*}}
\newcommand{\suite}{S}

\newcommand{\Chall}{\textsf{Ch}}
\newcommand{\MI}{\textnormal{MI}}
\newcommand{\MR}{\textnormal{MR}}
\newcommand{\MRCCA}{\textnormal{MR-CCA}}
\newcommand{\SAMP}{\textnormal{SAMP}}
\newcommand{\DTEgame}{\textnormal{SAMP}}
\newcommand{\KR}{\textnormal{KR}}
\newcommand{\advA}{{\cal A}}
\newcommand{\advB}{{\cal B}}
\newcommand{\advI}{{\cal I}}
\newcommand{\semi}{\;;\;}
\newcommand{\TabC}{\texttt{C}}
\newcommand{\TabR}{\texttt{R}}
\newcommand{\Hash}{H}
\newcommand{\Cipher}{\pi}
\newcommand{\CipherInv}{\pi^{-1}}
\newcommand{\simu}{{\mathcal S}}
\newcommand{\prim}{P}
\newcommand{\maxguess}{\gamma}


\newcommand{\bigO}{\mathcal{O}}
\newcommand{\calG}{{\mathcal G}}

\def\sqed{{\hspace{5pt}\rule[-1pt]{3pt}{9pt}}}
\def\qedsym{\hspace{2pt}\rule[-1pt]{3pt}{9pt}}

\newcommand{\Colon}{{\::\;\;}}
\newcommand{\good}{\textsf{Good}}

\newcommand\Tvsp{\rule{0pt}{2.6ex}}
\newcommand\Bvsp{\rule[-1.2ex]{0pt}{0pt}}
\newcommand{\TabPad}{\hspace*{5pt}}
\newcommand\TabSep{@{\hspace{5pt}}|@{\hspace{5pt}}}
\newcommand\TabSepLeft{|@{\hspace{5pt}}}
\newcommand\TabSepRight{@{\hspace{5pt}}|}


\DeclareMathOperator*{\argmin}{argmin}
\DeclareMathOperator*{\argmax}{argmax}
\newcommand{\comma}{\textnormal{,}}

\renewcommand{\paragraph}[1]{\vspace*{6pt}\noindent\textbf{#1}\;}

\newcommand{\weirvault}{\textsf{Pastebin}\xspace}
\newcommand{\ndssvault}{\textsf{DBCBW}\xspace}

\newcommand{\chbudget}{c}
\newcommand{\register}{\textnormal{\textsf{Reg}}\xspace}
\newcommand{\checker}{\textnormal{\textsf{Chk}}\xspace}
\newcommand{\checkerall}{\textnormal{\textsf{Chk-All}}\xspace}
\newcommand{\checkerbl}{\textnormal{\textsf{Chk-wBL}}\xspace}
\newcommand{\checkerapprox}{\textnormal{\textsf{Chk-AOp}}\xspace}
\newcommand{\opchecker}{\textnormal{\textsf{OpChk}}\xspace}

\newcommand{\exregister}{\textnormal{\textsf{ExReg}}\xspace}
\newcommand{\exchecker}{\textnormal{\textsf{ExChk}}\xspace}

\newcommand{\utility}{\textnormal{\textsf{Util}}\xspace}
\newcommand{\advantage}{\textnormal{\textsf{Adv}}\xspace}


%%%%%%%%%%%%%%%%%%%%%%%%%%%%%%%%%%%%%%%%%%%%%%%%%%%%%%%%%%%%%%%%%%%%%%%%%%%%%%
%
% Figure and table macros
%

\newcounter{mytable}

\def\mytable{\begin{centering}\refstepcounter{mytable}}
\def\endmytable{\end{centering}}

\def\mytablecaption#1{\vspace{2mm}
                      \centerline{Table \arabic{mytable}.~{#1}}
                      \vspace{6mm}
             \addcontentsline{lot}{table}{\protect\numberline{\arabic{mytable}}~{#1}}}


\newcounter{myfig}
\def\myfig{\begin{centering}\refstepcounter{myfig}}
\def\endmyfig{\end{centering}}

\def\myfigcaption#1{
             \vspace{2mm}
             \centerline{\textsf{Figure \arabic{myfig}.~{#1}}}
             \vspace{6mm}
             \addcontentsline{lof}{figure}{\protect\numberline{\arabic{myfig}}~{#1}}}


%%%%%%%%%%%%%%%%%%%%%%%%%%%%%%%%%%%%%%%%%%%%%%%%%%%%%%%%%%%%%%%%%%%%%%%%%%%%%%
%
% New commands:
%
\newcommand{\reminder}[1]{ [[[ \marginpar{\mbox{$<==$}} #1 ]]] }

%
% New theorem types:
%
\newtheorem{observation}{Observation}
\newtheorem{definition}{Definition}
\newtheorem{claim}{Claim}
\newtheorem{assumption}{Assumption}
\newtheorem{fact}{Fact}
\newtheorem{theorem}{Theorem}
\newtheorem{lemma}{Lemma}
\newtheorem{corollary}{Corollary}
\newtheorem{proposition}{Proposition}
\newtheorem{example}{Example}

%
% Definitions:
%
\def \blackslug{\hbox{\hskip 1pt \vrule width 4pt height 8pt
    depth 1.5pt \hskip 1pt}}
\def \qed{\quad\blackslug\lower 8.5pt\null\par}
% In-line QED, for ending a proof with a $$ formula
% In-line QED, for ending a proof with a $$ formula
\def \inQED{\quad\quad\blackslug}
\def \Qed{\QED}
\def \QUAD{$\Box$}
\def \Proof{\par\noindent{\bf Proof:~}}
\def \proof{\Proof}
\def \poly {\mbox{$\mathsf{poly}$}}
\def \binary {\mbox{$\mathsf{binary}$}}
\def \ones {\mbox{$\mathsf{ones}$}}
\def \rank {\mbox{$\mathsf{rank}$}}
\def \bits {\mbox{$\mathsf{bits}$}}
\def \factorial {\mbox{$\mathsf{factorial}$}}
\def \fr {\mbox{$\mathsf{fr}$}}
\def \pr {\mbox{$\mathsf{pr}$}}
\def \zon {\{0,1\}^n}
\def \zo  {\{0,1\}}
\def \zok {\{0,1\}^k}
\def \mo {s}

%\newcommand{\note}[2]{\textcolor{red}{Note from #1: #2}}
%\newcommand{\noteari}[1]{\note{Ari}{#1}}

\newcounter{mynote}[section]
\newcommand{\notecolor}{blue}
\newcommand{\thenote}{\thesection.\arabic{mynote}}
\newcommand{\tnote}[1]{\refstepcounter{mynote}{\bf \textcolor{cyan}{$\ll$TomR~\thenote: {\sf #1}$\gg$}}}
\newcommand{\anote}[1]{\refstepcounter{mynote}{\bf \textcolor{red}{$\ll$Ari~\thenote: {\sf #1}$\gg$}}}
\newcommand{\devd}[1]{\refstepcounter{mynote}{\bf \textcolor{red}{$\ll$Dev~\thenote: {\sf #1}$\gg$}}}

\newcommand{\fixme}[1]{{\textcolor{red}{[FIXME: #1]}}}


\newcommand\ignore[1]{}


\newcommand\simplescheme{simple}


\newcommand{\KDF}{\mathsf{KDF}}
\newcommand{\salt}{\mathsf{sa}}
\newcommand{\PRF}{F}
\newcommand{\subgram}{\mathsf{SG}}
\newcommand{\popdomains}{\mathcal{D}}

\newcommand{\retrieve}{\textsf{Sync}}
\newcommand{\update}{\textsf{Insert}}

\newcommand{\dictW}{\textbf{D1}\xspace}
\newcommand{\dictF}{\textbf{D2}\xspace}

\newcommand{\str}{\text{str}}
\newcommand{\calS}{{\mathcal S}}


% \newcommand{\name}{{\textsf{SweetPass}}\xspace}
% \newcommand{\name}{{\textsf{NoCrack}}\xspace}


\newcommand{\edistname}{{edit distance}\xspace}
%% For name of typos 
\newcommand{\same}{\textsf{same}\xspace}
\newcommand{\swcall}{\textsf{swc-all}\xspace}
\newcommand{\swcfirst}{\textsf{swc-first}\xspace}
\newcommand{\rmlast}{\textsf{rm-last}\xspace}
\newcommand{\rmlastl}{\textsf{rm-lastl}\xspace}
\newcommand{\rmlastd}{\textsf{rm-lastd}\xspace}
\newcommand{\rmlasts}{\textsf{rm-lasts}\xspace}
\newcommand{\rmfirst}{\textsf{rm-first}\xspace}
\newcommand{\dtoslast}{\textsf{n2s-last}\xspace}
\newcommand{\stodlast}{\textsf{s2n-last}\xspace}
%\newcommand{\swslast}{\textsf{sws-last}\xspace}
\newcommand{\keyprox}{\textsf{near-key}\xspace}
\newcommand{\other}{\textsf{other}\xspace}
\newcommand{\upncap}{\textsf{upncap}\xspace}

\newcommand{\typoset}{{\mathcal C}}
\newcommand{\topfive}{{\mathcal C}_{\textrm{top5}}}
\newcommand{\toptwo}{{\mathcal C}_{\textrm{top2}}}
\newcommand{\topthree}{{\mathcal C}_{\textrm{top3}}}
\newcommand{\otherfail}{\textrm{other}}

%% ------------------------- Rahul -----------------------
\newcommand{\rcnote}[1]{\refstepcounter{mynote}{\bf \textcolor{magenta}{$\ll$RC~\thenote: {\sf #1}$\gg$}}}

%\newcommand{\shepherd}[1]{\textcolor{blue}{ #1}}
\newcommand{\shepherd}[1]{{ #1}}

\newcommand{\NT}[1]{\texttt{#1}}
\DeclareMathSymbol{\mlq}{\mathord}{operators}{``}
\DeclareMathSymbol{\mrq}{\mathord}{operators}{`'}
\newcommand{\calO}{{\mathcal O}}
\newcommand{\calA}{{\mathcal A}}
\newcommand{\kamfplus}{Kamouflage\textbf{+}\xspace}
\newcommand{\starts}{\texttt{\string^}}
\newcommand{\symend}{\texttt{\$}}
\newcommand{\ngram}{$n$-gram}
\newcommand{\dte}[1]{\texttt{\bf #1}}
\newcommand{\RY}[1]{RY-{#1}}


\newcommand{\pfive}{\textsf{Top5}}
\newcommand{\pthree}{\textsf{Top3}}
\newcommand{\pfiveb}{\textsf{Top5-Unpop}}
\newcommand{\pthreeb}{\textsf{Top3-Unpop}}
\newcommand{\pfivebone}{\textsf{Top5-pop1}}
\newcommand{\pthreebone}{\textsf{Top3-pop1}}
\newcommand{\todo}[1]{{\textcolor{red}{[TODO: #1]}}}

\newcommand\Tstrut{\rule{0pt}{2.6ex}}         % = \tau{op' strut
\newcommand\Bstrut{\rule[-0.9ex]{0pt}{0pt}}   % = \beta{ottom' strut}}

\def\epwprob{\tilde{\pwprob}}
\def\etypoprob{\tilde{\typoprob}}


%%% Local Variables:
%%% mode: latex
%%% TeX-master: "main"
%%% End:



\title{pASSWORD typos and\\ How To Securely Correct Them}
\author{Rahul Chatterjee}
\begin{document}
%\maketitle
\section*{Application for Conference Grant}
%\subsection*{Rahul Chatterjee}
\vspace{0.2in}

% \begin{abstract}\em
%   We provide the first treatment of typo-tolerant password
%   authentication for arbitrary user-selected passwords. Such a system,
%   rather than simply rejecting a login attempt with an incorrect
%   password, tries to correct common typographical errors on behalf of
%   the user. Limited forms of typo-tolerance have been used in some
%   industry settings, but to date there has been no analysis of the
%   utility and security of such schemes.

%   In studies conducted on Amazon Mechanical Turk and via
%   instrumentation of the production Dropbox login infrastructure, we
%   quantify the kinds and rates of typos made by users. Our experiments
%   reveal that almost 10\% of login attempts fail due to a handful of
%   simple, easily correctable typos, such as capitalization errors.  We
%   show that correcting just a few of these typos would reduce login
%   delays for a significant fraction of users as well as enable an
%   additional 3\% of users to achieve successful login.

%   We introduce a framework for reasoning about typo-tolerance, and
%   investigate the seemingly inherent tension here between security and
%   usability of passwords. We use our framework to show that there
%   exist typo-tolerant authentication schemes that can get corrections
%   for ``free'': we prove they are as secure as schemes that always
%   reject mistyped passwords. Building off this theory, we detail a
%   variety of practical strategies for securely implementing
%   typo-tolerance.
% \end{abstract}

\section{Student's Statement} 
% \texttt{ Note: An application for a travel award will consist of a
%   single PDF file with the student's resume, a statement from the
%   student indicating if they would like to attend the symposium and/or
%   the workshops (and which workshop), and a letter from the student's
%   research advisor with a justification of financial need. The
%   student's statement should also include a summary of research
%   interests, and a brief discussion of why the applicant will benefit
%   from participating in the symposium.}\\

I would like to attend the 37th IEEE \underline{Symposium} on Security
and Privacy from May 23 to May 25, 2016 at The Fairmont, San Jose, CA.

\paragraph{Research Interest.}
My primary research interest targets making password-based
authentication and encryption schemes simple, safe and easy to use.
Passwords are ubiquitous in today's web authentication. Passwords need
no specialized hardware, such as a card or finger-print reader, to
authenticate. Moreover, if a user forgets their password, it is much
easier to reset them compared to other means of authentication system
that are based on smart-cards or bio-metrics.  However, users tend to
choose easily guessable passwords, and reuse them across multiple
websites. Moreover, users complain that complex passwords are hard to
remember and difficult to enter. My research targets addressing these
concerns.

% Given that nowadays users have many online accounts, remembering
% distinct complex passwords for each of them poses severe memory
% burden on users.

The biggest problem with password based encryption schemes is that,
attackers can mount {\em offline brute-force decryption} attacks,
should they get hold of a ciphertext. The attack works by making
repeated attempts to decrypt the ciphertext with different guesses for
the password without any need to communicate with any server that is
the legitimate owner of the ciphertext. Thanks to advances in modern
password cracking tools, attacker typically guesses the correct
password within a manageable number of queries. Additionally, every
wrong guess of the password results in decryption failure, making it
clear for the attacker to figure out whether the guess is right or
wrong. I am building new password based encryption schemes that resist
offline brute-force attacks by producing plausible looking decoy
plaintexts for each of the guessed password.  This makes it harder for
the attacker to learn whether or not the decryption was successful.  A
typical use-case of password based encryption scheme is to protect
credential stores, such as password managers, where passwords of
different programs or websites are stored, and the store is encrypted
under a master-password chosen by the user.  I have designed a new
generation credential store that aids offline cracking resistance, and
now I am exploring options to apply that technique to other places
where password based encryption is used.


To prevent passwords from being easily guessable, many security
experts advise users to choose {\em long and complex} passwords, which
have to be reproduced exactly every time they wish to log in. Long and
complex passwords are hard to remember as well as are time consuming
and error prone to type. This is worse if the user is using a
touchscreen device to log in. All this, in turn, discourages users
from choosing strong passwords.  Allowing small typographical errors
in passwords might alleviate these problems, but it is not clear,
upfront, how to perform typo-tolerance, and what is the effect on
security of this.  I am doing studies to quantify the usability and
security impact of tolerating small errors while entering
passwords. Our initial study reveals that a significant fraction of
users failed to log in or faced a delay in login due to a handful of
easily correctable typographical errors, while, if carefully
constructed, typo tolerance would not degrade security by more than a
negligible amount.  % The biggest challenge in analyzing typos in
% passwords is to collect real world typo data.
I am also looking into allowing typo tolerance in different contexts
where human-chosen passwords are used, such as password authenticated
key exchange, and perform more thorough analysis of security.
\\

% Why want to join the conference?
\paragraph{Reasons to join the symposium.}
IEEE S\&P is one of the best conferences in the field of security and
privacy. Above all, I am happy and excited to be able to present my
work in front of the experts in the field, and get valuable feedback
from them.  Attending this conference as a first year graduate
student, I will have ample networking opportunities. This will enable
me to build strong relationships with the members of the vibrant
community and broaden my scope to collaborate with other
researchers. To know and get known in the research community is
important for succeeding in graduate studies. Attending first tier
conferences, such as IEEE S\&P, will greatly help me to do so.


\section{Advisor's Note}
Rahul is going to present his work titled ``pASSWORD tYPOS and How to
Correct Them Securely'' at the 37th IEEE Symposium on Security and
Privacy. This is a very interesting work showing the benefits of
fixing typographical mistakes in passwords. Also, this work proposes
ways to securely correct typos, that has minimal security degradation
with respect to traditional password checkers that do not allow typos
in password submissions.

Attending this conference will be greatly beneficial for Rahul's
graduate studies as well as help him building research relationships
with the members of the community.  Moreover, obtaining the conference
grant will boost the confidence of a new researcher like Rahul, and
encourage him to continue working on challenging problems in security
and privacy that has larger impact on the society.

% \documentclass[]{article}
% \usepackage{lmodern}
% \usepackage{amssymb,amsmath}
% \usepackage{ifxetex,ifluatex}
% \usepackage{fixltx2e} % provides \textsubscript
% \ifnum 0\ifxetex 1\fi\ifluatex 1\fi=0 % if pdftex
%   \usepackage[T1]{fontenc}
%   \usepackage[utf8]{inputenc}
% \else % if luatex or xelatex
%   \ifxetex
%     \usepackage{mathspec}
%   \else
%     \usepackage{fontspec}
%   \fi
%   \defaultfontfeatures{Ligatures=TeX,Scale=MatchLowercase}
% \fi
% % use upquote if available, for straight quotes in verbatim environments
% \IfFileExists{upquote.sty}{\usepackage{upquote}}{}
% % use microtype if available
% \IfFileExists{microtype.sty}{%
% \usepackage{microtype}
% \UseMicrotypeSet[protrusion]{basicmath} % disable protrusion for tt fonts
% }{}
% \usepackage{hyperref}
% \hypersetup{unicode=true,
%             pdfborder={0 0 0},
%             breaklinks=true}
% \urlstyle{same}  % don't use monospace font for urls
% \IfFileExists{parskip.sty}{%
% \usepackage{parskip}
% }{% else
% \setlength{\parindent}{0pt}
% \setlength{\parskip}{6pt plus 2pt minus 1pt}
% }
% \setlength{\emergencystretch}{3em}  % prevent overfull lines
% \providecommand{\tightlist}{%
%   \setlength{\itemsep}{0pt}\setlength{\parskip}{0pt}}
% \setcounter{secnumdepth}{0}
% % Redefines (sub)paragraphs to behave more like sections
% \ifx\paragraph\undefined\else
% \let\oldparagraph\paragraph
% \renewcommand{\paragraph}[1]{\oldparagraph{#1}\mbox{}}
% \fi
% \ifx\subparagraph\undefined\else
% \let\oldsubparagraph\subparagraph
% \renewcommand{\subparagraph}[1]{\oldsubparagraph{#1}\mbox{}}
% \fi

% \date{}

% \begin{document}

\section{Resume}
\section*{Rahul Chatterjee}\label{rahul-chatterjee}
% {\bf\large Name:} Rahul Chatterjee 
\subsection*{Contact Information:}\label{contact-information}
\begin{itemize}
\tightlist
\item
  Mobile: +1-(414)-877-3378,
\item
  E-mail: rahul@cs.cornell.edu,
\item
  Web: \url{https://www.cs.cornell.edu/~rahul/}
\item
  Address: Cornell Tech, 111 8th Ave \#1202, New York, NY 10011
\end{itemize}

\subsection*{Research Interests}\label{research-interests}
Computer Security

\subsection*{Education}\label{education}

\begin{itemize}
\tightlist
\item
  Cornell University, PhD, Computer Science
\item
  University of Wisconsin---Madison, Masters, Computer Sciences
\item IIT Kharagpur, BTech, Computer
  Science and Engineering
\end{itemize}

\subsection*{Industry Experience}\label{work-experience}
\begin{itemize}
\item
  \textbf{Microsoft Research Technologies}, (Redmond, USA) (June 2015 - August 2015)\\
  Project: Analyzing Crypto Board Data to Infer Common Security
  Engineering Problems.
\item
  \textbf{Two Roads Technology Solutions} (Bangalore, India) (June 2012 - May 2013)\\
  Job Title: Software Developer and Quantitative Analyst
\item
  \textbf{Adobe Systems India Pvt. Ltd.} (Noida, India) (May 2011 - July 2011)\\
  Project: Generating Smart Tags for Images from Meta-data and Web
\end{itemize}

\renewcommand\refname{{\large Publications}}
\begin{thebibliography}{0}
\bibitem{pwtypo16}
  Rahul Chatterjee, Anish Athalye, Devdatta Akhawe, Ari Juels, Thomas
  Ristenpart, \textit{pASSWORD tYPOS and How to Correct Them Securely}, Submitted
  to 37th IEEE Symposium on Security and Privacy 2016 (Oakland).
\bibitem{pythia15}
  Adam Everspaugh, Rahul Chatterjee, Samuel Scott, Air Juels, Thomas
  Ristenpart, \textit{The Pythia PRF Service}, USENIX Security 2015.
\bibitem{nocrack15}
  Rahul Chatterjee, Joseph Bonneau, Ari Juels, Thomas Ristenpart,
  \textit{Cracking-Resistant Password Vaults using Natural Language Encoders},
  36th IEEE Symposium on Security and Privacy 2015 (Oakland).
\end{thebibliography}

% \subsection*{Notable Projects:}\label{notable-projects}
% \begin{itemize}
% \item
%   \textbf{pASSWORD tYPOS}\\
%   \begin{itemize}
%   \item Goal: Logging in using passwords often gets rejected due to
%     small typographical errors in the entered password, such as typing
%     `pASSWORD' instead of `Password'.
%   \item
%     I analyzed to what extent typos in password effect the usability using
%     Amazon Mechanical Turk and instrumenting Dropbox login server.
%   \item
%     I also analyzed the security degradation (if any) in allowing small
%     typos in passwords. Theoretically I showed that we can allow typos in
%     passwords during logging in without degradation of security provided
%     the distribution of passwords is available.

%   \end{itemize}
% \item
%   \textbf{Cracking Resistant Password Vault}\\
%   Goal: Encryption scheme for password vaults s.t. decryption with wrong
%   key should generate plausible looking but decoy vaults, forcing the
%   attacker to go for online verification.
% \item
%   First ever real world application of Honey Encryption. Used existing
%   password cracking methodologies in defense of password vaults to
%   create effective distribution transforming en- coder.
% \item
%   Showed effective attacks against state of the art approaches, e.g.,
%   Kamouflage by Bojinov et al.
% \item
%   \textbf{The Pythia PRF Services}\\
%   Goal: Create a remote PRF service that can be used for password
%   hardening with some added features, such as, the PRF only has access
%   to the blinded version of the password. One can rotate the secret key
%   of the PRF and update the stored hash values of the passwords without
%   access to the plaintext password.
% \item
%   Along with password generation, we also found lots of interesting
%   application like forward secure email system, secure effacing of
%   encrypted hard drive etc.
% \item
%   \textbf{Simultaneous Localization and Mapping using Relational
%   Trees}\\
% \item
%   Project Sponsored by, European Aeronautic Defence and Space Company
%   (EADS), Germany.
% \item
%   Developed an unsupervised method of learning with which a robot, which
%   is flying over an unknown region and taking snaps periodically, can
%   build a semantic map of the region and localize its position on the
%   map using those pictures.
% \item
%   Implemented the software using C, C++, OpenCV and built a front end
%   using Java.
% \item
%   \textbf{Information Retrieval and Natural Language Processing.}\\
%   Goal: Tried to correct and improve fluency of Machine generated text
%   or second language writer's texts with minimum language dependent
%   information.
% \item
%   This project was selected one of the best four BTech projects in the
%   department.
% \item
%   \textbf{Creating Artistic Effects on Image using Random Digital
%   Curve}\\
% \item
%   Developed algorithm for generating irreducible simple random digital
%   curves in a constrained domain. Drawing multiple of them on the edge
%   of binary image and then setting the intensity of each pixel
%   proportional to the number of times it was visited by the curves give
%   nice real pencil sketch effect.
% \item
%   This work is published in CAIP-2011, Seville, Spain.
% \end{itemize}

\subsection*{Academic Achievements Awards and
Scholarships}\label{academic-achievements-awards-and-scholarships}

\begin{itemize}
\tightlist
\item
  Awarded special CS fellowship from the department of Computer
  Sciences, UW--Madison.
\item Secured rank 4th in the North Central region
  and 2nd in the University in ACM-ICPC regional contest 2013.
\item
  Awarded IMPRS-CS fellowship for Master Studies in Germany by
  MPI-Informatics, Saar- brucken, Germany, 2012.
\item
  Jagadish Bose National Science Talent Search(JBNSTS) scholarship,
  2008.
% \item
%   Selected for admission to B.Math in Chennai Mathematical
%   Institute(CMI), Chennai, 2008. • Ranked 12 in West Bengal Joint
%   Entrance Examination(WBJEE), 2008 (out of 50 thousand students)
\end{itemize}

\subsection*{Extra Curricular
Activities}\label{extra-curricular-activities}

\begin{itemize}
\tightlist
\item
  Selected for in Indian National Mathematics Olympiad(INMO) exam in
  2007.
\item
  Executive Committee member of Bengali Association of Madison (BAM), WI
  since April, 2014. BAM organizes lots of Indian cultural events in
  Madison.
\item
  Captain of Hall Mathematics Olympiad team and won Silver, Gold in
  Inter Hall Mathematics Olympiad in 2011 and 2012 respectively.
\item
  Question setter of Overnite, a coding event of Kshitij-2011 (Asia's
  largest technomanagement fest), affiliated by ACM.
\item
  Secured Second position in Yahoo HackU 2012 at IIT Kharagpur.
\end{itemize}

% \end{document}

%%% Local Variables:
%%% mode: latex
%%% TeX-master: "grant"
%%% End:





\end{document}

%%% Local Variables:
%%% mode: latex
%%% TeX-master: t
%%% End:
