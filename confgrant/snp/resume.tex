% \documentclass[]{article}
% \usepackage{lmodern}
% \usepackage{amssymb,amsmath}
% \usepackage{ifxetex,ifluatex}
% \usepackage{fixltx2e} % provides \textsubscript
% \ifnum 0\ifxetex 1\fi\ifluatex 1\fi=0 % if pdftex
%   \usepackage[T1]{fontenc}
%   \usepackage[utf8]{inputenc}
% \else % if luatex or xelatex
%   \ifxetex
%     \usepackage{mathspec}
%   \else
%     \usepackage{fontspec}
%   \fi
%   \defaultfontfeatures{Ligatures=TeX,Scale=MatchLowercase}
% \fi
% % use upquote if available, for straight quotes in verbatim environments
% \IfFileExists{upquote.sty}{\usepackage{upquote}}{}
% % use microtype if available
% \IfFileExists{microtype.sty}{%
% \usepackage{microtype}
% \UseMicrotypeSet[protrusion]{basicmath} % disable protrusion for tt fonts
% }{}
% \usepackage{hyperref}
% \hypersetup{unicode=true,
%             pdfborder={0 0 0},
%             breaklinks=true}
% \urlstyle{same}  % don't use monospace font for urls
% \IfFileExists{parskip.sty}{%
% \usepackage{parskip}
% }{% else
% \setlength{\parindent}{0pt}
% \setlength{\parskip}{6pt plus 2pt minus 1pt}
% }
% \setlength{\emergencystretch}{3em}  % prevent overfull lines
% \providecommand{\tightlist}{%
%   \setlength{\itemsep}{0pt}\setlength{\parskip}{0pt}}
% \setcounter{secnumdepth}{0}
% % Redefines (sub)paragraphs to behave more like sections
% \ifx\paragraph\undefined\else
% \let\oldparagraph\paragraph
% \renewcommand{\paragraph}[1]{\oldparagraph{#1}\mbox{}}
% \fi
% \ifx\subparagraph\undefined\else
% \let\oldsubparagraph\subparagraph
% \renewcommand{\subparagraph}[1]{\oldsubparagraph{#1}\mbox{}}
% \fi

% \date{}

% \begin{document}

\section{Resume}
\section*{Rahul Chatterjee}\label{rahul-chatterjee}
% {\bf\large Name:} Rahul Chatterjee 
\subsection*{Contact Information:}\label{contact-information}
\begin{itemize}
\tightlist
\item
  Mobile: +1-(414)-877-3378,
\item
  E-mail: rahul@cs.cornell.edu,
\item
  Web: \url{https://www.cs.cornell.edu/~rahul/}
\item
  Address: Cornell Tech, 111 8th Ave \#1202, New York, NY 10011
\end{itemize}

\subsection*{Research Interests}\label{research-interests}
Computer Security

\subsection*{Education}\label{education}

\begin{itemize}
\tightlist
\item
  Cornell University, PhD, Computer Science
\item
  University of Wisconsin---Madison, Masters, Computer Sciences
\item IIT Kharagpur, BTech, Computer
  Science and Engineering
\end{itemize}

\subsection*{Industry Experience}\label{work-experience}
\begin{itemize}
\item
  \textbf{Microsoft Research Technologies}, (Redmond, USA) (June 2015 - August 2015)\\
  Project: Analyzing Crypto Board Data to Infer Common Security
  Engineering Problems.
\item
  \textbf{Two Roads Technology Solutions} (Bangalore, India) (June 2012 - May 2013)\\
  Job Title: Software Developer and Quantitative Analyst
\item
  \textbf{Adobe Systems India Pvt. Ltd.} (Noida, India) (May 2011 - July 2011)\\
  Project: Generating Smart Tags for Images from Meta-data and Web
\end{itemize}

\renewcommand\refname{{\large Publications}}
\begin{thebibliography}{0}
\bibitem{pwtypo16}
  Rahul Chatterjee, Anish Athalye, Devdatta Akhawe, Ari Juels, Thomas
  Ristenpart, \textit{pASSWORD tYPOS and How to Correct Them Securely}, Submitted
  to 37th IEEE Symposium on Security and Privacy 2016 (Oakland).
\bibitem{pythia15}
  Adam Everspaugh, Rahul Chatterjee, Samuel Scott, Air Juels, Thomas
  Ristenpart, \textit{The Pythia PRF Service}, USENIX Security 2015.
\bibitem{nocrack15}
  Rahul Chatterjee, Joseph Bonneau, Ari Juels, Thomas Ristenpart,
  \textit{Cracking-Resistant Password Vaults using Natural Language Encoders},
  36th IEEE Symposium on Security and Privacy 2015 (Oakland).
\end{thebibliography}

% \subsection*{Notable Projects:}\label{notable-projects}
% \begin{itemize}
% \item
%   \textbf{pASSWORD tYPOS}\\
%   \begin{itemize}
%   \item Goal: Logging in using passwords often gets rejected due to
%     small typographical errors in the entered password, such as typing
%     `pASSWORD' instead of `Password'.
%   \item
%     I analyzed to what extent typos in password effect the usability using
%     Amazon Mechanical Turk and instrumenting Dropbox login server.
%   \item
%     I also analyzed the security degradation (if any) in allowing small
%     typos in passwords. Theoretically I showed that we can allow typos in
%     passwords during logging in without degradation of security provided
%     the distribution of passwords is available.

%   \end{itemize}
% \item
%   \textbf{Cracking Resistant Password Vault}\\
%   Goal: Encryption scheme for password vaults s.t. decryption with wrong
%   key should generate plausible looking but decoy vaults, forcing the
%   attacker to go for online verification.
% \item
%   First ever real world application of Honey Encryption. Used existing
%   password cracking methodologies in defense of password vaults to
%   create effective distribution transforming en- coder.
% \item
%   Showed effective attacks against state of the art approaches, e.g.,
%   Kamouflage by Bojinov et al.
% \item
%   \textbf{The Pythia PRF Services}\\
%   Goal: Create a remote PRF service that can be used for password
%   hardening with some added features, such as, the PRF only has access
%   to the blinded version of the password. One can rotate the secret key
%   of the PRF and update the stored hash values of the passwords without
%   access to the plaintext password.
% \item
%   Along with password generation, we also found lots of interesting
%   application like forward secure email system, secure effacing of
%   encrypted hard drive etc.
% \item
%   \textbf{Simultaneous Localization and Mapping using Relational
%   Trees}\\
% \item
%   Project Sponsored by, European Aeronautic Defence and Space Company
%   (EADS), Germany.
% \item
%   Developed an unsupervised method of learning with which a robot, which
%   is flying over an unknown region and taking snaps periodically, can
%   build a semantic map of the region and localize its position on the
%   map using those pictures.
% \item
%   Implemented the software using C, C++, OpenCV and built a front end
%   using Java.
% \item
%   \textbf{Information Retrieval and Natural Language Processing.}\\
%   Goal: Tried to correct and improve fluency of Machine generated text
%   or second language writer's texts with minimum language dependent
%   information.
% \item
%   This project was selected one of the best four BTech projects in the
%   department.
% \item
%   \textbf{Creating Artistic Effects on Image using Random Digital
%   Curve}\\
% \item
%   Developed algorithm for generating irreducible simple random digital
%   curves in a constrained domain. Drawing multiple of them on the edge
%   of binary image and then setting the intensity of each pixel
%   proportional to the number of times it was visited by the curves give
%   nice real pencil sketch effect.
% \item
%   This work is published in CAIP-2011, Seville, Spain.
% \end{itemize}

\subsection*{Academic Achievements Awards and
Scholarships}\label{academic-achievements-awards-and-scholarships}

\begin{itemize}
\tightlist
\item
  Awarded special CS fellowship from the department of Computer
  Sciences, UW--Madison.
\item Secured rank 4th in the North Central region
  and 2nd in the University in ACM-ICPC regional contest 2013.
\item
  Awarded IMPRS-CS fellowship for Master Studies in Germany by
  MPI-Informatics, Saar- brucken, Germany, 2012.
\item
  Jagadish Bose National Science Talent Search(JBNSTS) scholarship,
  2008.
% \item
%   Selected for admission to B.Math in Chennai Mathematical
%   Institute(CMI), Chennai, 2008. • Ranked 12 in West Bengal Joint
%   Entrance Examination(WBJEE), 2008 (out of 50 thousand students)
\end{itemize}

\subsection*{Extra Curricular
Activities}\label{extra-curricular-activities}

\begin{itemize}
\tightlist
\item
  Selected for in Indian National Mathematics Olympiad(INMO) exam in
  2007.
\item
  Executive Committee member of Bengali Association of Madison (BAM), WI
  since April, 2014. BAM organizes lots of Indian cultural events in
  Madison.
\item
  Captain of Hall Mathematics Olympiad team and won Silver, Gold in
  Inter Hall Mathematics Olympiad in 2011 and 2012 respectively.
\item
  Question setter of Overnite, a coding event of Kshitij-2011 (Asia's
  largest technomanagement fest), affiliated by ACM.
\item
  Secured Second position in Yahoo HackU 2012 at IIT Kharagpur.
\end{itemize}

% \end{document}

%%% Local Variables:
%%% mode: latex
%%% TeX-master: "grant"
%%% End:
